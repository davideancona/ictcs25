\section{Partial derivatives of regular expressions}\label{sec:parDer}

\begin{figure}
 $$
  \begin{array}{c}
   \Rule{sym}{}{\sym\parDer{\sym}\eps}\quad
   \Rule{l-cat}{\re_0\parDer{\sym}\re'_0}{\re_0\catop\re_1\parDer{\sym}\re'_0\catop\re_1}{}\quad
   \Rule{r-cat}{\re_1\parDer{\sym}\re'_1}{\re_0\catop\re_1\parDer{\sym}\re'_1}{\hasEps{\re_0}=\eps} \\[4ex]
   \Rule{l-or}{\re_0\parDer{\sym}\re'_0}{\re_0\orop\re_1\parDer{\sym}\re'_0}{} \quad
   \Rule{r-or}{\re_1\parDer{\sym}\re'_1}{\re_0\orop\re_1\parDer{\sym}\re'_1}{} \quad
   \Rule{star}{\re\parDer{\sym}\re'}{\starop{\re}\parDer{\sym}\re'\starop{\re}}{}
  \end{array}
 $$
 \caption{Transition system defining the partial derivatives of regular expressions}
 \label{parDer}
\end{figure}

\begin{definition}
 \[
  \begin{array}{l}
   \\
   \re\parDer{\emptyWord}\re                                                                                        \\
   \re_0\parDer{\sym\strcatop\word}\re_1 \mbox{ iff } \re_0\parDer{\sym}{}\re \mbox{ and } \re\parDer{\word}{}\re_1 \\
   \opSem{\re}{\parDer{}}=\{\word\in\wordUniv \mid \mbox{there exists } \re' \mbox{ s.t. } \re\parDer{\word}\re' \mbox{ and } \hasEps{\re'}=\eps\}
  \end{array}
 \]
\end{definition}

\begin{definition}
 Let $\reSet_1$ and $\reSet_2$ be sets of regular expressions.
 \[
  \begin{array}{l}
   \reSet_1\derSet{\sym}\reSet_2 \mbox{ iff } \reSet_2=\{\re_1 \mid \mbox{ there exists } \re_0\in\reSet_1 \mbox{ s.t. } \re_0\parDer{\sym}{}\re_1 \} \\
   \hasEps{\reSet}=\eps \mbox{ iff there exists } \re\in\reSet \mbox{ s.t. } \hasEps{\re}                                                             \\
   \re_0\derSet{\sym\word}\re_1 \mbox{ iff } \re_0\derSet{\sym}\re \mbox{ and } \re\derSet{\word}\re_1                                                \\
   \re\derSet{\emptyWord}{}\re                                                                                                                        \\
   \opSem{\re}{\derSet{}}=\{\word\in\wordUniv \mid \mbox{there exists } \reSet \mbox{ s.t. } \{\re\}\derSet{\word}\reSet \mbox{ and } \hasEps{\reSet}=\eps\}
  \end{array}
 \]
\end{definition}

\begin{theorem}
 For all regular expressions $\re$
 \[
  \opSem{\re}{\parDer{}{}}=\opSem{\re}{\derSet{}}
 \]
\end{theorem}

\begin{theorem}
 For all regular expressions $\re$
 \[
  \opSem{\re}{\derSet{}}=\opSem{\re}{\der{}}
 \]
\end{theorem}

\begin{figure}
 $$
  \begin{array}{c}
   \theight{\eps}=\theight{\sym}=0                                                                                             \\
   \theight{\re_0\catop\re_1}=\theight{\re_0\orop\re_1}=\theight{\re_0\shuffleop\re_1}=1+\max(\theight{\re_0},\theight{\re_1}) \\
   \theight{\starop{\re_0}}=1+\theight{\re_0}
  \end{array}
 $$
 \caption{Definition of the size of regular expressions}
 \label{fig:size}
\end{figure}

\begin{figure}
 $$
  \begin{array}{l}
   % \gtFun{\re_0}{\re_1}=
   % \begin{cases}
   %   1, & \text{if } \theight{\re_0} > \theight{\re_1} \\
   %   0, & \text{ otherwise}
   % \end{cases} \quad
   \geqFun{\re_0}{\re_1}=
   \begin{cases}
    1, & \text{if } \theight{\re_0} \geq \theight{\re_1} \\
    0, & \text{ otherwise}
   \end{cases}                                                                                   \\[4ex]
   \incFun{\sym}{\emptyPath}=0\quad \incFun{\eps}{\emptyPath}=0                                                                           \\[2ex]
   \incFun{\re_0\catop\re_1}{0\pathCons\tpath}=\geqFun{\re_0}{\re_1}\cdot\max(\incFun{\re_0}{\tpath},\theight{\re_1}-\theight{\re_0})     \\[2ex]
   \incFun{\re_0\catop\re_1}{1\pathCons\tpath}=\incFun{\re_1}{\tpath}-1-\geqFun{\re_0}{\re_1}(\theight{\re_0}-\theight{\re_1})            \\[2ex]
   \incFun{\re_0\orop\re_1}{0\pathCons\tpath}=\incFun{\re_0}{\tpath}-1-\geqFun{\re_1}{\re_0}(\theight{\re_1}-\theight{\re_0})             \\[2ex]
   \incFun{\re_0\orop\re_1}{1\pathCons\tpath}=\incFun{\re_1}{\tpath}-1-\geqFun{\re_0}{\re_1}(\theight{\re_0}-\theight{\re_1})             \\[2ex]
   \incFun{\starop{\re_0}}{0\pathCons\tpath}=1 %%\mbox{ if } \incFun{\re}{\tpath}=\incConst
   \\[2ex]
   \incFun{\re_0\shuffleop\re_1}{0\pathCons\tpath}=\geqFun{\re_0}{\re_1}\cdot\max(\incFun{\re_0}{\tpath},\theight{\re_1}-\theight{\re_0}) \\[2ex]
   \incFun{\re_0\shuffleop\re_1}{1\pathCons\tpath}=\geqFun{\re_1}{\re_0}\cdot\max(\incFun{\re_1}{\tpath},\theight{\re_0}-\theight{\re_1})
  \end{array}
 $$
 \caption{Definition of the increment function $\incFun{\re}{\tpath}$}
 \label{fig:inc-pred}
\end{figure}

% \paragraph{Remark}: some of the cases which define $\incSym$ use recursion though this is not needed to compute its
% result and prove the following claims.
% However, in such a way the definition is more intutive and clear and the function is well-defined only on correct paths.

\begin{lemma}\label{red-step}
 If $\re\parDer{\sym}{}\re'$, then either $\re'=\eps$ or $\re'=\re_0\catop\re_1$ for some $\re_0$ and $\re_1$.
 \begin{proof}
  By induction on the rules in \cref{fig:reg-exp-nodet-der} and simple case analysis on their conclusions.
 \end{proof}
\end{lemma}
\begin{lemma}\label{inc}
 For all $\re$ and $\tpath$, $\incFun{\re}{\tpath}\leq 1$.
 \begin{proof}
  It follows by definition of $\gtSym$ and direct induction on the definition of $\incFun{\re}{\tpath}$.
 \end{proof}
\end{lemma}

\begin{theorem}\label{inc-bound}
 If $\re\parDer{\sym}{\tpath}\re'$, then $\theight{\re'}\leq \theight{\re}+\incFun{\re}{\tpath}$.
 \begin{proof}
  By induction on the derivation tree for $\re\parDer{\sym}{\tpath}\re'$.
  \begin{description}
   \item[base case:] The only base rule is for $\re=\sym$, $\tpath=\emptyPath$, and $\re'=\eps$.
    Hence, $\incFun{\sym}{\emptyPath}=0$ and
    $\theight{\eps}=0=\theight{\sym}\leq\theight{\sym}+\incFun{\sym}{\emptyPath}$.
   \item[inductive step:] the proof proceeds by case analysis on the rule applied
    at the root of the derivation tree, that is, by case analysis on $\re$ and $\tpath$.
    \begin{itemize}
     \item $\re=\re_0\catop\re_1$, $\tpath=0\pathCons\tpath'$. By definition
           of $\parDer{\sym}{\tpath}$, $\re_0\parDer{\sym}{\tpath'}\re'_0$ and $\re'=\re'_0\cdot\re_1$.
           By inductive hypothesis $\theight{\re'_0}\leq \theight{\re_0}+\incFun{\re_0}{\tpath'}$.

           The following inequality holds:
           \begin{align}
             &  & \theight{\re_1}=\theight{\re_0}+\theight{\re_1}-\theight{\re_0}\leq
            \theight{\re_0}+\max(\incFun{\re_0}{\tpath'},\theight{\re_1}-\theight{\re_0}) \label{diseq1} \\
             &  & \theight{\re'_0\catop\re_1}=\max(\theight{\re'_0},\theight{\re_1})+1
            \leq\max(\theight{\re_0}+\incFun{\re_0}{\tpath'},\theight{\re_1})+1 \label{diseq2}
           \end{align}

           From \cref{diseq1} and \cref{diseq2},
           $$
            \theight{\re'_0\catop\re_1}\leq
            \max(\theight{\re_0}+\incFun{\re_0}{\tpath'},
            \theight{\re_0}+\max(\incFun{\re_0}{\tpath'},\theight{\re_1}-\theight{\re_0}))+1
           $$
           hence
           \begin{align}
             &  & \theight{\re'_0\catop\re_1}\leq
            \theight{\re_0}+\max(\incFun{\re_0}{\tpath'},\theight{\re_1}-\theight{\re_0})+1 \label{diseq3}
           \end{align}

           If $\theight{\re_0}\geq\theight{\re_1}$, then $\theight{\re_0\catop\re_1}=\theight{\re_0}+1$, and
           $\gtFun{\re_0}{\re_1}=1$, hence
           $$\incFun{\re_0\catop\re_1}{0\pathCons\tpath'}=\max(\incFun{\re_0}{\tpath'},\theight{\re_1}-\theight{\re_0}).$$

           From \cref{diseq3},
           $$
            \theight{\re'_0\catop\re_1}\leq\theight{\re_0\catop\re_1}+\max(\incFun{\re_0}{\tpath'},\theight{\re_1}-\theight{\re_0})
            =\theight{\re_0\catop\re_1}+\incFun{\re_0\catop\re_1}{0\pathCons\tpath'}.$$

           If $\theight{\re_1}>\theight{\re_0}$, then $\theight{\re_1}-\theight{\re_0}\geq 1\geq \incFun{\re_0}{\tpath'}$
           by \cref{inc}, $\theight{\re_0\catop\re_1}=\theight{\re_1}+1$, and $\gtFun{\re_0}{\re_1}=0$, hence
           $\incFun{\re_0\catop\re_1}{0\pathCons\tpath'}=0$.

           From \cref{diseq3},
           $$
            \theight{\re'_0\catop\re_1}\leq\theight{\re_0}+\theight{\re_1}-\theight{\re_0}+1=\theight{\re_1}+1=
            \theight{\re_0\catop\re_1}=\theight{\re_0\catop\re_1}+\incFun{\re_0\catop\re_1}{0\pathCons\tpath'}.$$

     \item $\re=\re_0\catop\re_1$, $\tpath=1\pathCons\tpath'$. By definition
           of $\parDer{\sym}{\tpath}$, $\re_1\parDer{\sym}{\tpath'}\re'_1$ and $\re'=\re'_1$.
           By inductive hypothesis $\theight{\re'_1}\leq \theight{\re_1}+\incFun{\re_1}{\tpath'}$.

           If $\theight{\re_1}\geq\theight{\re_0}$, then $\theight{\re_1}=\theight{\re_0\catop\re_1}-1$ and
           $\gtFun{\re_0}{\re_1}=0$, hence
           $\incFun{\re_0\catop\re_1}{1\pathCons\tpath'}$=$\incFun{\re_1}{\tpath'}-1$,  and therefore
           $$
            \theight{\re'_1}\leq\theight{\re_1}+\incFun{\re_1}{\tpath'}=
            \theight{\re_0\catop\re_1}-1+\incFun{\re_1}{\tpath'}=
            \theight{\re_0\catop\re_1}+\incFun{\re_0\catop\re_1}{1\pathCons\tpath'}.
           $$

           If $\theight{\re_0}>\theight{\re_1}$, then $\theight{\re_0}=\theight{\re_0\catop\re_1}-1$ and
           $\gtFun{\re_0}{\re_1}=1$, hence
           $\incFun{\re_0\catop\re_1}{1\pathCons\tpath'}$=$\incFun{\re_1}{\tpath'}-2$,  and therefore
           $$
            \theight{\re'_1}\leq\theight{\re_1}+\incFun{\re_1}{\tpath'}\leq
            \theight{\re_0}-1+\incFun{\re_1}{\tpath'}=
            \theight{\re_0\catop\re_1}-2+\incFun{\re_1}{\tpath'}=$$
           $$
            \theight{\re_0\catop\re_1}+\incFun{\re_0\catop\re_1}{1\pathCons\tpath'}.
           $$

     \item $\re=\re_0\orop\re_1$, $\tpath=0\pathCons\tpath'$. By definition
           of $\parDer{\sym}{\tpath}$, $\re_0\parDer{\sym}{\tpath'}\re'_0$ and $\re'=\re'_0$.
           By inductive hypothesis $\theight{\re'_0}\leq \theight{\re_0}+\incFun{\re_0}{\tpath'}$.

           By symmetry, the proof is analogous to the
           previous case.
     \item $\re=\re_0\orop\re_1$, $\tpath=1\pathCons\tpath'$. By definition
           of $\parDer{\sym}{\tpath}$, $\re_1\parDer{\sym}{\tpath'}\re'_1$ and $\re'=\re'_1$.
           By inductive hypothesis $\theight{\re'_1}\leq \theight{\re_1}+\incFun{\re_1}{\tpath'}$.

           By symmetry, the proof is analogous to the
           previous case.

     \item $\re=\starop{\re_0}$, $\tpath=0\pathCons\tpath'$. By definition
           of $\parDer{\sym}{\tpath}$, $\re_0\parDer{\sym}{\tpath'}\re'_0$ and $\re'=\re'_0\catop\starop{\re_0}$.
           By inductive hypothesis $\theight{\re'_0}\leq \theight{\re_0}+\incFun{\re_0}{\tpath'}$, hence
           by definition of $\theight{\re'_0\catop\starop{\re_0}}$, $\theight{\starop{\re_0}}$, $\incFun{\starop{\re_0}}{0\pathCons\tpath'}$ and \cref{inc}
           $$
            \theight{\re'_0\catop\starop{\re_0}}=\max(\theight{\re'_0},\theight{\re_0}+1)+1\leq
            \max(\theight{\re_0}+\incFun{\re_0}{\tpath'},\theight{\re_0}+1)+1=$$
           $$\theight{\re_0}+1+1=
            \theight{\starop{\re_0}}+1=\theight{\starop{\re_0}}+\incFun{\starop{\re_0}}{0\pathCons\tpath'}
           $$
    \end{itemize}
  \end{description}
 \end{proof}
\end{theorem}

\begin{theorem}\label{zero-inc}
 If $\re\parDer{\sym}{}\re'$ and $\incFun{\re'}{\tpath}=k$, then $k\leq 0$.
 \begin{proof}
  By virtue of \cref{red-step} $\re'=\eps$ or $\re'=\re_0\catop\re_1$ for some $\re_0,\re_1$.

  If $\re'=\eps$, then by definition $\tpath=\emptyPath$ and $k=0$.

  If $\re'=\re_0\catop\re_1$, then the proof proceeds by induction on the definition of $\incFun{\re'}{\tpath}$.

  Only two different cases can occur: either $\tpath=0\pathCons\tpath'$ or $\tpath=1\pathCons\tpath'$ for some  $\tpath'$.

  In the former case $\incFun{\re_0\catop\re_1}{\tpath}=\geqFun{\re_0}{\re_1}\cdot\max(\incFun{\re_0}{\tpath'},\theight{\re_1}-\theight{\re_0})$.

  If $\theight{\re_0}\geq\theight{\re_1}$ then $\incFun{\re_0\catop\re_1}{\tpath}=\max(\incFun{\re_0}{\tpath'},\theight{\re_1}-\theight{\re_0})$.
  By induction $\incFun{\re_0}{\tpath'}\leq 0$, furthermore $\theight{\re_1}-\theight{\re_0}<0$, therefore $\max(\incFun{\re_0}{\tpath'},\theight{\re_1}-\theight{\re_0})\leq 0$.

  If $\theight{\re_0}<\theight{\re_1}$ then trivially $\incFun{\re_0\catop\re_1}{\tpath}=0$.


 \end{proof}
\end{theorem}
\section{Partial derivatives of regular expressions}\label{sec:regexp-par}

\begin{figure}
 $$
  \begin{array}{c}
   \simpleRule{}{\sym\parDer{\sym}\eps}{}\quad
   \simpleRule{}{\all\parDer{\sym}\all}{}                                                                                   \\[4ex]
   \simpleRule{\re_0\parDer{\sym}\re'_0}{\re_0\catop\re_1\parDer{\sym}\re'_0\catop\re_1}{}\quad
   \simpleRule{\re_0\noParDer{\sym}\quad\re_1\parDer{\sym}\re'_1}{\re_0\catop\re_1\parDer{\sym}\re'_1}{\hasEps{\re_0}=\eps} \\[4ex]
   \simpleRule{\re_0\parDer{\sym}\re'_0}{\re_0\orop\re_1\parDer{\sym}\re'_0}{}\quad
   \simpleRule{\re_0\noParDer{\sym}\quad\re_1\parDer{\sym}\re'_1}{\re_0\orop\re_1\parDer{\sym}\re'_1}{}\quad
   \simpleRule{\re_0\parDer{\sym}\re'_0\quad\re_1\parDer{\sym}\re'_1}{\re_0\andop\re_1\parDer{\sym}\re'_0\andop\re'_1}{}    \\[4ex]
   \simpleRule{\re_0\parDer{\sym}\re'_0}{\re_0\shuffleop\re_1\parDer{\sym}\re'_0\shuffleop\re_1}{}\quad
   \simpleRule{\re_0\noParDer{\sym}\quad\re_1\parDer{\sym}\re'_1}{\re_0\shuffleop\re_1\parDer{\sym}\re_0\shuffleop\re'_1}{}\quad
   \simpleRule{\re\parDer{\sym}\re'}{\starop{\re_0}\parDer{\sym}\re'\starop{\re_0}}{}                                       \\[4ex]
   \simpleRule{}{\eps\noParDer{\sym}}{}\quad
   \simpleRule{}{\sym\noParDer{\asym}}{\asym\neq\sym}\quad
   \simpleRule{}{\none\noParDer{\sym}}{}                                                                                    \\[4ex]
   \simpleRule{\re_0\noParDer{\sym}}{\re_0\catop\re_1\noParDer{\sym}}{\hasEps{\re_0}=\none}\quad
   \simpleRule{\re_0\noParDer{\sym}\quad\re_1\noParDer{\sym}}{\re_0\catop\re_1\noParDer{\sym}}{} \quad
   \simpleRule{\re_0\noParDer{\sym}\quad\re_1\noParDer{\sym}}{\re_0\orop\re_1\noParDer{\sym}}{}                             \\[4ex]
   \simpleRule{\re_0\noParDer{\sym}}{\re_0\andop\re_1\noParDer{\sym}}{}\quad
   \simpleRule{\re_1\noParDer{\sym}}{\re_0\andop\re_1\noParDer{\sym}}{}\quad
   \simpleRule{\re_0\noParDer{\sym}\quad\re_1\noParDer{\sym}}{\re_0\shuffleop\re_1\noParDer{\sym}}{}\quad
   \simpleRule{\re\noParDer{\sym}}{\starop{\re_0}\noParDer{\sym}}{}
  \end{array}
 $$
 \caption{Partial derivatives of regular expressions, deterministic definition}
 \label{fig:reg-exp-par-der}
\end{figure}


\begin{theorem}
 For all regular expressions $\re$
 \[
  \opSem{\re}{\parDer{}}\subseteq\opSem{\re}{\parDer{}{}}
 \]
\end{theorem}

\begin{theorem}
 There exist regular expressions $\re$ s.t.
 \[
  \opSem{\re}{\parDer{}{}}\not\subseteq\opSem{\re}{\parDer{}}
 \]
\end{theorem}

\begin{definition}
 \[
  \begin{array}{l}
   \first{\none}=\emptyset\quad \first{\all}=\symAlph \quad \first{\eps}=\emptyset \quad \first{\sym}=\{\sym\}                                      \\[2ex]
   \first{\re_0\catop\re_1}=\first{\re_0}\cup(\sem{\hasEps{\re_0}}\catop\first{\re_1}) \quad \first{\re_0\orop\re_1}=\first{\re_0}\cup\first{\re_1} \\[2ex]
   \first{\re_0\andop\re_1}=\first{\re_0}\cap\first{\re_1} \quad \first{\re_0\shuffleop\re_1}=\first{\re_0}\cup\first{\re_1} \quad
   \first{\starop{\re_0}}=\first{\re}
  \end{array}
 \]
\end{definition}

\begin{theorem}
 For all $\sym$, $\re$,
 \[
  \sym\in\first{\re} \mbox{ iff there exists } \re' \mbox{ s.t. } \re\parDer{\sym}\re'
 \]
\end{theorem}

\begin{theorem}
 For all $\sym$, $\word$, $\re$,
 \[
  \mbox{ if } \sym\catop\word\in\sem{\re} \mbox{ then } \sym\in\first{\re}
 \]
\end{theorem}

\begin{theorem}
 For all $\sym$, $\re$,
 \[
  \mbox{ if } \sym\in\first{\re} \mbox{ and } \sem{\re}\neq\emptyset \mbox{ then there exists } \word \mbox{ s.t. } \sym\word\in\sem{\re}
 \]
\end{theorem}

