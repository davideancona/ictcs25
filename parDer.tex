\section{Partial derivatives of regular expressions}\label{sec:parDer}

\begin{figure}
 $$
  \begin{array}{c}
   \Rule{sym}{}{\sym\parDer{\sym}\eps}\quad
   \Rule{l-cat}{\re_0\parDer{\sym}\re'_0}{\re_0\catop\re_1\parDer{\sym}\re'_0\catop\re_1}{}\quad
   \Rule{r-cat}{\re_1\parDer{\sym}\re'_1}{\re_0\catop\re_1\parDer{\sym}\re'_1}{\hasEps{\re_0}=\eps} \\[4ex]
   \Rule{l-or}{\re_0\parDer{\sym}\re'_0}{\re_0\orop\re_1\parDer{\sym}\re'_0}{} \quad
   \Rule{r-or}{\re_1\parDer{\sym}\re'_1}{\re_0\orop\re_1\parDer{\sym}\re'_1}{} \quad
   \Rule{star}{\re\parDer{\sym}\re'}{\starop{\re}\parDer{\sym}\re'\catop\starop{\re}}{}
  \end{array}
 $$
 \caption{Transition system defining the partial derivatives of regular expressions}
 \label{fig:parDer}
\end{figure}

~\Cref{fig:parDer} contains the transition rules defining the partial derivatives of regular expressions \wrt~ a single symbol.
The definition follows the work of Antimirov \cite{Antimirov96}, but
as happens for the definition of derivatives, our presentation based on a labelled transition system deviates from that of Antimirov. More importantly, here we deliberately define a non-deterministic system, where a single transition step yields a single non-empty partial derivative among all possible ones, while in Antimirov's definition the set of all partial derivatives is returned.

Compared to the rules in \cref{fig:der}, here we have removed all rules which return the empty derivative. Therefore, there are cases where the partial derivative of a regular expression \wrt\ a symbol (or word) does not exist. This is useful when partial derivatives are used for RV, because in this way errors can be detected with no further checks on the partial derivative: if no moves can be taken, then the partial derivative is empty and  the trace of events (that is, the word) considered so far cannot be the prefix of any correct trace, therefore a violation of the specification can be reported.

Another important difference \wrt\ the definitions in \cref{fig:der} regards the number of rules for concatenation and union. Instead of having a single rule which covers all cases by introducing the $\orop$ operator in the partial derivative, here cases are covered non-deterministically by two dual rules: \rn{l-cat}, \rn{r-cat} and \rn{l-or}, \rn{r-or}  for concatenation and union, respectively.
For instance, according to Antimirov's definition\footnote{See \cite{Antimirov96}, definition 2.8.},
the partial derivative of $\sym \asym \orop \sym\yasym$ \wrt~symbol $\sym $ returns the set of regular expressions $\{\eps\catop\asym,\eps\catop\yasym\}$ (that is, $\delta_\sym(\sym\asym \orop \sym\yasym)=\{\eps\catop\asym,\eps\catop\yasym\}$ with Antimirov's notation). This corresponds to the fact that there exist two possible transition steps from $\sym \asym \orop \sym\yasym$ labeled with $\sym $, namely,
$\sym \asym \orop \sym\yasym\parDer{\sym}\eps\catop\asym$ and $\sym \asym \orop \sym \yasym\parDer{\sym}\eps\catop\yasym$.

This difference is motivated by the main aim of our work: while Antimirov's approach uses partial derivatives for generating an NFA, here
we use them to define a non-deterministic transition system which already represents the corresponding NFA.

\Cref{def:parDer} is dual to \Cref{def:der}.
\begin{definition}\label{def:parDer}
 For all $\re,\re_0,\re_1\in\reSet$, $\sym\in\symAlph$, $\word\in\symAlph^*$
 \[
  \begin{array}{l}
   \\
   \re\parDer{\emptyWord}\re                                                                                       \qquad
   \re_0\parDer{\sym\strcons\word}\re_1 \mbox{ iff } \re_0\parDer{\sym}{}\re \mbox{ and } \re\parDer{\word}{}\re_1 \\
   \opSem{\re}{\parDer{}}=\{\word\in\symAlph^* \mid \mbox{there exists } \re'\in\reSet \mbox{ s.t. } \re\parDer{\word}\re' \mbox{ and } \hasEps{\re'}=\eps\}
  \end{array}
 \]
\end{definition}

The claims below are dual to \cref{theo:der}, and  directly follow from the results of Antimirov.

Let $\delta_\word(\re)$ denotes the set of all partial derivatives of $\re\in\reSet$ \wrt~ $\word\in\symAlph^*$:
\[
 \delta_\word(\re)=\{\re'\in\reSet \mid \re\parDer{\word}\re'\}.
\]

% \begin{definition}
%  Let $\reSet_1$ and $\reSet_2$ be sets of regular expressions.
%  \[
%   \begin{array}{l}
%    \reSet_1\derSet{\sym}\reSet_2 \mbox{ iff } \reSet_2=\{\re_1 \mid \mbox{ there exists } \re_0\in\reSet_1 \mbox{ s.t. } \re_0\parDer{\sym}{}\re_1 \} \\
%    \hasEps{\reSet}=\eps \mbox{ iff there exists } \re\in\reSet \mbox{ s.t. } \hasEps{\re}                                                             \\
%    \re_0\derSet{\sym\word}\re_1 \mbox{ iff } \re_0\derSet{\sym}\re \mbox{ and } \re\derSet{\word}\re_1                                                \\
%    \re\derSet{\emptyWord}{}\re                                                                                                                        \\
%    \opSem{\re}{\derSet{}}=\{\word\in\wordUniv \mid \mbox{there exists } \reSet \mbox{ s.t. } \{\re\}\derSet{\word}\reSet \mbox{ and } \hasEps{\reSet}=\eps\}
%   \end{array}
%  \]
% \end{definition}

\begin{theorem}\label[theorem]{theo:parDer}
 For all $\re\in\reSet$
 \[
  \begin{array}{l}
   \word^{-1}(\sem{\re})=\bigcup_{\re'\in\delta_\word(\re)}\sem{\re'} \\
   \opSem{\re}{\parDer{}{}}=\opSem{\re}{\der{}}
  \end{array}
 \]
\end{theorem}

\begin{corollary}\label[corollary]{cor:parDer}
 For all $\re\in\reSet$
 \[
  \opSem{\re}{\parDer{}{}}=\sem{\re}
 \]
\end{corollary}
% \begin{theorem}
%  For all regular expressions $\re$
%  \[
%   \opSem{\re}{\derSet{}}=\opSem{\re}{\der{}}
%  \]
% \end{theorem}

\subsection{Height of partial derivatives}\label{sec:height}
\begin{figure}
 $$
  \begin{array}{c}
   \theight{\eps}=\theight{\sym}=0                                                              \\
   \theight{\re_0\catop\re_1}=\theight{\re_0\orop\re_1}=\max(\theight{\re_0},\theight{\re_1})+1 \\
   \theight{\starop{\re}}=\theight{\re}+1
  \end{array}
 $$
 \caption{Definition of the height of regular expressions}
 \label{fig:height}
\end{figure}

\begin{figure}
 $$
  \begin{array}{l}
   % \gtFun{\re_0}{\re_1}=
   % \begin{cases}
   %   1, & \text{if } \theight{\re_0} > \theight{\re_1} \\
   %   0, & \text{ otherwise}
   % \end{cases} \quad
   \geqFun{\re_0}{\re_1}=
   \begin{cases}
    1, & \text{if } \theight{\re_0} \geq \theight{\re_1} \\
    0, & \text{if } \theight{\re_0} < \theight{\re_1}
   \end{cases} \\[4ex]
   \incFun{\sym}=\incFun{\eps}=0                                      \qquad
   \incFun{\re_0\catop\re_1}=\geqFun{\re_0}{\re_1}\cdot\incFun{\re_0} \qquad
   \incFun{\re_0\orop\re_1}=0                                         \qquad
   \incFun{\starop{\re}}=1
  \end{array}
 $$
 \caption{Definition of the function $\incFun{\re}$}
 \label{fig:incFun}
\end{figure}

In this section we investigate the upper bound on the hight of the partial derivatives of a given regular expression.

The standard definition of the height of a regular expression considered as a tree can be found in \cref{fig:height}.

The height of the partial derivative of an expression $\re$  \wrt~ a word $\word$ both depends on the shape of $\re$ and on $\word$.
Here we need to reason on any possible $\re$ and $\word$. Moreover, we aim at providing a general proof methodology which can be followed to prove results when considering also shuffle and the size of expressions.

To show a couple of examples let us consider the following reduction steps which compute the partial derivatives of $\starop{\sym}\catop\starop{\asym}$~
\wrt~ $\sym\asym$ and $\asym\asym$:
\begin{flushleft}
 $
  \begin{array}{l}
   \starop{\sym}\catop\starop{\asym}\parDer{\sym}(\eps\catop\starop{\sym})\catop\starop{\asym}\parDer{\asym}\eps\catop\starop{\asym} \\
   \starop{\sym}\catop\starop{\asym}\parDer{\asym}\eps\catop\starop{\asym}\parDer{\asym}\eps\catop\starop{\asym}                     \\
   \theight{\starop{\sym}\catop\starop{\asym}}=2\qquad \theight{(\eps\catop\starop{\sym})\catop\starop{\asym}}=3 \qquad \theight{\eps\catop\starop{\asym}}=2
  \end{array}
 $
\end{flushleft}
In the first example the height increases by one after the first step and decreases by one after the second, while in the second example the height of the expressions is unchanged.

What we are going to prove are the following main properties:
\begin{enumerate}
 \item the height of the partial derivative of a regular expression \wrt~ a symbol can increase \textbf{at most by one};
 \item the height of the partial derivative of a partial derivative \wrt~a non-empty word, \textbf{cannot} increase;
 \item from 1. and 2. and from the definition of partial derivative, one can deduce that if $\re\parDer{\word}\re'$, then $\theight{\re'}\leq\theight{\re}+1$.
\end{enumerate}
Although these properties can be proved in a direct way, they must be adapted when the size of expressions is considered.
Furthermore, they do not hold in this stronger form, when shuffle is added. Therefore, as motived above, we provide a more general proof strategy which works for both metrics on expressions, and can be adapted to the case of shuffle, where weaker claims hold.

In order to do that, \cref{fig:incFun} defines the function $\incFun{\re}$ which returns an upper bound for $\theight{\re'}-\theight{\re}$, where
$\re\parDer{\sym}\re'$, with $\re'\in\reSet$, $\sym\in\symAlph$.
That is, if $M\stackrel{\mathrm{def.}}{=}\max\{\theight{\re'}-\theight{\re} \mid \re\parDer{\sym}\re', \re'\in\reSet, \sym\in\symAlph\}$, then
a sound definition of $\incFun{\re}$ has to satisfy the constraint $M\leq\incFun{\re}$.

Before commenting the definition of $\incFun{\re}$, we introduce the following straightforward lemma.
\begin{lemma}\label[lemma]{lemma:bounds}
 For all $\re\in\reSet$
 \[0\leq\incFun{\re}\leq 1\]
\end{lemma}
\begin{proof}
 Directly by induction on the definition of $\incFun{\re}$.
\end{proof}

The constraint $0\leq\incFun{\re}$ is ensured by the definition of $\incFun{\re}$ because we are interested in investigating the upper bound\footnote{$M$ as defined above, can be an arbitrary negative integer. For instance, the partial derivatives of $\sym\orop\asym$ \wrt~a symbol all have height 0, hence $M=-1$ in this case.}. The constraint $\incFun{\re}\leq 1$ corresponds to the intuition shown above, that needs to be proved.

The definition of $\incFun{\re}$ is driven by the reduction rules in \cref{fig:parDer}. We comment only the non-trivial cases: if $\re=\re_0\orop\re_1$, then we expect $\theight{\re'_i}\leq\theight{\re_i}+1$, $i=0,1$,
hence $\theight{\re'_i}\leq\theight{\re}$ and $\incFun{\re}=0$;
if $\re=\starop{\re_0}$, then we expect $\theight{\re'_0}\leq\theight{\re_0}+1$,
hence $\theight{\re'_0\catop\starop{\re_0}}\leq\theight{\re}+1$ and $\incFun{\re}=1$. Finally, the definition for $\re=\re_0\catop\re_1$ requires more care:
if rule $\rn{l-cat}$ is applied, then $\theight{\re_0'\catop\re_1}=\theight{\re}+1$, but only when $\theight{\re_0'}=\theight{\re_0}+1$ and $\theight{\re_0}\geq\theight{\re_1}$, otherwise $\theight{\re_0'\catop\re_1}\leq\theight{\re}$; if rule $\rn{r-cat}$ is applied, then $\theight{\re'_1}\leq\theight{\re}$, as happens for rule $\rn{r-or}$. Therefore $\incFun{\re}=1$ iff $\incFun{\re_0}=1$ and $\theight{\re_0}\geq\theight{\re_1}$, that is, $\geqFun{\re_0}{\re_1}=1$. In all other cases, that is $\incFun{\re_0}=0$ or $\geqFun{\re_0}{\re_1}=0$, $\incFun{\re}=0$.

The following theorem proves the soundness of the definition of $\incFun{}$ \wrt~ its intended meaning. As a consequence, the height of the derivative of an expression $\re$ \wrt~ a symbol is always bounded by $\theight{\re}+1$.
\begin{theorem}\label[theorem]{theo:inc-bound}
 For all $\re,\re'\in\reSet$, $\sym\in\symAlph$, if $\re\parDer{\sym}\re'$, then $\theight{\re'}\leq \theight{\re}+\incFun{\re}$.
\end{theorem}
\begin{proof}
 By induction and case analysis on the rules defining $\re\parDer{\sym}\re'$.
 \begin{description}
  \item[base case:] The only base rule is \rn{sym}.
   Hence, $\re=\sym$, $\re'=\eps$, and
   $\theight{\eps}=0=\theight{\sym}=\theight{\sym}+0=\theight{\sym}+\incFun{\sym}$.
  \item[inductive step:] the proof proceeds by case analysis on the rule applied
   at the root of the derivation tree.
   \begin{itemize}
    \item $\Rule{l-cat}{\re_0\parDer{\sym}\re'_0}{\re_0\catop\re_1\parDer{\sym}\re'_0\catop\re_1}{}$\\[2ex]
          By inductive hypothesis $\theight{\re_0'}\leq \theight{\re_0}+\incFun{\re_0}$.
          We distinguish two cases:
          \begin{itemize}
           \item $\theight{\re_0}\geq\theight{\re_1}$:
                 $\theight{\re_0'\catop\re_1}\stackrel{\mathrm{def}}{=}\max(\theight{\re'_0},\theight{\re_1})+1\leq\max(\theight{\re_0}+\incFun{\re_0},\theight{\re_1})+1\leq\max(\theight{\re_0},\theight{\re_1})+1+\incFun{\re_0}=\theight{\re_0\catop\re_1}+\incFun{\re_0}=\theight{\re_0\catop\re_1}+\incFun{\re_0\catop\re_1}$, by the inductive hypothesis, the definition of $\max$, $\theight{\ }$, $\incFun{}$, and $\geqSym()$, the assumption $\theight{\re_0}\geq\theight{\re_1}$, and \cref{lemma:bounds}.

           \item $\theight{\re_0}<\theight{\re_1}$:
                 $\theight{\re_0'\catop\re_1}\stackrel{\mathrm{def}}{=}\max(\theight{\re'_0},\theight{\re_1})+1\leq\max(\theight{\re_0}+\incFun{\re_0},\theight{\re_1})+1=\theight{\re_0\catop\re_1}=\theight{\re_0\catop\re_1}+\incFun{\re_0\catop\re_1}$, by the inductive hypothesis, the definition of $\max$, $\theight{\ }$, $\incFun{}$, and $\geqSym()$, the assumption $\theight{\re_0}<\theight{\re_1}$, and \cref{lemma:bounds}.
          \end{itemize}

    \item $\Rule{r-cat}{\re_1\parDer{\sym}\re'_1}{\re_0\catop\re_1\parDer{\sym}\re'_1}{\hasEps{\re_0}=\eps}$\\[2ex]
          By inductive hypothesis $\theight{\re_1'}\leq \theight{\re_1}+\incFun{\re_1}$.

          Therefore, $\theight{\re_1'}\leq \theight{\re_1}+\incFun{\re_1}\leq\theight{\re_1}+1\leq\theight{\re_0\catop\re_1}\leq\theight{\re_0\catop\re_1}+\incFun{\re_0\catop\re_1}$, by the inductive hypothesis, the definition of $\max$, $\theight{\ },$ and $\incFun{}$, and \cref{lemma:bounds}.

    \item $\Rule{l-or}{\re_0\parDer{\sym}\re'_0}{\re_0\orop\re_1\parDer{\sym}\re'_0}{}$\\[2ex]
          By inductive hypothesis $\theight{\re_0'}\leq \theight{\re_0}+\incFun{\re_0}$.

          Therefore, $\theight{\re_0'}\leq \theight{\re_0}+\incFun{\re_0}\leq\theight{\re_0}+1\leq\theight{\re_0\orop\re_1}=\theight{\re_0\orop\re_1}+\incFun{\re_0\orop\re_1}$, by the inductive hypothesis, the definition of $\max$, $\theight{\ },$ and $\incFun{}$, and \cref{lemma:bounds}.
    \item rule \rn{r-or} is symmetric to  rule \rn{l-or}.
    \item $\Rule{star}{\re\parDer{\sym}\re'}{\starop{\re}\parDer{\sym}\re'\catop\starop{\re}}{}$\\[2ex]
          By inductive hypothesis $\theight{\re'}\leq \theight{\re}+\incFun{\re}$.

          Therefore, $\theight{\re'\catop\starop{\re}}\stackrel{\mathrm{def}}{=}\max(\theight{\re'},{\theight{\starop{\re}}})+1\leq \max(\theight{\re}+\incFun{\re},{\theight{\re}}+1)+1=\theight{\re}+1+1=\theight{\starop{\re}}+1=\theight{\starop{\re}}+\incFun{\starop{\re}}$, by the inductive hypothesis, the definition of $\max$, $\theight{\ },$ and $\incFun{}$, and \cref{lemma:bounds}.
   \end{itemize}
 \end{description}
\end{proof}

The following corollary can be directly derived from \cref{theo:inc-bound} and \cref{lemma:bounds}.
\begin{corollary}\label[corollary]{cor:bound}
 For all $\re,\re'\in\reSet$, and $\sym\in\symAlph$, if $\re\parDer{\sym}\re'$, then $\theight{\re'}\leq\theight{\re}+1$.
\end{corollary}
~\Cref{theo:inc-bound} shows that the height of the derivative of an expression $\re$ \wrt~ a symbol is bounded by $\theight{\re}+1$.
The following results prove a strong property: the height of the partial derivative of a partial derivative cannot increase; that is, after the first reduction step, the height of the derivative of an expression $\re$ \wrt~ a symbol is bounded by $\theight{\re}$.
Therefore, one can conclude that the height of the derivative of an expression $\re$ \wrt~ an arbitrary  word (not just a symbol) is bounded by $\theight{\re}+1$.
\begin{lemma}\label[lemma]{lemma:zero-inc}
 For all $\re,\re'\in\reSet$, and $\sym\in\symAlph$, if $\re\parDer{\sym}\re'$, then $\incFun{\re'}=0$.
\end{lemma}
\begin{proof}
 By induction and case analysis on the rules defining $\re\parDer{\sym}\re'$.
 \begin{description}
  \item[base case:] The only base rule is \rn{sym}.
   Hence, $\re=\sym$, $\re'=\eps$, and
   $\incFun{\eps}=0$.
  \item[inductive step:] the proof proceeds by case analysis on the rule applied
   at the root of the derivation tree.
   \begin{itemize}
    \item $\Rule{l-cat}{\re_0\parDer{\sym}\re'_0}{\re_0\catop\re_1\parDer{\sym}\re'_0\catop\re_1}{}$\\[2ex]
          If $\geqFun{\re_0}{\re_1}=1$, then by definition of $\incFun{}$ and inductive hypothesis, $\incFun{\re'_0\catop\re_1}=\incFun{\re'_0}=0$.
          If $\geqFun{\re_0}{\re_1}=0$, then by definition of $\incFun{}$ $\incFun{\re'_0\catop\re_1}=0$.

    \item $\Rule{r-cat}{\re_1\parDer{\sym}\re'_1}{\re_0\catop\re_1\parDer{\sym}\re'_1}{\hasEps{\re_0}=\eps}$\\[2ex]
          $\incFun{\re'_1}=0$ directly follows from the inductive hypothesis.

    \item the proof for rules \rn{l-or} and \rn{r-or} is the same as for rule \rn{r-cat}.

    \item the proof for rule \rn{star} is the same as for rule \rn{l-cat}.
   \end{itemize}
 \end{description}
\end{proof}

Given \cref{lemma:zero-inc}, the following corollary may look useless,
but it provides a general form of invariant which can be proved also for all other considered cases, where the strong claim of \cref{lemma:zero-inc} no longer holds.

\begin{corollary}\label[corollary]{cor:inv}
 For all $\re,\re'\in\reSet$, $\sym\in\symAlph$, if $\re\parDer{\sym}\re'$, then $\theight{\re'}\leq \theight{\re}+\incFun{\re}-\incFun{\re'}$.
\end{corollary}
\begin{proof}
 A direct consequence of \cref{theo:parDer} and \cref{lemma:zero-inc}.
\end{proof}

The following theorem shows that the invariant property of \cref{cor:inv} can be extended to derivatives \wrt~ any words.
\begin{theorem}\label[theorem]{theo:gen-inc-bound}
 For all $\re,\re'\in\reSet$, and $\word\in\symAlph^*$, if $\re\parDer{\word}\re'$, then $\theight{\re'}\leq\theight{\re}+\incFun{\re}-\incFun{\re'}$.
\end{theorem}
\begin{proof}
 By induction on the length of $\word$.
 \begin{description}
  \item[base case:] if $\word=\emptyWord$, then by definition $\re'=\re$, hence
   $\theight{\re'}=\theight{\re}$ and $\incFun{\re}=\incFun{\re'}$, therefore $\theight{\re'}=\theight{\re}\leq\theight{\re}+\incFun{\re}-\incFun{\re'}$.

  \item[inductive step:]
   if $\word=\sym\strcons\word'$ for some $\sym\in\symAlph$, $\word'\in\symAlph^*$, then by definition there exists $\re''\in\reSet$ s.t. $\re\parDer{\sym}\re''$ and $\re''\parDer{\word'}\re'$. By \cref{cor:inv} $\theight{\re''}\leq\theight{\re}+\incFun{\re}-\incFun{\re''}$ and by inductive hypothesis $\theight{\re'}\leq\theight{\re''}+\incFun{\re''}-\incFun{\re'}$. Therefore by transitivity,
   $\theight{\re'}\leq\theight{\re''}+\incFun{\re''}-\incFun{\re'}\leq\theight{\re}+\incFun{\re}-\incFun{\re''}+\incFun{\re''}-\incFun{\re'}=\theight{\re}+\incFun{\re}-\incFun{\re'}$.
 \end{description}
\end{proof}

The proof of \cref{theo:gen-inc-bound} is independent from the definition of $\incFun{}$, providing that the invariant of \cref{cor:inv} holds. Therefore, the crucial point  is the definition of the function returning the upper bound of the increment of a partial derivation.

The result on the space complexity of derivatives can be directly derived from  \cref{theo:gen-inc-bound} and \cref{lemma:bounds}.

\begin{corollary}\label[corollary]{cor:gen-inc-bound}
 For all $\re,\re'\in\reSet$, and $\word\in\symAlph^*$, if $\re\parDer{\word}\re'$, then $\theight{\re'}\leq\theight{\re}+1$.
\end{corollary}

\begin{proof}
 By \cref{theo:gen-inc-bound} $\theight{\re'}\leq\theight{\re}+\incFun{\re}-\incFun{\re'}$. By \cref{lemma:bounds}
 $\incFun{\re}-\incFun{\re'}\leq 1$, hence
 $\theight{\re'}\leq\theight{\re}+1$.
\end{proof}

\subsection{Size of partial derivatives}\label{sec:size}
\begin{figure}
 $$
  \begin{array}{c}
   \tsize{\eps}=\tsize{\sym}=1                                                    \\
   \tsize{\re_0\catop\re_1}=\tsize{\re_0\orop\re_1}=\tsize{\re_0}+\tsize{\re_1}+1 \\
   \tsize{\starop{\re}}=\tsize{\re}+1
  \end{array}
 $$
 \caption{Definition of the size of regular expressions}
 \label{fig:size}
\end{figure}

\begin{figure}
 $$
  \begin{array}{l}
   % \gtFun{\re_0}{\re_1}=
   % \begin{cases}
   %   1, & \text{if } \theight{\re_0} > \theight{\re_1} \\
   %   0, & \text{ otherwise}
   % \end{cases} \quad
   \incFunSize{\sym}=\incFunSize{\eps}=0                                      \qquad
   \incFunSize{\re_0\catop\re_1}=\max(\incFunSize{\re_0},\incFunSize{\re_1}-\tsize{\re_0}-1) \\
   \incFunSize{\re_0\orop\re_1}= \max(\incFunSize{\re_0}-\tsize{\re_1}-1,\incFunSize{\re_1}-\tsize{\re_0}-1,0)                                        \qquad
   \incFunSize{\starop{\re}}=\tsize{\re}+\incFunSize{\re}+1
  \end{array}
 $$
 \caption{Definition of the increment function $\incFunSize{\re}$}
 \label{fig:incFunSize}
\end{figure}

In this section we investigate the upper bound on the size of the partial derivatives of a given regular expression.

The standard definition of the size of a regular expression considered as a tree can be found in \cref{fig:size}.

The results of~\cref{sec:height} must be adapted to the less trivial case of the size of expressions. Indeed, while for the height the upper bound of the increment is the constant 1, for the size this value depends on the square of the initial regular expression.

For instance, let us consider the following reduction steps which compute the partial derivative of $\starop{\starop{\starop{\sym}}}$
\wrt~ $\sym\sym$:
\begin{flushleft}
 $
  \begin{array}{l}
   \starop{\starop{\starop{\sym}}}\parDer{\sym}((\eps\catop\starop{\sym})\catop\starop{\starop{\sym}})\catop\starop{\starop{\starop{\sym}}}\parDer{\sym}((\eps\catop\starop{\sym})\catop\starop{\starop{\sym}})\catop\starop{\starop{\starop{\sym}}} \\
   \tsize{\starop{\starop{\starop{\sym}}}}=4\qquad  \tsize{((\eps\catop\starop{\sym})\catop\starop{\starop{\sym}})\catop\starop{\starop{\starop{\sym}}}}=13
  \end{array}
 $
\end{flushleft}
After the first step the size increases by 9 and remains unchanged after the second step.
For expressions of the general shape $\re=\starop{\sym}\ldots*$ it can be proved by induction on $n=\tsize{\re}\geq 2$ that if $\re\parDer{\sym}\re'$, then
$\tsize{\re'}=\tsize{\re}+\frac{n^2+n}{2}-1$.

As another example, let us consider the following reduction steps which compute the partial derivative of $(\starop{\sym}\catop\starop{\asym})\catop\starop{\yasym}$
\wrt~ $\sym\asym\yasym$:
\begin{flushleft}
 $
  \begin{array}{l}
   \sym\catop\starop{\starop{\asym}}\parDer{\sym}\eps\catop\starop{\starop{\asym}}\parDer{\asym}(\eps\catop\starop{\asym})\catop\starop{\starop{\asym}} \\
   \tsize{\sym\catop\starop{\starop{\asym}}}=5\qquad  \tsize{\eps\catop\starop{\starop{\asym}}}=5 \qquad \tsize{(\eps\catop\starop{\asym})\catop\starop{\starop{\asym}}}=8
  \end{array}
 $
\end{flushleft}
After the first step the size does not change, while increases by 3 after the second steps. This example shows that the strong claim of \cref{lemma:zero-inc} cannot hold for $\tsize{\ }$.

% As another example, let us consider the following reduction steps which compute the partial derivative of $(\starop{\sym}\catop\starop{\asym})\catop\starop{\yasym}$
% \wrt~ $\sym\asym\yasym$:
% \begin{flushleft}
%  $
%   \begin{array}{l}
%    (\starop{\sym}\catop\starop{\asym})\catop\starop{\yasym}\parDer{\sym} ((\eps\catop\starop{\sym})\catop\starop{\asym})\catop\starop{\yasym}\parDer{\asym} (\eps\catop\starop{\asym})\catop\starop{\yasym}\parDer{\yasym}\eps\catop\starop{\yasym} \\
%    \tsize{(\starop{\sym}\catop\starop{\asym})\catop\starop{\yasym}}=8\qquad  \tsize{((\eps\catop\starop{\sym})\catop\starop{\asym})\catop\starop{\yasym}}=10 \qquad \tsize{(\eps\catop\starop{\asym})\catop\starop{\yasym}}=7 \qquad \tsize{\eps\catop\starop{\yasym}}=4
%   \end{array}
%  $
% \end{flushleft}
% After the first step the size increases by 2, while decreases by 3 after each of the other steps.

We follow the same strategy as in \cref{sec:height} and define in \cref{fig:incFunSize} the function $\incFunSize{\re}$, analogous to $\incFun{\re}$, for the size of $\re$.

The following lemma is analogous to \cref{lemma:bounds}.
\begin{lemma}\label[lemma]{lemma:bounds-size}
 For all $\re\in\reSet$
 \[0\leq\incFunSize{\re}\leq \tsize{\re}^2\]
\end{lemma}
\begin{proof}
 Directly by induction on the definition of $\incFunSize{\re}$.
 We only show the proof for $\incFunSize{\re}\leq \tsize{\re}^2$ and omit the straightforward proof for $0\leq\incFunSize{\re}$.
 \begin{description}
  \item[base case:]  $\incFunSize{\sym}=\incFunSize{\eps}=0\leq 1 = \tsize{\sym}^2=\tsize{\eps}^2$
  \item[inductive step:]
   the proof proceeds by case analysis on the shape of $\re$.
   \begin{itemize}
    \item $\incFunSize{\re_0\catop\re_1}=\max(\incFunSize{\re_0},\incFunSize{\re_1}-\tsize{\re_0}-1)\leq\max(\incFunSize{\re_0},\incFunSize{\re_1})\leq\max(\tsize{\re_0}^2,\tsize{\re_1}^2)\leq \tsize{\re_0}^2+\tsize{\re_1}^2 \leq (\tsize{\re_0}+\tsize{\re_1})^2\leq (\tsize{\re_0}+\tsize{\re_1}+1)^2=\tsize{\re_0\catop\re_1}^2$, by the inductive hypothesis, the definition of $\max$, of square and of $\tsize{\ }$, and by the straightforward property $\tsize{\re}\geq 1$.
    \item $\incFunSize{\re_0\orop\re_1}\leq\tsize{\re_0\orop\re_1}^2$ can be proven similarly as in the previous case.
    \item $\incFunSize{\starop{\re}}=\tsize{\re}+\incFunSize{\re}+1\leq \tsize{\re}^2+\tsize{\re}+1\leq \tsize{\re}^2+2\cdot\tsize{\re}+1=(\tsize{\re}+1)^2=\tsize{\starop{\re}}^2$, by the inductive hypothesis, the definition of square and of $\tsize{\ }$, and by the straightforward property $\tsize{\re}\geq 1$.
   \end{itemize}
 \end{description}

\end{proof}

As for $\incFun{\re}$, the definition of $\incFunSize{\re}$ is driven by the reduction rules in \cref{fig:parDer}. We comment only the non-trivial cases: if $\re=\re_0\orop\re_1$, then the partial derivative can be obtained
by applying either rule \rn{l-or} or \rn{r-or}. For the former rule,
the partial derivative of $\re_0$ can increase of at most $\incFunSize{\re_0}$, but all the nodes of $\re_1$ and the $\orop$ operator are removed, which corresponds to $\incFunSize{\re_0}-\tsize{\re_1}-1$. The reasoning for the latter rule is symmetric. Then the maximum between these two values has to be considered to return a valid upper bound; furthermore, the bound must be non-negative to ensure that \cref{lemma:bounds-size} holds.
If $\re=\starop{\re_0}$, then the new term $\re'$ needs to be considered, whose size is bounded by $\tsize{\re}+\incFunSize{\re}$. Furthermore, the concatenation operator is added, hence the function returns $\tsize{\re}+\incFunSize{\re}+1$. Finally, in the definition for $\re=\re_0\catop\re_1$ rule \rn{r-cat} is managed as rule \rn{r-or} (or \rn{l-or}).
In rule \rn{l-cat} $\re_0$ is replaced by $\re_0'$, while $\re_1$ and concatenation are kept, hence the upper bound of the increment is $\incFunSize{\re_0}$. As for $\re_0\orop\re_1$ the maximum needs to be considered.

Although the analogous of \cref{lemma:zero-inc} cannot be proved for $\incFunSize{}$, the invariant of \cref{cor:inv} holds, and can be proved directly.
\begin{theorem}\label[theorem]{theo:inc-bound-size}
 For all $\re,\re'\in\reSet$, $\sym\in\symAlph$, if $\re\parDer{\sym}\re'$, then $\tsize{\re'}\leq \tsize{\re}+\incFunSize{\re}-\incFunSize{\re'}$.
\end{theorem}
\begin{proof}
 By induction and case analysis on the rules defining $\re\parDer{\sym}\re'$.
 \begin{description}
  \item[base case:] The only base rule is \rn{sym}.
   Hence, $\re=\sym$, $\re'=\eps$, and
   $\tsize{\eps}=0=\tsize{\sym}=\tsize{\sym}+0-0=\tsize{\sym}+\incFunSize{\sym}-\incFunSize{\eps}$.
  \item[inductive step:] the proof proceeds by case analysis on the rule applied
   at the root of the derivation tree.
   \begin{itemize}
    \item $\Rule{l-cat}{\re_0\parDer{\sym}\re'_0}{\re_0\catop\re_1\parDer{\sym}\re'_0\catop\re_1}{}$\\[2ex]
          By inductive hypothesis $\tsize{\re_0'}\leq \tsize{\re_0}+\incFunSize{\re_0}-\incFunSize{\re_0'}$.
          We distinguish two cases:
          \begin{itemize}
           \item $\incFunSize{\re_0'}\geq\incFunSize{\re_1}-\tsize{\re_0'}-1$:
                 $\tsize{\re_0'\catop\re_1}\stackrel{\mathrm{def}}{=}\tsize{\re'_0}+\tsize{\re_1}+1\leq\tsize{\re_0}+\incFunSize{\re_0}-\incFunSize{\re_0'}+\tsize{\re_1}+1=\tsize{\re_0\catop\re_1}+\incFunSize{\re_0}-\incFunSize{\re_0'}=\tsize{\re_0\catop\re_1}+\incFunSize{\re_0}-\incFunSize{\re_0'\catop\re_1}\leq\tsize{\re_0\catop\re_1}+\incFunSize{\re_0\catop\re_1}-\incFunSize{\re_0'\catop\re_1}$, by the inductive hypothesis, the definition of $\max$, $\tsize{\ }$, $\incFunSize{}$, and the assumption $\incFunSize{\re_0'}\geq\incFunSize{\re_1}-\tsize{\re_0'}-1$.

           \item $\incFunSize{\re_0'}<\incFunSize{\re_1}-\tsize{\re_0'}-1$:
                 $\tsize{\re_0'\catop\re_1}\stackrel{\mathrm{def}}{=}\tsize{\re'_0}+\tsize{\re_1}+1=\tsize{\re'_0}+\tsize{\re_1}+1+\incFunSize{\re'_0\catop\re_1}-\incFunSize{\re'_0\catop\re_1}=\tsize{\re'_0}+\tsize{\re_1}+1+\incFunSize{\re_1}-\tsize{\re_0'}-1-\incFunSize{\re'_0\catop\re_1}=\tsize{\re_1}+\incFunSize{\re_1}-\incFunSize{\re'_0\catop\re_1}=\tsize{\re_0\catop\re_1}+\incFunSize{\re_1}-\tsize{\re_0}-1-\incFunSize{\re'_0\catop\re_1}\leq\tsize{\re_0\catop\re_1}+\incFunSize{\re_0\catop\re_1}-\incFunSize{\re'_0\catop\re_1}$, by the definition of $\max$, $\tsize{\ }$, $\incFunSize{}$, and the assumption $\incFunSize{\re_0'}<\incFunSize{\re_1}-\tsize{\re_0'}-1$.
          \end{itemize}

    \item $\Rule{r-cat}{\re_1\parDer{\sym}\re'_1}{\re_0\catop\re_1\parDer{\sym}\re'_1}{\hasEps{\re_0}=\eps}$\\[2ex]
          By inductive hypothesis $\tsize{\re_1'}\leq \tsize{\re_1}+\incFunSize{\re_1}-\incFunSize{\re'_1}$.

          Therefore, $\tsize{\re_1'}\leq \tsize{\re_1}+\incFunSize{\re_1}-\incFunSize{\re'_1}=\tsize{\re_0\catop\re_1}+\incFunSize{\re_1}-\tsize{\re_0}-1-\incFunSize{\re'_1}\leq\tsize{\re_0\catop\re_1}+\incFunSize{\re_0\catop\re_1}-\incFunSize{\re'_1}$, by the inductive hypothesis, the definition of $\max$, $\tsize{\ },$ and $\incFunSize{}$.

    \item the case for rules \rn{l-or} and \rn{r-or} is analogous to that for rule \rn{r-cat}.
    \item $\Rule{star}{\re\parDer{\sym}\re'}{\starop{\re}\parDer{\sym}\re'\catop\starop{\re}}{}$\\[2ex]
          By inductive hypothesis $\tsize{\re'}\leq \tsize{\re}+\incFunSize{\re}-\incFunSize{\re'}$.

          We distinguish two cases:
          \begin{itemize}
           \item $\incFunSize{\re'}\geq\incFunSize{\starop{\re}}-\tsize{\re'}-1$:
                 $\tsize{\re'\catop\starop{\re}}\stackrel{\mathrm{def}}{=}\tsize{\re'}+\tsize{\starop{\re}}+1\leq\tsize{\re}+\incFunSize{\re}-\incFunSize{\re'}+\tsize{\starop{\re}}+1=\tsize{\starop{\re}}+\incFunSize{\starop{\re}}-\incFunSize{\re'}=\tsize{\starop{\re}}+\incFunSize{\starop{\re}}-\incFunSize{\re'\catop\starop{\re}}$, by the inductive hypothesis, the definition of $\max$, $\tsize{\ }$, $\incFunSize{}$, and the assumption $\incFunSize{\re'}\geq\incFunSize{\starop{\re}}-\tsize{\re'}-1$.

           \item $\incFunSize{\re'}<\incFunSize{\starop{\re}}-\tsize{\re'}-1$:
                 $\tsize{\re'\catop\starop{\re}}\stackrel{\mathrm{def}}{=}\tsize{\re'}+\tsize{\starop{\re}}+1=\tsize{\re'}+\tsize{\starop{\re}}+1+\incFunSize{\re'\catop\starop{\re}}-\incFunSize{\re'\catop\starop{\re}}=\tsize{\re'}+\tsize{\starop{\re}}+1+\incFunSize{\starop{\re}}-\tsize{\re'}-1-\incFunSize{\re'\catop\starop{\re}}=\tsize{\starop{\re}}+\incFunSize{\starop{\re}}-\incFunSize{\re'\catop\starop{\re}}$, by the definition of $\max$, $\tsize{\ }$, $\incFunSize{}$, and the assumption $\incFunSize{\re'}<\incFunSize{\starop{\re}}-\tsize{\re'}-1$.
          \end{itemize}
   \end{itemize}
 \end{description}
\end{proof}

Having proved the invariant of \cref{theo:inc-bound-size}, we can derive the main result analogously as done in \cref{sec:height}.
\begin{theorem}\label[theorem]{theo:gen-inc-bound-size}
 For all $\re,\re'\in\reSet$, and $\word\in\symAlph^*$, if $\re\parDer{\word}\re'$, then $\tsize{\re'}\leq\tsize{\re}+\incFunSize{\re}-\incFunSize{\re'}$.
\end{theorem}
\begin{proof}
 See \cref{theo:gen-inc-bound}
\end{proof}

The result on the space complexity of derivatives directly follows from  \cref{theo:gen-inc-bound-size} and \cref{lemma:bounds-size}.

\begin{corollary}\label[corollary]{cor:gen-inc-bound-size}
 For all $\re,\re'\in\reSet$, and $\word\in\symAlph^*$, if $\re\parDer{\word}\re'$, then $\tsize{\re'}\leq\tsize{\re}+\tsize{\re}^2$. Hence $\tsize{\re'}$ is $O(\tsize{\re}^2)$.
\end{corollary}

\begin{proof}
 By \cref{theo:gen-inc-bound-size} $\tsize{\re'}\leq\tsize{\re}+\incFunSize{\re}-\incFunSize{\re'}$. By \cref{lemma:bounds}
 $\incFunSize{\re}-\incFunSize{\re'}\leq\incFunSize{\re}\leq\tsize{\re}^2$, hence
 $\tsize{\re'}\leq\tsize{\re}+\tsize{\re}^2$.
\end{proof}
