\section{Partial derivatives of regular expressions}\label{sec:parDer}

\begin{figure}
 $$
  \begin{array}{c}
   \Rule{sym}{}{\sym\parDer{\sym}\eps}\quad
   \Rule{l-cat}{\re_0\parDer{\sym}\re'_0}{\re_0\catop\re_1\parDer{\sym}\re'_0\catop\re_1}{}\quad
   \Rule{r-cat}{\re_1\parDer{\sym}\re'_1}{\re_0\catop\re_1\parDer{\sym}\re'_1}{\hasEps{\re_0}=\eps} \\[4ex]
   \Rule{l-or}{\re_0\parDer{\sym}\re'_0}{\re_0\orop\re_1\parDer{\sym}\re'_0}{} \quad
   \Rule{r-or}{\re_1\parDer{\sym}\re'_1}{\re_0\orop\re_1\parDer{\sym}\re'_1}{} \quad
   \Rule{star}{\re\parDer{\sym}\re'}{\starop{\re}\parDer{\sym}\re'\starop{\re}}{}
  \end{array}
 $$
 \caption{Transition system defining the partial derivatives of regular expressions}
 \label{parDer}
\end{figure}

~\Cref{parDer} contains the transition rules defining the partial derivatives of regular expressions.
As for the definition of derivatives, the presentation deviates from that of Antimirov \cite{Antimirov96}, and keeps following the transition system style. Besides the different notation, more importantly, here we deliberately define a non-deterministic system, where a single transition step yields a single (non-zero) partial derivative among all possible ones, while in Antimirov's definition the set of all partial derivatives is returned.

For instance, according to Antimirov's definition\footnote{See \cite{Antimirov96}, definition 2.8.},
the partial derivative of $ab \orop ac$ w.r.t. $a$ returns the set of regular expressions $\{b,c\}$ (that is, $\delta_a(ab \orop ac)=\{b,c\}$ with Antimirov's notation). This corresponds to the fact that there exist two possible transition steps from $ab \orop ac$ labeled with $a$, namely,
$ab \orop ac\parDer{a}b$ and $ab \orop ac\parDer{a}c$.

This difference is directly related to the opposite aim of our work: while Antimirov's approach uses partial derivatives for generating an NFA, here
we use them to define a non-deterministic transition system which already represents the corresponding NFA.

\begin{definition}
 For all $\re,\re_0,\re_1\in\reSet$, $\sym\in\symAlph$, $\word\in\symAlph^*$
 \[
  \begin{array}{l}
   \\
   \re\parDer{\emptyWord}\re                                                                                       \\
   \re_0\parDer{\sym\strcons\word}\re_1 \mbox{ iff } \re_0\parDer{\sym}{}\re \mbox{ and } \re\parDer{\word}{}\re_1 \\
   \opSem{\re}{\parDer{}}=\{\word\in\wordUniv \mid \mbox{there exists } \re'\in\reSet \mbox{ s.t. } \re\parDer{\word}\re' \mbox{ and } \hasEps{\re'}=\eps\}
  \end{array}
 \]
\end{definition}


\begin{proposition}
 Let $\delta_\word(\re)$ denotes the set of all partial derivatives of $\re\in\reSet$ w.r.t. $\word\in\symAlph^*$.

 Then $\delta_\word(\re)=\{\re'\in\reSet \mid \re\parDer{\word}\re'\}$.
\end{proposition}

% \begin{definition}
%  Let $\reSet_1$ and $\reSet_2$ be sets of regular expressions.
%  \[
%   \begin{array}{l}
%    \reSet_1\derSet{\sym}\reSet_2 \mbox{ iff } \reSet_2=\{\re_1 \mid \mbox{ there exists } \re_0\in\reSet_1 \mbox{ s.t. } \re_0\parDer{\sym}{}\re_1 \} \\
%    \hasEps{\reSet}=\eps \mbox{ iff there exists } \re\in\reSet \mbox{ s.t. } \hasEps{\re}                                                             \\
%    \re_0\derSet{\sym\word}\re_1 \mbox{ iff } \re_0\derSet{\sym}\re \mbox{ and } \re\derSet{\word}\re_1                                                \\
%    \re\derSet{\emptyWord}{}\re                                                                                                                        \\
%    \opSem{\re}{\derSet{}}=\{\word\in\wordUniv \mid \mbox{there exists } \reSet \mbox{ s.t. } \{\re\}\derSet{\word}\reSet \mbox{ and } \hasEps{\reSet}=\eps\}
%   \end{array}
%  \]
% \end{definition}

\begin{theorem}
 For all $\re\in\reSet$
 \[
  \opSem{\re}{\parDer{}{}}=\opSem{\re}{\der{}}
 \]
\end{theorem}

\begin{corollary}
 For all $\re\in\reSet$
 \[
  \opSem{\re}{\parDer{}{}}=\sem{\re}
 \]
\end{corollary}
% \begin{theorem}
%  For all regular expressions $\re$
%  \[
%   \opSem{\re}{\derSet{}}=\opSem{\re}{\der{}}
%  \]
% \end{theorem}

\begin{figure}
 $$
  \begin{array}{c}
   \theight{\eps}=\theight{\sym}=0                                                              \\
   \theight{\re_0\catop\re_1}=\theight{\re_0\orop\re_1}=\max(\theight{\re_0},\theight{\re_1})+1 \\
   \theight{\starop{\re_0}}=\theight{\re_0}+1
  \end{array}
 $$
 \caption{Definition of the height of regular expressions}
 \label{fig:height}
\end{figure}

\begin{figure}
 $$
  \begin{array}{l}
   % \gtFun{\re_0}{\re_1}=
   % \begin{cases}
   %   1, & \text{if } \theight{\re_0} > \theight{\re_1} \\
   %   0, & \text{ otherwise}
   % \end{cases} \quad
   \geqFun{\re_0}{\re_1}=
   \begin{cases}
    1, & \text{if } \theight{\re_0} \geq \theight{\re_1} \\
    0, & \text{if } \theight{\re_0} < \theight{\re_1}
   \end{cases} \\[4ex]
   \incFun{\sym}=\incFun{\eps}=0                                      \qquad
   \incFun{\re_0\catop\re_1}=\geqFun{\re_0}{\re_1}\cdot\incFun{\re_0} \qquad
   \incFun{\re_0\orop\re_1}=0                                         \qquad
   \incFun{\starop{\re_0}}=1
  \end{array}
 $$
 \caption{Definition of the increment function $\incFun{\re}$}
 \label{fig:inc-pred}
\end{figure}

\begin{lemma}\label{bounds}
 For all $\re\in\reSet$
 \[0\leq\incFun{\re}\leq 1\]
\end{lemma}
\begin{proof}
 Directly by induction on the definition of $\incFun{\re}$.
\end{proof}

\begin{theorem}\label{inc-bound}
 For all $\re,\re'\in\reSet$, $\sym\in\symAlph$, if $\re\parDer{\sym}\re'$, then $\theight{\re'}\leq \theight{\re}+\incFun{\re}$.
\end{theorem}
\begin{proof}
 By induction and case analysis on rules defining $\re\parDer{\sym}\re'$.
 \begin{description}
  \item[base case:] The only base rule is \rn{sym}.
   Hence, $\re=\sym$, $\re'=\eps$, and
   $\theight{\eps}=0=\theight{\sym}=\theight{\sym}+0=\theight{\sym}+\incFun{\sym}$.
  \item[inductive step:] the proof proceeds by case analysis on the rule applied
   at the root of the derivation tree.
   \begin{itemize}
    \item $\Rule{l-cat}{\re_0\parDer{\sym}\re'_0}{\re_0\catop\re_1\parDer{\sym}\re'_0\catop\re_1}{}$\\[2ex]
          By inductive hypothesis $\theight{\re_0'}\leq \theight{\re_0}+\incFun{\re_0}$.
          We distinguish two cases:
          \begin{itemize}
           \item $\theight{\re_0}\geq\theight{\re_1}$:
                 $\theight{\re_0'\catop\re_1}\stackrel{\mathrm{def}}{=}\max(\theight{\re'_0},\theight{\re_1})+1\leq\max(\theight{\re_0}+\incFun{\re_0},\theight{\re_1})+1\leq\max(\theight{\re_0},\theight{\re_1})+1+\incFun{\re_0}=\theight{\re_0\catop\re_1}+\incFun{\re_0}=\theight{\re_0\catop\re_1}+\incFun{\re_0\catop\re_1}$, where inequalities are derived from the inductive hypothesis, the definition of $\max$, $\theight{\ }$, $\incFun{}$, and $\geqSym()$, the assumption $\theight{\re_0}\geq\theight{\re_1}$, and lemma~\ref{bounds}.

           \item $\theight{\re_0}<\theight{\re_1}$:
                 $\theight{\re_0'\catop\re_1}\stackrel{\mathrm{def}}{=}\max(\theight{\re'_0},\theight{\re_1})+1\leq\max(\theight{\re_0}+\incFun{\re_0},\theight{\re_1})+1=\theight{\re_0\catop\re_1}=\theight{\re_0\catop\re_1}+\incFun{\re_0\catop\re_1}$, where inequalities are derived from the inductive hypothesis, the definition of $\max$, $\theight{\ }$, $\incFun{}$, and $\geqSym()$, the assumption $\theight{\re_0}<\theight{\re_1}$, and lemma~\ref{bounds}.
          \end{itemize}

    \item $\Rule{r-cat}{\re_1\parDer{\sym}\re'_1}{\re_0\catop\re_1\parDer{\sym}\re'_1}{\hasEps{\re_0}=\eps}$\\[2ex]
          By inductive hypothesis $\theight{\re_1'}\leq \theight{\re_1}+\incFun{\re_1}$.

          Therefore, $\theight{\re_1'}\leq \theight{\re_1}+\incFun{\re_1}\leq\theight{\re_1}+1\leq\theight{\re_0\catop\re_1}\leq\theight{\re_0\catop\re_1}+\incFun{\re_0\catop\re_1}$, where inequalities are derived from the inductive hypothesis, the definition of $\max$, $\theight{\ },$ and $\incFun{}$, and lemma~\ref{bounds}.

    \item $\Rule{l-or}{\re_0\parDer{\sym}\re'_0}{\re_0\orop\re_1\parDer{\sym}\re'_0}{}$\\[2ex]
          By inductive hypothesis $\theight{\re_0'}\leq \theight{\re_0}+\incFun{\re_0}$.

          Therefore, $\theight{\re_0'}\leq \theight{\re_0}+\incFun{\re_0}\leq\theight{\re_0}+1\leq\theight{\re_0\orop\re_1}=\theight{\re_0\orop\re_1}+\incFun{\re_0\orop\re_1}$, where inequalities are derived from the inductive hypothesis, the definition of $\max$, $\theight{\ },$ and $\incFun{}$, and lemma~\ref{bounds}.
    \item rule \rn{r-or} is symmetric to  rule \rn{l-or}.
    \item $\Rule{star}{\re\parDer{\sym}\re'}{\starop{\re}\parDer{\sym}\re'\starop{\re}}{}$\\[2ex]
          By inductive hypothesis $\theight{\re'}\leq \theight{\re}+\incFun{\re}$.

   \end{itemize}
 \end{description}
\end{proof}

\begin{theorem}\label{zero-inc}
 If $\re\parDer{\sym}{}\re'$ and $\incFun{\re'}=k$, then $k\leq 0$.
 \begin{proof}
  By virtue of \cref{red-step} $\re'=\eps$ or $\re'=\re_0\catop\re_1$ for some $\re_0,\re_1$.

  If $\re'=\eps$, then by definition $\tpath=\emptyPath$ and $k=0$.

  If $\re'=\re_0\catop\re_1$, then the proof proceeds by induction on the definition of $\incFun{\re'}$.

  Only two different cases can occur: either $\tpath=0\pathCons\tpath'$ or $\tpath=1\pathCons\tpath'$ for some  $\tpath'$.

  In the former case $\incFun{\re_0\catop\re_1}=\geqFun{\re_0}{\re_1}\cdot\max(\incFun{\re_0}{\tpath'},\theight{\re_1}-\theight{\re_0})$.

  If $\theight{\re_0}\geq\theight{\re_1}$ then $\incFun{\re_0\catop\re_1}=\max(\incFun{\re_0}{\tpath'},\theight{\re_1}-\theight{\re_0})$.
  By induction $\incFun{\re_0}{\tpath'}\leq 0$, furthermore $\theight{\re_1}-\theight{\re_0}<0$, therefore $\max(\incFun{\re_0}{\tpath'},\theight{\re_1}-\theight{\re_0})\leq 0$.

  If $\theight{\re_0}<\theight{\re_1}$ then trivially $\incFun{\re_0\catop\re_1}=0$.


 \end{proof}
\end{theorem}
\section{Partial derivatives of regular expressions}\label{sec:regexp-par}

\begin{figure}
 $$
  \begin{array}{c}
   \simpleRule{}{\sym\parDer{\sym}\eps}{}\quad
   \simpleRule{}{\all\parDer{\sym}\all}{}                                                                                   \\[4ex]
   \simpleRule{\re_0\parDer{\sym}\re'_0}{\re_0\catop\re_1\parDer{\sym}\re'_0\catop\re_1}{}\quad
   \simpleRule{\re_0\noParDer{\sym}\quad\re_1\parDer{\sym}\re'_1}{\re_0\catop\re_1\parDer{\sym}\re'_1}{\hasEps{\re_0}=\eps} \\[4ex]
   \simpleRule{\re_0\parDer{\sym}\re'_0}{\re_0\orop\re_1\parDer{\sym}\re'_0}{}\quad
   \simpleRule{\re_0\noParDer{\sym}\quad\re_1\parDer{\sym}\re'_1}{\re_0\orop\re_1\parDer{\sym}\re'_1}{}\quad
   \simpleRule{\re_0\parDer{\sym}\re'_0\quad\re_1\parDer{\sym}\re'_1}{\re_0\andop\re_1\parDer{\sym}\re'_0\andop\re'_1}{}    \\[4ex]
   \simpleRule{\re_0\parDer{\sym}\re'_0}{\re_0\shuffleop\re_1\parDer{\sym}\re'_0\shuffleop\re_1}{}\quad
   \simpleRule{\re_0\noParDer{\sym}\quad\re_1\parDer{\sym}\re'_1}{\re_0\shuffleop\re_1\parDer{\sym}\re_0\shuffleop\re'_1}{}\quad
   \simpleRule{\re\parDer{\sym}\re'}{\starop{\re_0}\parDer{\sym}\re'\starop{\re_0}}{}                                       \\[4ex]
   \simpleRule{}{\eps\noParDer{\sym}}{}\quad
   \simpleRule{}{\sym\noParDer{\asym}}{\asym\neq\sym}\quad
   \simpleRule{}{\none\noParDer{\sym}}{}                                                                                    \\[4ex]
   \simpleRule{\re_0\noParDer{\sym}}{\re_0\catop\re_1\noParDer{\sym}}{\hasEps{\re_0}=\none}\quad
   \simpleRule{\re_0\noParDer{\sym}\quad\re_1\noParDer{\sym}}{\re_0\catop\re_1\noParDer{\sym}}{} \quad
   \simpleRule{\re_0\noParDer{\sym}\quad\re_1\noParDer{\sym}}{\re_0\orop\re_1\noParDer{\sym}}{}                             \\[4ex]
   \simpleRule{\re_0\noParDer{\sym}}{\re_0\andop\re_1\noParDer{\sym}}{}\quad
   \simpleRule{\re_1\noParDer{\sym}}{\re_0\andop\re_1\noParDer{\sym}}{}\quad
   \simpleRule{\re_0\noParDer{\sym}\quad\re_1\noParDer{\sym}}{\re_0\shuffleop\re_1\noParDer{\sym}}{}\quad
   \simpleRule{\re\noParDer{\sym}}{\starop{\re_0}\noParDer{\sym}}{}
  \end{array}
 $$
 \caption{Partial derivatives of regular expressions, deterministic definition}
 \label{fig:reg-exp-par-der}
\end{figure}


\begin{theorem}
 For all regular expressions $\re$
 \[
  \opSem{\re}{\parDer{}}\subseteq\opSem{\re}{\parDer{}{}}
 \]
\end{theorem}

\begin{theorem}
 There exist regular expressions $\re$ s.t.
 \[
  \opSem{\re}{\parDer{}{}}\not\subseteq\opSem{\re}{\parDer{}}
 \]
\end{theorem}

\begin{definition}
 \[
  \begin{array}{l}
   \first{\none}=\emptyset\quad \first{\all}=\symAlph \quad \first{\eps}=\emptyset \quad \first{\sym}=\{\sym\}                                      \\[2ex]
   \first{\re_0\catop\re_1}=\first{\re_0}\cup(\sem{\hasEps{\re_0}}\catop\first{\re_1}) \quad \first{\re_0\orop\re_1}=\first{\re_0}\cup\first{\re_1} \\[2ex]
   \first{\re_0\andop\re_1}=\first{\re_0}\cap\first{\re_1} \quad \first{\re_0\shuffleop\re_1}=\first{\re_0}\cup\first{\re_1} \quad
   \first{\starop{\re_0}}=\first{\re}
  \end{array}
 \]
\end{definition}

\begin{theorem}
 For all $\sym$, $\re$,
 \[
  \sym\in\first{\re} \mbox{ iff there exists } \re' \mbox{ s.t. } \re\parDer{\sym}\re'
 \]
\end{theorem}

\begin{theorem}
 For all $\sym$, $\word$, $\re$,
 \[
  \mbox{ if } \sym\catop\word\in\sem{\re} \mbox{ then } \sym\in\first{\re}
 \]
\end{theorem}

\begin{theorem}
 For all $\sym$, $\re$,
 \[
  \mbox{ if } \sym\in\first{\re} \mbox{ and } \sem{\re}\neq\emptyset \mbox{ then there exists } \word \mbox{ s.t. } \sym\word\in\sem{\re}
 \]
\end{theorem}

