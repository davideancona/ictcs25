\section{Regular expressions with shuffle}\label{sec:shuffle}

\subsection{Basic definitions}
In this section we extend the syntax of regular expressions by adding the shuffle operator\footnote{In the examples we assume that the shuffle has lower precedence than the concatenation and union operator.} $\re_0\shuffleop\re_1$ whose semantics is defined as follows by extending \cref{def:langOp} and \cref{def:sem}:
\[
 \begin{array}{l}
  \emptyWord \shuffleop \word = \word \shuffleop \emptyWord = \{\word\}                                                                                                      \qquad
  (\sym_1\word_1) \shuffleop (\sym_2\word_2) = \{ \sym_1  \} \strcatop (\word_1 \strshuffleop (\sym_2\word_2)) \cup \{\sym_2\}\strcatop((\sym_1\word_1)\strshuffleop\word_2) \\
  \lang_1\strshuffleop\lang_2=\bigcup_{\word_1\in\lang_1,\word_2\in\lang_2}(\word_1\strshuffleop\word_2)                                                                     \qquad
  \sem{\re_0\shuffleop\re_1}=\sem{\re_0}\strshuffleop\sem{\re_1}
 \end{array}
\]
We denote with $\reSetExt$ the set of regular expressions extended with the shuffle operator.

Although it is well-known that the shuffle operator does not increase the abstract expressive power of regular expressions, in practice it is still very useful in RV to write much more compact and clear
specifications when correct system behaviors can be described as independently interleaved event traces. Indeed, it has been proved \cite{BrodaEtAl18,MayerElAl94} that there exists a family of regular expressions $\re$ with shuffle for which the equivalent NFA needs at least $2^{\tsize{\re}}$ states.

Let us consider for instance the distinct events $o_n$, $a_n$, and $c_n$, with the meaning
``file $n$ has been opened'', ``accessed'', and ``closed'', respectively, where $n=1,2$ is the corresponding file descriptor.
If the SUS is allowed to manage the two files independently, then a specification for the correct use of them can be defined quite concisely by the regular expression $o_1\catop\starop{a_1}\catop c_1\shuffleop o_2\catop\starop{a_2}\catop c_2$.

If, for simplicity, we assume that files can be accessed only once, and we generalize over the number of files, then the specification becomes the regular expression $o_1\catop a_1\catop c_1\shuffleop\ldots\shuffleop o_n\catop a_n \catop c_n$ over the alphabet of $3n$ distinct symbols $\{o_1,a_1,c_1,\dots, o_n,a_n,c_n\}$. For such an expression, an equivalent NFA must have at least
$4^n$ states.

\begin{figure}[h]
 $$
  \begin{array}{c}
   \Rule{shf}{\re_0\der{\sym}\re'_0\quad\re_1\der{\sym}\re'_1}{\re_0\shuffleop\re_1\der{\sym}(\re'_0\shuffleop\re_1)\orop(\re_0\shuffleop\re'_1)}{}\qquad \hasEps{\re_0\shuffleop\re_1}=\hasEps{\re_0}\conj\hasEps{\re_1} \\[4ex]
   \Rule{l-shf}{\re_0\parDer{\sym}\re'_0}{\re_0\shuffleop\re_1\parDer{\sym}\re'_0\shuffleop\re_1}{} \qquad
   \Rule{r-shf}{\re_1\parDer{\sym}\re'_1}{\re_0\shuffleop\re_1\parDer{\sym}\re_0\shuffleop\re'_1}{}
  \end{array}
 $$
 \caption{Transition rules defining the derivative and the partial derivatives for the shuffle operator}
 \label{fig:shfParDer}
\end{figure}

The transition rules defining the derivative and the partial derivatives for the shuffle operator
can be found in \cref{fig:shfParDer}. They all correspond to the intuition that events can be interleaved; moreover, the empty trace is contained in $\re_0\shuffleop\re_1$ iff it is contained in $\re_0$ and $\re_1$.

As happens for the other operators, also with the shuffle  the derivative is always defined, while the partial derivative is defined only for non-empty derivatives. For instance, if we assume $\sym_2\neq\sym_0$ and $\sym_2\neq\sym_1$, then
$\sym_0\shuffleop \sym_1\der{\sym_2}(\none\shuffleop\sym_1)\orop(\sym_0\shuffleop\none)$, and $\hasEps{(\none\shuffleop\sym_1)\orop(\sym_0\shuffleop\none)}=\none$, but there exists no $\re$ s.t.
$\sym_0\shuffleop \sym_1\parDer{\sym_2}\re$.

Theorems~\ref{theo:der} and \ref{theo:parDer} still hold along with  \cref{prop:parDer} and \cref{cor:parDer}, when $\reSetExt$ is considered.

\subsection{Height of partial derivatives with the shuffle operator}

We extend the result of \cref{theo:gen-inc-bound} and \cref{cor:gen-inc-bound} to the case of the shuffle operators.

Unfortunately the proofs are more challenging because the claim of \cref{lemma:zero-inc} no longer holds: There exist partial derivatives $\re$ \st~$\incFun{\re}>0$.
As a counter example, let us consider the following reduction steps computing the partial derivatives of $(\eps\shuffleop\starop{\sym})\catop(\asym\shuffleop\starop{\sym})$ \wrt~ $\sym\asym\sym$:
\begin{flushleft}
 $
  \begin{array}{l}
   \re_0\parDer{\sym}\re_1\parDer{\asym}\re_2\parDer{\sym}\re_3 \\
   \re_0=(\eps\shuffleop\starop{\sym})\catop(\asym\shuffleop\starop{\sym}) \qquad
   \re_1=(\eps\shuffleop\eps\catop\starop{\sym})\catop(\asym\shuffleop\starop{\sym})                                     \qquad
   \re_2=\eps\shuffleop\starop{\sym}                                       \qquad \re_3=\eps\shuffleop\eps\catop\starop{\sym}
  \end{array}
 $
\end{flushleft}
We have $\theight{\re_1}=\theight{\re_0}+1$ (first reduction step), but also $\theight{\re_3}=\theight{\re_2}+1$ (third reduction step). Therefore, necessarily $\incFun{\re_2}>0$, by definition of $\incFun{}$.

To prove the extended versions of the results of \cref{sec:parDer} we first extend the definition of $\incFun{}$ given in \cref{fig:incFun} with the case for the shuffle.
\[
 \incFun{\re_0\shuffleop\re_1}=\max(\geqFun{\re_0}{\re_1}\cdot\incFun{\re_0},\geqFun{\re_1}{\re_0}\cdot\incFun{\re_1})
\]
Before explaining the definition of $\incFun{}$ above, we note that the claim of \cref{lemma:bounds} holds also for $\reSetExt$ and can still be proved by induction on the definition of $\incFun{}$: $0\leq\incFun{\re}\leq 1$.
Consequently, if the height of one sub-expression $\re_i$ is strictly greater than the other $\re_{1-i}$, then only the partial derivative of $\re_i$ can contribute to the increment of the height of the partial derivative of $\re_0\shuffleop\re_1$, because increments are always bounded by 1.
Note that $\max(\incFun{\re_i},0)=\incFun{\re_i}$, since $\incFun{\re_i}\geq 0$, therefore $\incFun{\re_0\shuffleop\re_1}=\incFun{\re_i}$, if $\theight{\re_i}>\theight{\re_{1-i}}$ for $i=0,1$.

If $\theight{\re_0}=\theight{\re_1}$, then both the derivatives of $\re_0$ and $\re_1$ can contribute to the increment of the height of the partial derivative of $\re_0\shuffleop\re_1$.
Since \rn{l-shf} or \rn{r-shf} can be non-deterministically applied, the maximum increment
$\max(\incFun{\re_0},\incFun{\re_1})$ needs to be considered.

The correctness of the definition of $\incFun{\re_0\shuffleop\re_1}$ is still ensured by the claim of \cref{theo:inc-bound} extended to $\reSetExt$.

\subsubsection*{Proof of \cref{theo:inc-bound} extended to $\reSetExt$
}
For all $\re,\re'\in\reSetExt$, $\sym\in\symAlph$, if $\re\parDer{\sym}\re'$, then $\theight{\re'}\leq \theight{\re}+\incFun{\re}$.

\begin{proof}
 The structure of the proof is the same, but the two new rules for the shuffle operator need to be considered.
 \begin{itemize}
  \item %$\Rule{l-shf}{\re_0\parDer{\sym}\re'_0}{\re_0\shuffleop\re_1\parDer{\sym}\re'_0\shuffleop\re_1}{}$\\[2ex]
        rule \rn{l-shf}: By inductive hypothesis $\theight{\re_0'}\leq \theight{\re_0}+\incFun{\re_0}$.
        We distinguish two cases:
        \begin{itemize}
         \item $\theight{\re_0}\geq\theight{\re_1}$:
               $\theight{\re_0'\shuffleop\re_1}\stackrel{\mathrm{def}}{=}\max(\theight{\re'_0},\theight{\re_1})+1\leq\max(\theight{\re_0}+\incFun{\re_0},\theight{\re_1})+1\leq\max(\theight{\re_0},\theight{\re_1})+1+\incFun{\re_0}\leq\max(\theight{\re_0},\theight{\re_1})+1+\incFun{\re_0\shuffleop\re_1}=\theight{\re_0\shuffleop\re_1}+\incFun{\re_0\shuffleop\re_1}$, where inequalities are derived from the inductive hypothesis, the definition of $\max$, $\theight{\ }$, $\incFun{}$, and $\geqSym()$, the assumption $\theight{\re_0}\geq\theight{\re_1}$, and \cref{lemma:bounds}.

         \item $\theight{\re_0}<\theight{\re_1}$ (that is, $\theight{\re_0}+\incFun{\re_0}\leq\theight{\re_1}$ by \cref{lemma:bounds}):
               $\theight{\re_0'\shuffleop\re_1}\stackrel{\mathrm{def}}{=}\max(\theight{\re'_0},\theight{\re_1})+1\leq\max(\theight{\re_0}+\incFun{\re_0},\theight{\re_1})+1=\theight{\re_0\shuffleop\re_1}\leq\theight{\re_0\shuffleop\re_1}+\incFun{\re_0\shuffleop\re_1}$, where inequalities are derived from the inductive hypothesis, the definition of $\max$, $\theight{\ }$, $\incFun{}$, and $\geqSym()$, the assumption $\theight{\re_0}<\theight{\re_1}$, and \cref{lemma:bounds}.
        \end{itemize}
  \item rule (r-shf) is symmetric to rule (l-shf).
 \end{itemize}
\end{proof}

Since both \cref{lemma:bounds} and \cref{theo:inc-bound} hold for $\reSetExt$, \cref{cor:bound} holds for $\reSetExt$ as well.

To prove the invariant of \cref{cor:inv} for $\reSetExt$, we modularize the proof by introducing two lemmas.

The first lemma is a weaker version of \cref{lemma:zero-inc}.

\begin{lemma}\label[lemma]{lemma:ext-zero-inc}
 For all $\re,\re'\in\reSetExt$, and $\sym\in\symAlph$, if $\re\parDer{\sym}\re'$ and $\theight{\re'}=\theight{\re}+1$, then $\incFun{\re'}=0$.
\end{lemma}

The second lemma establishes an invariant on $\incFun{}$ when the height of the derivatives does not change.

\begin{lemma}\label[lemma]{lemma:ext-leq-inc}
 For all $\re,\re'\in\reSetExt$, and $\sym\in\symAlph$, if $\re\parDer{\sym}\re'$ and $\theight{\re'}=\theight{\re}$, then $\incFun{\re'}\leq\incFun{\re}$.
\end{lemma}

\begin{theorem}\label[theorem]{theo:inv}
 For all $\re,\re'\in\reSetExt$, $\sym\in\symAlph$, if $\re\parDer{\sym}\re'$, then $\theight{\re'}\leq\theight{\re}+\incFun{\re}-\incFun{\re'}$.
\end{theorem}

From \cref{theo:inv} and \cref{lemma:bounds} (extended to $\reSetExt$) we can
extend for free \cref{theo:gen-inc-bound} and \cref{cor:gen-inc-bound}  to $\reSetExt$.

\begin{theorem}
 For all $\re,\re'\in\reSetExt$, and $\word\in\symAlph^*$, if $\re\parDer{\word}\re'$, then $\theight{\re'}\leq\theight{\re}+\incFun{\re}-\incFun{\re'}$.
\end{theorem}

\begin{corollary}
 For all $\re,\re'\in\reSetExt$, and $\word\in\symAlph^*$, if $\re\parDer{\word}\re'$, then $\theight{\re'}\leq\theight{\re}+1$.
\end{corollary}

% It is now possible to prove \cref{theo:gen-inc-bound} extended to $\reSetExt$.

% \subsubsection*{Proof of \cref{theo:gen-inc-bound} extended to $\reSetExt$
% }
% For all $\re,\re'\in\reSetExt$, $\word\in\symAlph^*$, if $\re\parDer{\word}\re'$, then $\theight{\re'}\leq \theight{\re}+\incFun{\re}$.

% \begin{proof}
%  By induction on the length of $\word$.
%  \begin{description}
%   \item[base case:] if $\word=\emptyWord$, then by definition $\re'=\re$, hence
%    $\theight{\re'}=\theight{\re}$, and by \cref{lemma:bounds}, $\theight{\re'}=\theight{\re}\leq\theight{\re}+\incFun{\re}$.

%   \item[inductive step:]
%    if $\word=\sym\strcons\word'$ for some $\sym\in\symAlph$, $\word'\in\symAlph^*$, then by definition there exists $\re''\in\reSet$ s.t. $\re\parDer{\sym}\re''$ and $\re''\parDer{\word'}\re'$. By inductive hypothesis and by \cref{lemma:triang}, $\theight{\re'}\leq\theight{\re''}+\incFun{\re''}=\theight{\re''}-\theight{\re}+\incFun{\re''}+\theight{\re}\leq\theight{\re}+\incFun{\re}$.

%    %    Since by \cref{cor:bound} $\theight{\re''}\leq\theight{\re}+1$, we can distinguish three cases:
%    %    \begin{itemize}
%    %     \item $\theight{\re''}=\theight{\re}+1$: by \cref{lemma:ext-zero-inc}, $\incFun{\re''}=0$,
%    %           therefore $\theight{\re'}\leq\theight{\re}+\incFun{\re}+\incFun{\re''}=\theight{\re}+\incFun{\re}$.
%    %     \item $\theight{\re''}=\theight{\re}$: by inductive hypothesis and \cref{lemma:ext-leq-inc},
%    %           $\theight{\re'}\leq\theight{\re''}+\incFun{\re''}=\theight{\re}+\incFun{\re''}\leq\theight{\re}+\incFun{\re}$.
%    %     \item $\theight{\re''}<\theight{\re}$ (that is, $\theight{\re''}\leq\theight{\re}-1$): by inductive hypothesis and \cref{lemma:bounds},
%    %           $\theight{\re'}\leq\theight{\re''}+\incFun{\re''}\leq\theight{\re}-1+\incFun{\re''}\leq\theight{\re}\leq\theight{\re}+\incFun{\re}$.
%    %    \end{itemize}
%  \end{description}
% \end{proof}

% Finally, from the versions of \cref{theo:gen-inc-bound} and \cref{lemma:bounds}, both extended to $\reSetExt$, \cref{cor:gen-inc-bound} can be directly extended as well: For all $\re,\re'\in\reSetExt$, and $\word\in\symAlph^*$, if $\re\parDer{\word}\re'$, then $\theight{\re'}\leq\theight{\re}+1$.

\subsection{Size of partial derivatives with the shuffle operator}

Similar as done in the previous section, we extend the result of \cref{theo:gen-inc-bound-size} and \cref{cor:gen-inc-bound-size} to the case of the shuffle operators in the case of the size $\tsize{\re}$ of a regular expression $\re$.

Also for the size, the introduction of shuffle does not affect the space complexity of partial derivatives. However, in this case the proofs can be extended in an easier way, although the definition of $\incFunSize{\re_0\shuffleop\re_1}$ requires some care.

Indeed, one would be tempted to consider the following (incorrect) definition:
\[
 \incFunSize{\re_0\shuffleop\re_1}=\max(\incFunSize{\re_0},\incFunSize{\re_1})
 \qquad\qquad\qquad\qquad\mbox{(incorrect definition)}
\]
This is in line with $\incFunSize{\re_0\catop\re_1}$,
but, unfortunately, verifies only the weaker inequality
$\tsize{\re}\leq\tsize{\re_0\shuffleop\re_1}+\incFunSize{\re_0\shuffleop\re_1}$, if $\re_0\shuffleop\re_1\parDer{\sym}\re$, but not $\tsize{\re}\leq\tsize{\re_0\shuffleop\re_1}+\incFunSize{\re_0\shuffleop\re_1}-\incFunSize{\re}$.

To show this, let us consider the following reduction step computing the partial derivative of $\starop{\sym}\shuffleop\starop{\asym}$ \wrt~ $\sym$:
\begin{flushleft}
 $
  \begin{array}{l}
   \starop{\sym}\shuffleop\starop{\asym}\parDer{\sym}\eps\catop\starop{\sym}\shuffleop\starop{\asym} \\
   \tsize{\starop{\sym}\shuffleop\starop{\asym}}=5 \qquad
   \incFunSize{\starop{\sym}\shuffleop\starop{\asym}}=2 \qquad
   \tsize{\eps\catop\starop{\sym}\shuffleop\starop{\asym}}=7 \qquad
   \incFunSize{\eps\catop\starop{\sym}\shuffleop\starop{\asym}}=2
  \end{array}
 $
\end{flushleft}
We have $\tsize{\eps\catop\starop{\sym}\shuffleop\starop{\asym}}=7\leq 5+2=\tsize{\starop{\sym}\shuffleop\starop{\asym}}+\incFunSize{\starop{\sym}\shuffleop\starop{\asym}}$, but
$\tsize{\eps\catop\starop{\sym}\shuffleop\starop{\asym}}=7\not\leq 5+2-2=\tsize{\starop{\sym}\shuffleop\starop{\asym}}+\incFunSize{\starop{\sym}\shuffleop\starop{\asym}}-\incFunSize{\eps\catop\starop{\sym}\shuffleop\starop{\asym}}$.

To guarantee the invariant $\tsize{\re'}\leq\tsize{\re}+\incFunSize{\re}-\incFunSize{\re'}$, the following definition has to be considered:
\[
 \incFunSize{\re_0\shuffleop\re_1}=\incFunSize{\re_0}+\incFunSize{\re_1} \qquad\qquad\qquad\qquad\mbox{(correct definition)}
\]
This new definition takes into account increments due to multiple reduction steps. In this way, we obtain $\tsize{\eps\catop\starop{\sym}\shuffleop\starop{\asym}}=7\leq 5+4-2=\tsize{\starop{\sym}\shuffleop\starop{\asym}}+\incFunSize{\starop{\sym}\shuffleop\starop{\asym}}-\incFunSize{\eps\catop\starop{\sym}\shuffleop\starop{\asym}}$.

The correctness of the definition of $\incFunSize{\re_0\shuffleop\re_1}$ is ensured by the claim of \cref{theo:inc-bound-size} extended to $\reSetExt$.

\subsubsection*{Proof of \cref{theo:inc-bound-size} extended to $\reSetExt$
}
For all $\re,\re'\in\reSetExt$, $\sym\in\symAlph$, if $\re\parDer{\sym}\re'$, then $\tsize{\re'}\leq \theight{\re}+\incFunSize{\re}-\incFunSize{\re'}$.

\begin{proof}
 The structure of the proof is the same, but the two new rules for the shuffle operator need to be considered.
 \begin{itemize}
  \item %%$\Rule{l-shf}{\re_0\parDer{\sym}\re'_0}{\re_0\shuffleop\re_1\parDer{\sym}\re'_0\shuffleop\re_1}{}$\\[2ex]
        rule~\rn{l-shf}: By inductive hypothesis $\tsize{\re_0'}\leq \tsize{\re_0}+\incFunSize{\re_0}-\incFunSize{\re_0'}$.

        We have $\tsize{\re'_0\shuffleop\re_1}=\tsize{\re_0'}+\tsize{\re_1}+1\leq\tsize{\re_0}+\incFunSize{\re_0}-\incFunSize{\re_0'}+\tsize{\re_1}+1=\tsize{\re_0\shuffleop\re_1}+\incFunSize{\re_0\shuffleop\re_1}-\incFunSize{\re_1}-\incFunSize{\re_0'\shuffleop\re_1}+\incFunSize{\re_1}=\tsize{\re_0\shuffleop\re_1}+\incFunSize{\re_0\shuffleop\re_1}-\incFunSize{\re_0'\shuffleop\re_1}$, by the inductive hypothesis, the definition of $\tsize{\ }$, and $\incFunSize{}$.
  \item rule (r-shf) is symmetric to rule (l-shf).
 \end{itemize}
\end{proof}

From \cref{theo:inc-bound-size} and \cref{lemma:bounds-size} (extended to $\reSetExt$) we can
extend for free \cref{theo:gen-inc-bound-size} and \cref{cor:gen-inc-bound-size}  to $\reSetExt$.

\begin{theorem}
 For all $\re,\re'\in\reSetExt$, and $\word\in\symAlph^*$, if $\re\parDer{\word}\re'$, then $\tsize{\re'}\leq\tsize{\re}+\incFunSize{\re}-\incFunSize{\re'}$.
\end{theorem}

\begin{corollary}
 For all $\re,\re'\in\reSetExt$, and $\word\in\symAlph^*$, if $\re\parDer{\word}\re'$, then $\tsize{\re'}\leq\tsize{\re}+\tsize{\re}^2$. Hence $\tsize{\re'}$ is $O(\tsize{\re}^2)$.
\end{corollary}