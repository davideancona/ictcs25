\section{Regular expressions with shuffle}\label{sec:shuffle}

\subsection{Basic definitions}
In this section we extend the syntax of regular expressions by adding the shuffle operator\footnote{In the examples we assume that the shuffle has lower precedence than the concatenation and union operator.} $\re_0\shuffleop\re_1$ whose semantics is defined as follows by extending \cref{def:langOp} and \cref{def:sem}:
\[
 \begin{array}{l}
  \emptyWord \shuffleop \word = \word \shuffleop \emptyWord = \{\word\}                                                                                                      \\
  (\sym_1\word_1) \shuffleop (\sym_2\word_2) = \{ \sym_1  \} \strcatop (\word_1 \strshuffleop (\sym_2\word_2)) \cup \{\sym_2\}\strcatop((\sym_1\word_1)\strshuffleop\word_2) \\
  \lang_1\strshuffleop\lang_2=\bigcup_{\word_1\in\lang_1,\word_2\in\lang_2}(\word_1\strshuffleop\word_2)                                                                     \\[1ex]
  \sem{\re_0\shuffleop\re_1}=\sem{\re_0}\strshuffleop\sem{\re_1}
 \end{array}
\]
We denote with $\reSetExt$ the set of regular expressions extended with the shuffle operator.

Although it is well-known that the shuffle operator does not increase the abstract expressive power of regular expressions, in practice it is still very useful in RV to write more compact and clear
specifications when correct system behaviors can be described as independently interleaved event traces.

Let us consider for instance the events $o_n$, $a_n$, and $c_n$, with the meaning
``file $n$ has been opened'', ``accessed'', and ``closed'', respectively, where $n=0,1$ is the corresponding file descriptor.
If the SUS is allowed to manage the two files independently, then a specification for the correct use of them can be defined quite concisely by the regular expression $o_0\catop\starop{a_0}\catop c_0\shuffleop o_1\catop\starop{a_1}\catop c_1$.

\begin{figure}[h]
 $$
  \begin{array}{c}
   \Rule{shf}{\re_0\der{\sym}\re'_0\quad\re_1\der{\sym}\re'_1}{\re_0\shuffleop\re_1\der{\sym}(\re'_0\shuffleop\re_1)\orop(\re_0\shuffleop\re'_1)}{}\qquad \hasEps{\re_0\shuffleop\re_1}=\hasEps{\re_0}\conj\hasEps{\re_1} \\[4ex]
   \Rule{l-shf}{\re_0\parDer{\sym}\re'_0}{\re_0\shuffleop\re_1\parDer{\sym}\re'_0\shuffleop\re_1}{} \qquad
   \Rule{r-shf}{\re_1\parDer{\sym}\re'_1}{\re_0\shuffleop\re_1\parDer{\sym}\re_0\shuffleop\re'_1}{}
  \end{array}
 $$
 \caption{Transition rules defining the derivative and the partial derivatives for the shuffle operator}
 \label{fig:shfParDer}
\end{figure}

The transition rules defining the derivative and the partial derivatives for the shuffle operator
can be found in \cref{fig:shfParDer}. They all correspond to the intuition that events can be interleaved; moreover, the empty trace is contained in $\re_0\shuffleop\re_1$ iff it is contained in $\re_0$ and $\re_1$.

As happens for the other operators, also with the shuffle  the derivative is always defined, while the partial derivative is defined only for non-zero derivatives. For instance, if we assume $\sym_2\neq\sym_0$ and $\sym_2\neq\sym_1$, then
$\sym_0\shuffleop \sym_1\der{\sym_2}(\none\shuffleop\sym_1)\orop(\sym_0\shuffleop\none)$, and $\hasEps{(\none\shuffleop\sym_1)\orop(\sym_0\shuffleop\none)}=\none$, but there exists no $\re$ s.t.
$\sym_0\shuffleop \sym_1\parDer{\sym_2}\re$.

Theorems~\ref{theo:der} and \ref{theo:parDer} still hold along with  \cref{prop:parDer} and \cref{cor:parDer}, when $\reSetExt$ is considered.

\subsection{Height of partial derivatives with the shuffle operator}

We extend the result of \cref{theo:gen-inc-bound} and \cref{cor:gen-inc-bound} to the case of the shuffle operators.

Unfortunately the proofs are more challenging because the property stating that after the first reduction step the height of the derivative of an expression $\re$ w.r.t. a symbol is bounded by $\theight{\re}$ no longer holds.
As a counter example, let us consider the following reduction steps:
\[
 \begin{array}{ll}
  \re_0\parDer{\sym}\re_1\parDer{\asym}\re_2\parDer{\sym}\re_3 & \mbox{where }\sym\neq\asym
 \end{array}
 \begin{array}[t]{ll}
  \re_0=(\eps\shuffleop\starop{\sym})\catop(\asym\shuffleop\starop{\sym}) &
  \re_1=(\eps\shuffleop\eps\catop\starop{\sym})\catop(\asym\shuffleop\starop{\sym})                                     \\
  \re_2=\eps\shuffleop\starop{\sym}                                       & \re_3=\eps\shuffleop\eps\catop\starop{\sym}
 \end{array}
\]
We have $\theight{\re_1}=\theight{\re_0}+1$ (first reduction step), but also $\theight{\re_3}=\theight{\re_2}+1$ (third reduction step). In particular, $\incFun{\re_2}=
 \max(0,1\cdot\incFun{\starop{\sym}})=1$, therefore \cref{lemma:zero-inc} no longer holds.

To prove the extended versions of the results of \cref{sec:parDer} we first extend the definition of $\incFun{}$ given in \cref{fig:incFun} with the case for the shuffle.
\[
 \incFun{\re_0\shuffleop\re_1}=\max(\geqFun{\re_0}{\re_1}\cdot\incFun{\re_0},\geqFun{\re_1}{\re_0}\cdot\incFun{\re_1})
\]
Before explaining the definition of $\incFun{}$ above, we note that the claim of \cref{lemma:bounds} holds also for $\reSetExt$ and can still be proved by induction on the definition of $\incFun{}$: $0\leq\incFun{\re}\leq 1$.
Consequently, if the height of one sub-expression $\re_i$ is strictly greater than the other $\re_{1-i}$, then only the partial derivative of $\re_i$ can contribute to the increment of the height of the partial derivative of $\re_0\shuffleop\re_1$, because increments are always bounded by 1.
Note that $\max(\incFun{\re_i},0)=\incFun{\re_i}$, since $\incFun{\re_i}\geq 0$, therefore $\incFun{\re_0\shuffleop\re_1}=\incFun{\re_i}$, if $\theight{\re_i}>\theight{\re_{1-i}}$ for $i=0,1$.

If $\theight{\re_0}=\theight{\re_1}$, then both the derivatives of $\re_0$ and $\re_1$ can contribute to the increment of the height of the partial derivative of $\re_0\shuffleop\re_1$.
Since \rn{l-shf} or \rn{r-shf} can be non-deterministically applied, the maximum increment
$\max(\incFun{\re_0},\incFun{\re_1})$ needs to be considered.

The correctness of the definition of $\incFun{\re_0\shuffleop\re_1}$ is still ensured by the claim of \cref{theo:inc-bound}.

\subsubsection*{Proof of \cref{theo:inc-bound} extended to $\reSetExt$
}
For all $\re,\re'\in\reSetExt$, $\sym\in\symAlph$, if $\re\parDer{\sym}\re'$, then $\theight{\re'}\leq \theight{\re}+\incFun{\re}$.

\begin{proof}
 The structure of the proof is the same, but the two new rules for the shuffle need to be considered.
 \begin{itemize}
  \item $\Rule{l-shf}{\re_0\parDer{\sym}\re'_0}{\re_0\shuffleop\re_1\parDer{\sym}\re'_0\shuffleop\re_1}{}$\\[2ex]
        By inductive hypothesis $\theight{\re_0'}\leq \theight{\re_0}+\incFun{\re_0}$.
        We distinguish two cases:
        \begin{itemize}
         \item $\theight{\re_0}\geq\theight{\re_1}$:
               $\theight{\re_0'\shuffleop\re_1}\stackrel{\mathrm{def}}{=}\max(\theight{\re'_0},\theight{\re_1})+1\leq\max(\theight{\re_0}+\incFun{\re_0},\theight{\re_1})+1\leq\max(\theight{\re_0},\theight{\re_1})+1+\incFun{\re_0}\leq\max(\theight{\re_0},\theight{\re_1})+1+\incFun{\re_0\shuffleop\re_1}=\theight{\re_0\shuffleop\re_1}+\incFun{\re_0\shuffleop\re_1}$, where inequalities are derived from the inductive hypothesis, the definition of $\max$, $\theight{\ }$, $\incFun{}$, and $\geqSym()$, the assumption $\theight{\re_0}\geq\theight{\re_1}$, and \cref{lemma:bounds}.

         \item $\theight{\re_0}<\theight{\re_1}$ (that is, $\theight{\re_0}+\incFun{\re_0}\leq\theight{\re_1}$ by \cref{lemma:bounds}):
               $\theight{\re_0'\shuffleop\re_1}\stackrel{\mathrm{def}}{=}\max(\theight{\re'_0},\theight{\re_1})+1\leq\max(\theight{\re_0}+\incFun{\re_0},\theight{\re_1})+1=\theight{\re_0\shuffleop\re_1}\leq\theight{\re_0\shuffleop\re_1}+\incFun{\re_0\shuffleop\re_1}$, where inequalities are derived from the inductive hypothesis, the definition of $\max$, $\theight{\ }$, $\incFun{}$, and $\geqSym()$, the assumption $\theight{\re_0}<\theight{\re_1}$, and \cref{lemma:bounds}.
        \end{itemize}
  \item rule (r-shf) is symmetric to rule (l-shf).
 \end{itemize}
\end{proof}

To be able to prove the generalization of \cref{theo:gen-inc-bound} for $\reSetExt$ two lemmas have to be introduced.

The first lemma is a weaker version of \cref{lemma:zero-inc}.

\begin{lemma}\label[lemma]{lemma:ext-zero-inc}
 For all $\re,\re'\in\reSetExt$, and $\sym\in\symAlph$, if $\re\parDer{\sym}\re'$ and $\theight{\re'}=\theight{\re}+1$, then $\incFun{\re'}=0$.
\end{lemma}
\begin{proof}
 By induction and case analysis on the rules defining $\re\parDer{\sym}\re'$.
 \begin{description}
  \item[base case:] The only base rule is \rn{sym}.
   The case is vacuous because $\re=\sym$, $\re'=\eps$, and $\theight{\sym}=\theight{\eps}=0$.
  \item[inductive step:] \hspace*{\fill}
   \begin{itemize}
    \item $\Rule{l-cat}{\re_0\parDer{\sym}\re'_0}{\re_0\catop\re_1\parDer{\sym}\re'_0\catop\re_1}{}$\\[2ex]
          From the hypothesis and the definition of $\theight{\ }$
          \begin{equation}
           \label{eq:l-cat-one}
           \max(\theight{\re_0'},\theight{\re_1})=\max(\theight{\re_0},\theight{\re_1})+1
          \end{equation}
          Therefore $\theight{\re_1}\leq\max(\theight{\re_0},\theight{\re_1})<\max(\theight{\re_0},\theight{\re_1})+1=\max(\theight{\re_0'},\theight{\re_1})$ by the definition of $\max$ and \cref{eq:l-cat-one}.

          Therefore, by the definition of $\max$
          \begin{equation}
           \label{eq:l-cat-two}
           \max(\theight{\re_0'},\theight{\re_1})=\theight{\re_0'}
          \end{equation}
          Hence $\theight{\re_0}+1\leq\max(\theight{\re_0},\theight{\re_1})+1=\theight{\re'_0}$ by the definition of $\max$, \cref{eq:l-cat-one} and \cref{eq:l-cat-two}.

          Moreover, by \cref{cor:bound} $\theight{\re_0'}\leq\theight{\re_0}+1$, therefore
          $\theight{\re_0'}=\theight{\re_0}+1$, and by inductive hypothesis $\incFun{\re'_0}=0$, which implies $\incFun{\re'_0\catop\re_1}=0$ by definition of $\incFun{}$.

    \item $\Rule{r-cat}{\re_1\parDer{\sym}\re'_1}{\re_0\catop\re_1\parDer{\sym}\re'_1}{\hasEps{\re_0}=\eps}$\\[2ex]
          This case is vacuous because $\theight{\re_1'}\leq\theight{\re_1}+1\leq\max(\theight{\re_0},\theight{\re_1})+1=\theight{\re_0\catop\re_1}$ by \cref{cor:bound} and the definition of $\max$ and $\theight{\ }$.

    \item The case for rules \rn{l-or} and \rn{r-or} is vacuous for the same reason shown for \rn{r-cat}.
    \item $\Rule{l-shf}{\re_0\parDer{\sym}\re'_0}{\re_0\shuffleop\re_1\parDer{\sym}\re'_0\shuffleop\re_1}{}$\\[2ex]
          The same proof for \rn{l-cat} shows that $\incFun{\re_0'}=\incFun{\re_0}+1$ and $\theight{\re_1}<\theight{\re'_0}$, therefore $\incFun{\re'_0}=0$ by inductive hypothesis, and $\incFun{\re'_0\shuffleop\re_1}=\max(1\cdot\incFun{\re_0'},0\cdot\incFun{\re_1})=0$ by $\theight{\re_1}<\theight{\re'_0}$ and the definition of $\max$, $\incFun{}$ and $\geqSym()$.
    \item rule \rn{r-shf} is symmetric to rule \rn{l-shf}.
    \item $\Rule{star}{\re\parDer{\sym}\re'}{\starop{\re}\parDer{\sym}\re'\catop\starop{\re}}{}$\\[2ex]
          By definition of $\theight{\ }$ and by the hypothesis $\theight{\re'\catop\starop{\re}}=\theight{\starop{\re}}+1$ we have
          $\max(\theight{\re'},\theight{\starop{\re}})+1=\theight{\starop{\re}}+1$, therefore
          $\max(\theight{\re'},\theight{\starop{\re}})=\theight{\starop{\re}}$.

          If $\theight{\re'}<\theight{\starop{\re}}$ then by the definition of $\incFun{}$ and $\geqSym()$ we have $\incFun{\re'\catop\starop{\re}}=\geqFun{\re'}{\starop{\re}}\cdot\incFun{\re'}=0\cdot\incFun{\re'}=0$.

          If $\theight{\re'}\geq\theight{\starop{\re}}$ then $\theight{\re'}=\theight{\starop{\re}}$, since $\max(\theight{\re'},\theight{\starop{\re}})=\theight{\starop{\re}}$.
          By the definition of $\theight{\ }$ we have $\theight{\re'}=\theight{\re}+1$, therefore
          by inductive hypothesis $\incFun{\re'}=0$ and by definition of $\incFun{}$ and $\geqSym()$ we have $\incFun{\re'\catop\starop{\re}}=\geqFun{\re'}{\starop{\re}}\cdot\incFun{\re'}=1\cdot\incFun{\re'}=0$.
   \end{itemize}
 \end{description}
\end{proof}

The second lemma establishes an invariant on $\incFun{}$ when the height of the derivatives does not change.

\begin{lemma}\label[lemma]{lemma:ext-leq-inc}
 For all $\re,\re'\in\reSetExt$, and $\sym\in\symAlph$, if $\re\parDer{\sym}\re'$ and $\theight{\re'}=\theight{\re}$, then $\incFun{\re'}\leq\incFun{\re}$.
\end{lemma}
\begin{proof}
 By induction and case analysis on the rules defining $\re\parDer{\sym}\re'$.
 \begin{description}
  \item[base case:] The only base rule is \rn{sym}.
   In this case $\re=\sym$, $\re'=\eps$, therefore $\incFun{\eps}=0=\incFun{\sym}$.
  \item[inductive step:] \hspace*{\fill}
   \begin{itemize}
    \item $\Rule{l-cat}{\re_0\parDer{\sym}\re'_0}{\re_0\catop\re_1\parDer{\sym}\re'_0\catop\re_1}{}$\\[2ex]
          From the hypothesis and the definition of $\theight{\ }$
          \begin{equation}
           \label{eq:l-cat-three}
           \max(\theight{\re_0'},\theight{\re_1})=\max(\theight{\re_0},\theight{\re_1})
          \end{equation}
          Two different cases may occur:
          \begin{itemize}
           \item $\theight{\re_0}\geq\theight{\re_1}$\\
                 From \cref{eq:l-cat-three} and the definition of $\max$, $\max(\theight{\re_0'},\theight{\re_1})=\theight{\re_0}$, hence $\theight{\re_0'}\leq\theight{\re_0}$ by the definition of $\max$.

                 If $\theight{\re_0'}<\theight{\re_0}$, then $\theight{\re_0'}<\theight{\re_1}$ by \cref{eq:l-cat-three} and the definition of $\max$. Therefore $\incFun{\re_0'\catop\re_1}=0\leq\incFun{\re_0\catop\re_1}$ by definition of $\incFun{}$ and \cref{lemma:bounds}.

                 Therefore $\theight{\re_1}\leq\max(\theight{\re_0},\theight{\re_1})<\max(\theight{\re_0},\theight{\re_1})+1=\max(\theight{\re_0'},\theight{\re_1})$ by the definition of $\max$ and \cref{eq:l-cat-one}.

                 If $\theight{\re_0'}=\theight{\re_0}$, then $\incFun{\re_0'}\leq\incFun{\re_0}$ by inductive hypothesis. Since $\theight{\re_0'}=\theight{\re_0}\geq\theight{\re_1}$, we have $\incFun{\re_0'\catop\re_1}=\incFun{\re_0'}\leq\incFun{\re_0}=\incFun{\re_0\catop\re_1}$ by the definition of $\incFun{}$ and $\geqSym()$.
           \item $\theight{\re_0}<\theight{\re_1}$\\
                 In this case $\incFun{\re_0\catop\re_1}=0$ by the definition of $\incFun{}$ and $\geqSym()$. Moreover, from \cref{eq:l-cat-three} $\max(\theight{\re_0'},\theight{\re_1})=\theight{\re_1}$, hence $\theight{\re_0'}\leq\theight{\re_1}$ by the definition of $\max$.

                 If $\theight{\re_0'}<\theight{\re_1}$ then $\incFun{\re_0'\catop\re_1}=0=\incFun{\re_0\catop\re_1}$ by the definition of $\incFun{}$ and $\geqSym()$. If $\theight{\re_0'}=\theight{\re_1}$ then $\theight{\re_0}<\theight{\re_1}=\theight{\re_0'}$, therefore $\theight{\re_0}+1\leq\theight{\re_0'}$. Moreover, by \cref{cor:bound} $\theight{\re'_0}\leq\theight{\re_0}+1$, hence $\theight{\re_0'}=\theight{\re_0}+1$, and, by \cref{lemma:ext-zero-inc}, $\incFun{\re_0'}=0$. Finally,
                 $\incFun{\re_0'\catop\re_1}=\incFun{\re_0'}=0=\incFun{\re_0\catop\re_1}$ by the definition of $\incFun{}$ and $\geqSym()$.
          \end{itemize}

    \item $\Rule{r-cat}{\re_1\parDer{\sym}\re'_1}{\re_0\catop\re_1\parDer{\sym}\re'_1}{\hasEps{\re_0}=\eps}$\\[2ex]
          From the hypothesis and the definition of $\theight{\ }$
          \begin{equation}
           \label{eq:l-cat}
           \theight{\re_1'}=\max(\theight{\re_0},\theight{\re_1})+1
          \end{equation}
          Two different cases may occur:
          \begin{itemize}
           \item $\theight{\re_0}\geq\theight{\re_1}$\\
                 From \cref{eq:l-cat} and the definition of $\max$, $\theight{\re_1'}=\max(\theight{\re_0},\theight{\re_1})+1\geq\theight{\re_1}+1$. Moreover, by \cref{lemma:bounds},
                 $\theight{\re_1'}\leq\theight{\re}+1$, therefore $\theight{\re_1'}=\theight{\re}+1$, hence by \cref{lemma:ext-zero-inc} $\incFun{\re_1'}=0$.
                 Finally, $\incFun{\re_1'}=0\leq\incFun{\re_0\catop\re_1}$ by \cref{lemma:bounds}.

           \item $\theight{\re_0}<\theight{\re_1}$\\
                 In this case $\incFun{\re_0\catop\re_1}=0$ by the definition of $\incFun{}$ and $\geqSym()$. Moreover, from \cref{eq:l-cat} and the definition of $\max$, $\theight{\re_1'}=\theight{\re_1}+1$, hence  by \cref{lemma:ext-zero-inc}, $\incFun{\re_1'}=0$. Finally,
                 $\incFun{\re_1'}=0\leq\incFun{\re_0\catop\re_1}$ by \cref{lemma:bounds}.
          \end{itemize}

    \item The proofs for the rules \rn{l-or} and \rn{r-or} are analogous to the proof for the rule \rn{r-cat}.
    \item $\Rule{l-shf}{\re_0\parDer{\sym}\re'_0}{\re_0\shuffleop\re_1\parDer{\sym}\re'_0\shuffleop\re_1}{}$\\[2ex]
          The proof is similar to that for \rn{l-cat}, except from some details on the definition of $\incFun{\re_0\shuffleop\re_1}$, which differs from that of $\incFun{\re_0\catop\re_1}$. We report it for the sake of completeness.

          From the hypothesis and the definition of $\theight{\ }$
          \begin{equation}
           \label{eq:l-shuffle}
           \max(\theight{\re_0'},\theight{\re_1})=\max(\theight{\re_0},\theight{\re_1})
          \end{equation}
          Two different cases may occur:
          \begin{itemize}
           \item $\theight{\re_0}\geq\theight{\re_1}$\\
                 From \cref{eq:l-shuffle} and the definition of $\max$, $\max(\theight{\re_0'},\theight{\re_1})=\theight{\re_0}$, hence $\theight{\re_0'}\leq\theight{\re_0}$ by the definition of $\max$.

                 If $\theight{\re_0'}<\theight{\re_0}$, then $\theight{\re_0'}<\theight{\re_1}=\theight{\re_0}$ by \cref{eq:l-shuffle} and the definition of $\max$. Therefore $\geqFun{\re_0'}{\re_1}= 0$ and $\geqFun{\re_1}{\re_0'}=\geqFun{\re_0}{\re_1}=\geqFun{\re_1}{\re_0}=1$ by the definition of $\geqSym()$, and
                 $\incFun{\re_0'\shuffleop\re_1}=\incFun{\re_1}\leq\max(\incFun{\re_0},\incFun{\re_1})=\incFun{\re_0\shuffleop\re_1}$ by the definition of $\incFun{}$ and $\max$.


                 If $\theight{\re_0'}=\theight{\re_0}$, then $\incFun{\re_0'}\leq\incFun{\re_0}$ by inductive hypothesis. Moreover,
                 $\geqFun{\re_0}{\re_1}=\geqFun{\re'_0}{\re_1}=1$ and $\geqFun{\re_1}{\re_0}=\geqFun{\re_1}{\re'_0}$ by the definition of $\geqSym()$.
                 Therefore, $\incFun{\re_0'\shuffleop\re_1}=\max(\incFun{\re'_0},\geqFun{\re_1}{\re_0'}\cdot\incFun{\re_1})\leq\max(\incFun{\re_0},\geqFun{\re_1}{\re_0}\cdot\incFun{\re_1})=\incFun{\re_0\shuffleop\re_1}$ by the definition of $\max$.
           \item $\theight{\re_0}<\theight{\re_1}$\\
                 In this case $\incFun{\re_0\shuffleop\re_1}=\incFun{\re_1}$ by the definition of $\incFun{}$ and $\geqSym()$. Moreover, from \cref{eq:l-shuffle} $\max(\theight{\re_0'},\theight{\re_1})=\theight{\re_1}$, hence $\theight{\re_0'}\leq\theight{\re_1}$ by the definition of $\max$.

                 If $\theight{\re_0'}<\theight{\re_1}$ then $\incFun{\re_0'\shuffleop\re_1}=\incFun{\re_1}=\incFun{\re_0\shuffleop\re_1}$ by the definition of $\incFun{}$ and $\geqSym()$. If $\theight{\re_0'}=\theight{\re_1}$ then $\theight{\re_0}<\theight{\re_1}=\theight{\re_0'}$, therefore $\theight{\re_0}+1\leq\theight{\re_0'}$. Moreover, by \cref{cor:bound} $\theight{\re'_0}\leq\theight{\re_0}+1$, hence $\theight{\re_0'}=\theight{\re_0}+1$, and, by \cref{lemma:ext-zero-inc}, $\incFun{\re_0'}=0$. Finally,
                 $\incFun{\re_0'\shuffleop\re_1}=\max(\incFun{\re_0'},\incFun{\re_1})=\max(0,\incFun{\re_1})=\incFun{\re_0\shuffleop\re_1}$ by the definition of $\incFun{}$ and $\geqSym()$.
          \end{itemize}


    \item The rule \rn{r-shf} is symmetric to \rn{l-shf}.
    \item $\Rule{star}{\re\parDer{\sym}\re'}{\starop{\re}\parDer{\sym}\re'\catop\starop{\re}}{}$\\[2ex]
          Directly by the definition of $\incFun{}$ and \cref{lemma:bounds}
          $\incFun{\re'\catop\starop{\re}}\leq 1=\incFun{\starop{\re}}$.
   \end{itemize}
 \end{description}
\end{proof}

It is now possible to prove \cref{theo:gen-inc-bound} extended to $\reSetExt$.

\subsubsection*{Proof of \cref{theo:gen-inc-bound} extended to $\reSetExt$
}
For all $\re,\re'\in\reSetExt$, $\word\in\symAlph^*$, if $\re\parDer{\word}\re'$, then $\theight{\re'}\leq \theight{\re}+\incFun{\re}$.

\begin{proof}
 By induction on the length of $\word$.
 \begin{description}
  \item[base case:] if $\word=\emptyWord$, then by definition $\re'=\re$, hence
   $\theight{\re'}=\theight{\re}$, and by \cref{lemma:bounds}, $\theight{\re'}=\theight{\re}\leq\theight{\re}+\incFun{\re}$.

  \item[inductive step:]
   if $\word=\sym\strcons\word'$ for some $\sym\in\symAlph$, $\word'\in\symAlph^*$, then by definition there exists $\re''\in\reSet$ s.t. $\re\parDer{\sym}\re''$ and $\re''\parDer{\word'}\re'$. By \cref{theo:inc-bound} $\theight{\re''}\leq\theight{\re}+\incFun{\re}$ and by inductive hypothesis $\theight{\re'}\leq\theight{\re''}+\incFun{\re''}$. Therefore,
   by transitivity,  $\theight{\re'}\leq\theight{\re}+\incFun{\re}+\incFun{\re''}$.

   Since by \cref{cor:bound} $\theight{\re''}\leq\theight{\re}+1$, we can distinguish three cases:
   \begin{itemize}
    \item $\theight{\re''}=\theight{\re}+1$: by \cref{lemma:ext-zero-inc}, $\incFun{\re''}=0$,
          therefore $\theight{\re'}\leq\theight{\re}+\incFun{\re}+\incFun{\re''}=\theight{\re}+\incFun{\re}$.
    \item $\theight{\re''}=\theight{\re}$: by inductive hypothesis and \cref{lemma:ext-leq-inc},
          $\theight{\re'}\leq\theight{\re''}+\incFun{\re''}=\theight{\re}+\incFun{\re''}\leq\theight{\re}+\incFun{\re}$.
    \item $\theight{\re''}<\theight{\re}$ (that is, $\theight{\re''}\leq\theight{\re}-1$): by inductive hypothesis and \cref{lemma:bounds},
          $\theight{\re'}\leq\theight{\re''}+\incFun{\re''}\leq\theight{\re}-1+\incFun{\re''}\leq\theight{\re}\leq\theight{\re}+\incFun{\re}$.
   \end{itemize}
 \end{description}
\end{proof}

Finally, from the versions of \cref{theo:gen-inc-bound} and \cref{lemma:bounds}, both extended to $\reSetExt$, \cref{cor:gen-inc-bound} can be directly extended as well: For all $\re,\re'\in\reSetExt$, and $\word\in\symAlph^*$, if $\re\parDer{\word}\re'$, then $\theight{\re'}\leq\theight{\re}+1$.