%% The first command in your LaTeX source must be the \documentclass command.
%%
%% Options:
%% twocolumn : Two column layout.
%% hf: enable header and footer.
\documentclass{ceurart}

%%
%% One can fix some overfulls
\sloppy

%%
%% Minted listings support 
%% Need pygment <http://pygments.org/> <http://pypi.python.org/pypi/Pygments>
\usepackage{listings}
%% auto break lines
\lstset{breaklines=true,mathescape=true,basicstyle={\ttfamily\small},keywordstyle={\ttfamily\bfseries}}
\usepackage{centernot}
\usepackage{graphicx}
\usepackage{proof} % compact proof trees for examples
\usepackage{hyperref} % links
\usepackage{verbatim} % multiline comments
\usepackage[dvipsnames]{xcolor}
\usepackage{mathtools}
\usepackage{amssymb}
\usepackage{stmaryrd}
%\usepackage{xspace}
\usepackage{color}
\usepackage{amssymb}
\usepackage{pifont}
%\usepackage{amsmath}
%\usepackage{amsthm}
\usepackage{pifont}
\usepackage{stmaryrd}
%\usepackage{xspace}
\usepackage{listings}
\usepackage{color}
\usepackage{pgfplots}
\usepackage{subcaption}
\usepackage{cleveref}
%
%%
%% end of the preamble, start of the body of the document source.

% editing macros
\newif\ifsubmit
\submitfalse
%\submittrue

\ifsubmit
 \newcommand{\DA}[1]{{#1}}
 \newcommand{\DAComm}[1]{}
 \newcommand{\AF}[1]{{#1}}
 \newcommand{\AFComm}[1]{}
\else
 \newcommand{\DA}[1]{\textcolor{blue}{#1}}
 \newcommand{\DAComm}[1]{{\scriptsize\textcolor{blue}{[\bf{Angelo: }#1}]}}
 \newcommand{\AF}[1]{\textcolor{red}{#1}}
 \newcommand{\AFComm}[1]{{\scriptsize\textcolor{red}{[\bf{Angelo: }#1}]}}
\fi

%environments

\newtheorem{definition}{Definition}[section]
\newtheorem{theorem}[definition]{Theorem}
\newtheorem{lemma}[definition]{Lemma}
\newtheorem{corollary}[definition]{Corollary}
\newtheorem{proposition}[definition]{Proposition}

\newenvironment{proof}{\textbf{Proof:}\\}{\hspace*{\fill}$\Box$}

%matematiche
\newcommand{\dom}[1]{\mathit{dom}(#1)}
\newcommand{\Tuple}[1]    {({#1})}
\newcommand{\Pair}[2]     {\Tuple{{#1},{#2}}}
%\newcommand{\Triple}[3]     {\Tuple{{#1},{#2},{#3}}}
%\newcommand{\fv}[1]{\textit{fv}(#1)}
%\newcommand{\vars}[1]{\textit{vars}(#1)}
\newcommand{\simpleRule}[3]{\displaystyle\frac{#1}{#2}\ \begin{array}{l} #3 \end{array}}
\newcommand{\rn}[1]{\textsc{({#1})}} %% rule name
\newcommand{\Rule}[4]{{\tiny{\rn{#1}}}\displaystyle\frac{#2}{#3}\ \begin{array}{l} #4 \end{array}}
\newcommand{\infeRule}[3]{\infer[\tiny\textsc{({#1})}]{#3}{#2}}

%syntax
\newenvironment{grammar}{$\begin{array}[t]{llll}}{\end{array}$}
\newcommand{\production}[3]{#1&{:}{:}=&#2 & \mbox{{\small{#3}}}}
\newcommand{\nlproduction}[1]{&&#1}
\newcommand{\term}[1]{\ensuremath{\mathtt {#1}}}
\newcommand{\nonterm}[1]{\ensuremath{\mathit {#1}}}
\newcommand{\re}{\nonterm{e}}
\newcommand{\reSet}{\nonterm{RE}}
\newcommand{\reSetExt}{\nonterm{RE+}}
\newcommand{\eps}{\term{\epsilon}}
\newcommand{\none}{\term{0}}
\newcommand{\sym}{\term{a}}
\newcommand{\asym}{\term{b}}
\newcommand{\all}{\term{1}}
\newcommand{\orop}{\mathbin{+}}
\newcommand{\shuffleop}{\mathbin{||}}
\newcommand{\catop}{}
\newcommand{\strcons}{\mathbin{:}}
\newcommand{\strcatop}{\mathbin{\cdot}}
\newcommand{\strshuffleop}{\shuffleop}
\newcommand{\andop}{\mathbin{\&}}
\newcommand{\starop}[1]{{#1}^*}
\newcommand{\symAlph}{\Sigma}

%derivative
\newcommand{\der}[1]{\xrightarrow{#1}}
\newcommand{\hasEps}[1]{\nu({#1})}
\newcommand{\conj}{\mathbin{\mathsf{and}}}
\newcommand{\disj}{\mathbin{\mathsf{or}}}
\newcommand{\noDetDer}[2]{\xhookrightarrow{#1}_{#2}}
\newcommand{\parDer}[1]{\xrightharpoonup{#1}}
\newcommand{\noParDer}[1]{\centernot{\parDer{#1}}}
\newcommand{\derSet}[1]{\xRightarrow{#1}}
\newcommand{\emptyPath}{\lambda}
\newcommand{\pathCons}{\mathbin{:}}
\newcommand{\tpath}{p} %% generic tree path
\newcommand{\pathEl}{b} %% generic path element
\newcommand{\incSym}{\Delta_{\max}}
\newcommand{\incFun}[1]{{\incSym}({#1})}
\newcommand{\incConst}{\mathit{k}}
\newcommand{\theight}[1]{{\mid}{#1}{\mid}} %% tree height
\newcommand{\gtSym}{\mathit{gt}}
\newcommand{\gtFun}[2]{\gtSym({#1},{#2})}
\newcommand{\geqSym}{\mathit{geq}}
\newcommand{\geqFun}[2]{\geqSym({#1},{#2})}

%semantics
\newcommand{\word}{\nonterm{w}}
\newcommand{\emptyWord}{\lambda}
\newcommand{\sem}[1]{\left\llbracket{#1}\right\rrbracket}
\newcommand{\opSem}[2]{\left\llbracket{#1}\right\rrbracket_{#2}}
\newcommand{\wordUniv}{\starop{\symAlph}}
\newcommand{\first}[1]{\nonterm{first}(#1)}
\newcommand{\lang}{\nonterm{L}}


\begin{document}

\conference{ICTCS 2025}

%
\title{On The Space Complexity of Partial Derivatives of Regular Expressions with Shuffle}

\author[1]{Davide Ancona}[orcid=0000-0002-6297-2011,email=davide.ancona@unige.it]

\cormark[1]

\address[1]{University of Genova, Italy}

\author[2]{Angelo Ferrando}[orcid=0000-0002-8711-4670,
 email=angelo.ferrando@unimore.it
]

\address[2]{University of Modena and Reggio Emilia, Italy}


%% Footnotes
\cortext[1]{Corresponding author.}

\begin{keywords}
 partial derivatives \sep
 regular expressions with shuffle \sep
 rewriting-based runtime verification \sep
 space complexity
\end{keywords}

\maketitle
%
\begin{abstract}
 Partial derivatives of regular expressions are an elegant formalism introduced by Antimirov to define an algorithm that generates, from regular expressions, equivalent non-deterministic finite automata (NFA) with a limited number of states.

 While Antimirov's work addresses the classical problem of word recognition in regular languages, here we are interested in runtime verification (RV) of simple properties expressible with regular expressions. In this context, the words to be recognized are finite traces of monitorable events, which define the alphabet of the language, and the corresponding NFA generated from the regular expression may have an intractable number of states.

 This often occurs because the properties to be verified consist of sub-traces of mutually independent events, which are allowed to interleave during the execution of the system under scrutiny (SUS). For this reason, regular expressions used for RV are extended with the shuffle operator. Although regular languages are closed under shuffle, the operator allows for more readable and compact specifications.

 By exploiting partial derivatives, it is possible to follow a rewriting-based approach to RV, where only a single partial derivative needs to be stored at each reduction step, instead of generating a finite automaton with an intractable number of states.

 This raises the question of the space complexity of the largest partial derivative that can be generated. While the upper bound on the total number of generated partial derivatives has been shown to be linear in the size of the initial regular expression, no results can be found in the literature regarding the size of the largest derivative.

 In this paper, we investigate this problem with respect to two different metrics of regular expressions: their height and their total number of nodes when considered as trees. In particular, we show that the height of the largest partial derivative can increase by at most one, while the number of nodes of the largest partial derivative is bounded by $n^2$, where $n$ is the number of nodes of the initial regular expression.

 Surprisingly, these results still hold when regular expressions are extended with shuffle.
\end{abstract}
%
%
\section{Derivatives for regular expressions: a standard deterministic definition}\label{regExp}

\begin{figure}[h]
 \begin{small}
  \begin{center}
   \begin{grammar}
    \production{\re}{\none \mid \eps \mid \sym \mid \re_0\catop\re_1 \mid \re_0\orop\re_1 \mid \starop{\re}}{with $\sym\in\symAlph$} \\
   \end{grammar}
  \end{center}
 \end{small}
 \caption{Syntax of regular expressions}
 \label{regExpSyn}
\end{figure}

\begin{definition}
 \[
  \word^{-1}(\lang) = \{ \word' \mid \word\strcatop\word'\in\lang \}
 \]
\end{definition}

\begin{figure}[h]
 \begin{center}
  $$
   \begin{array}{c}
    \Rule{empty}{}{\none\der{\sym}\none}\quad
    \Rule{eps}{}{\eps\der{\sym}\none}\quad
    \Rule{sym-eq}{}{\sym\der{\sym}\eps}\quad
    \Rule{sym-neq}{}{\sym\der{\asym}\none}{}                                 \\[4ex]
    \Rule{or}{\re_0\der{\sym}\re'_0\quad\re_1\der{\sym}\re'_1}{\re_0\catop\re_1\der{\sym}\re'_0\catop\re_1\orop\hasEps{\re_0}\catop\re'_1}{} \quad
    \Rule{cat}{\re_0\der{\sym}\re'_0\quad\re_1\der{\sym}\re'_1}{\re_0\orop\re_1\der{\sym}\re'_0\orop\re'_1}{}\quad
    \Rule{star}{\re\der{\sym}\re'}{\starop{\re}\der{\sym}\re'\starop{\re}}{} \\[8ex]
    \hasEps{\eps}=\hasEps{\starop{\re_0}}=\eps\qquad\hasEps{\none}=\hasEps{\sym}=\none \quad
    \hasEps{\re_0\catop\re_1}=\hasEps{\re_0}\conj\hasEps{\re_1}\quad
    \hasEps{\re_0\orop\re_1}=\hasEps{\re_0}\disj\hasEps{\re_1}               \\[2ex]
    %% \eps\conj\eps=\eps\quad\none\conj\re=\re\conj\none=\none\qquad\none\disj\none=\none\quad\eps\disj\re=\re\disj\eps=\eps
   \end{array}
  $$
  \begin{tabular}{ccc}
   \begin{tabular}{|c|c|c|}
    \hline
    $\conj$ & $\quad\none\quad$ & $\quad\eps\quad$ \\
    \hline
    $\none$ & $\none$           & $\none$          \\
    \hline
    $\eps$  & $\none$           & $\eps$           \\
    \hline
   \end{tabular}
    & \qquad &
   \begin{tabular}{|c|c|c|}
    \hline
    $\disj$ & $\quad\none\quad$ & $\quad\eps\quad$ \\
    \hline
    $\none$ & $\none$           & $\eps$           \\
    \hline
    $\eps$  & $\eps$            & $\eps$           \\
    \hline
   \end{tabular}
  \end{tabular}
 \end{center}

 \caption{Transition system defining the derivatives of regular expressions}
 \label{deriv}
\end{figure}

\begin{definition}
 \[
  \begin{array}{l}
   \re\der{\emptyWord}\re \\
   \re_0\der{\sym\strcatop\word}\re_1 \mbox{ iff } \re_0\der{\sym}\re \mbox{ and } \re\der{\word}\re_1
   \\
   \opSem{\re}{\der{}}=\{\word\in\wordUniv \mid \mbox{there exists } \re' \mbox{ s.t. } \re\der{\word}\re' \mbox{ and } \hasEps{\re'}=\eps\}
  \end{array}
 \]
\end{definition}

~\Cref{deriv} contains the transition rules defining the derivatives of regular expressions.
Except for the presentation style, the definition coincides with that introduced by Brzozowski~\cite{Brzozowski64}. By using the notation of Antimirov \cite{Antimirov96}, we have
that $\sym^{-1}(\re_0)=\re_1$ iff $\re_0\der{\sym}\re_1$, and $\word^{-1}(\re_0)=\re_1$ iff $\re_0\der{\word}\re_1$.


\begin{definition}
 \[
  \begin{array}{l}
   \lang_1\strcatop\lang_2=\{\word_1\strcatop\word_2\mid \word_1\in\lang_1,\word_2\in\lang_2\} \\
   \lang^0=\{\emptyWord\}, \lang^{n+1}=\lang\strcatop\lang^n \mbox{for all $n\geq 0$}          \\
   \starop{\lang} = \bigcup_{n\geq 0}\lang^n
  \end{array}
 \]
\end{definition}
\begin{definition}
 \[
  \begin{array}{l}
   \sem{\none}=\emptyset\quad \sem{\eps}=\{\emptyWord\} \quad \sem{\sym}=\{\sym\} \quad
   \sem{\re_0\catop\re_1}=\sem{\re_0}\strcatop\sem{\re_1} \quad \sem{\re_0\orop\re_1}=\sem{\re_0}\cup\sem{\re_1} \\ \sem{\starop{\re_0}}=\starop{\sem{\re_0}}
  \end{array}
 \]
\end{definition}

\begin{theorem}
 For all regular expressions $\re$
 \[
  \begin{array}{l}
   \hasEps{\re} = \eps \mbox{ iff } \emptyWord\in\sem{\re}      \\
   \re\der{\word}\re' \implies \word^{-1}(\sem{\re})=\sem{\re'} \\
   \sem{\re}=\opSem{\re}{\der{}}
  \end{array}
 \]
\end{theorem}


\section{Partial derivatives of regular expressions}\label{sec:parDer}

\begin{figure}
 $$
  \begin{array}{c}
   \Rule{sym}{}{\sym\parDer{\sym}\eps}\quad
   \Rule{l-cat}{\re_0\parDer{\sym}\re'_0}{\re_0\catop\re_1\parDer{\sym}\re'_0\catop\re_1}{}\quad
   \Rule{r-cat}{\re_1\parDer{\sym}\re'_1}{\re_0\catop\re_1\parDer{\sym}\re'_1}{\hasEps{\re_0}=\eps} \\[4ex]
   \Rule{l-or}{\re_0\parDer{\sym}\re'_0}{\re_0\orop\re_1\parDer{\sym}\re'_0}{} \quad
   \Rule{r-or}{\re_1\parDer{\sym}\re'_1}{\re_0\orop\re_1\parDer{\sym}\re'_1}{} \quad
   \Rule{star}{\re\parDer{\sym}\re'}{\starop{\re}\parDer{\sym}\re'\catop\starop{\re}}{}
  \end{array}
 $$
 \caption{Transition system defining the partial derivatives of regular expressions}
 \label{fig:parDer}
\end{figure}

~\Cref{parDer} contains the transition rules defining the partial derivatives of regular expressions.
As for the definition of derivatives, the presentation deviates from that of Antimirov \cite{Antimirov96}, and keeps following the transition system style. Besides the different notation, more importantly, here we deliberately define a non-deterministic system, where a single transition step yields a single (non-zero) partial derivative among all possible ones, while in Antimirov's definition the set of all partial derivatives is returned.

For instance, according to Antimirov's definition\footnote{See \cite{Antimirov96}, definition 2.8.},
the partial derivative of $ab \orop ac$ \wrt~ $a$ returns the set of regular expressions $\{b,c\}$ (that is, $\delta_a(ab \orop ac)=\{b,c\}$ with Antimirov's notation). This corresponds to the fact that there exist two possible transition steps from $ab \orop ac$ labeled with $a$, namely,
$ab \orop ac\parDer{a}b$ and $ab \orop ac\parDer{a}c$.

This difference is directly related to the opposite aim of our work: while Antimirov's approach uses partial derivatives for generating an NFA, here
we use them to define a non-deterministic transition system which already represents the corresponding NFA.

\begin{definition}
 For all $\re,\re_0,\re_1\in\reSet$, $\sym\in\symAlph$, $\word\in\symAlph^*$
 \[
  \begin{array}{l}
   \\
   \re\parDer{\emptyWord}\re                                                                                       \\
   \re_0\parDer{\sym\strcons\word}\re_1 \mbox{ iff } \re_0\parDer{\sym}{}\re \mbox{ and } \re\parDer{\word}{}\re_1 \\
   \opSem{\re}{\parDer{}}=\{\word\in\wordUniv \mid \mbox{there exists } \re'\in\reSet \mbox{ s.t. } \re\parDer{\word}\re' \mbox{ and } \hasEps{\re'}=\eps\}
  \end{array}
 \]
\end{definition}


\begin{proposition}\label[proposition]{prop:parDer}
 Let $\delta_\word(\re)$ denotes the set of all partial derivatives of $\re\in\reSet$ \wrt~ $\word\in\symAlph^*$.

 Then $\delta_\word(\re)=\{\re'\in\reSet \mid \re\parDer{\word}\re'\}$.
\end{proposition}

% \begin{definition}
%  Let $\reSet_1$ and $\reSet_2$ be sets of regular expressions.
%  \[
%   \begin{array}{l}
%    \reSet_1\derSet{\sym}\reSet_2 \mbox{ iff } \reSet_2=\{\re_1 \mid \mbox{ there exists } \re_0\in\reSet_1 \mbox{ s.t. } \re_0\parDer{\sym}{}\re_1 \} \\
%    \hasEps{\reSet}=\eps \mbox{ iff there exists } \re\in\reSet \mbox{ s.t. } \hasEps{\re}                                                             \\
%    \re_0\derSet{\sym\word}\re_1 \mbox{ iff } \re_0\derSet{\sym}\re \mbox{ and } \re\derSet{\word}\re_1                                                \\
%    \re\derSet{\emptyWord}{}\re                                                                                                                        \\
%    \opSem{\re}{\derSet{}}=\{\word\in\wordUniv \mid \mbox{there exists } \reSet \mbox{ s.t. } \{\re\}\derSet{\word}\reSet \mbox{ and } \hasEps{\reSet}=\eps\}
%   \end{array}
%  \]
% \end{definition}

\begin{theorem}\label[theorem]{theo:parDer}
 For all $\re\in\reSet$
 \[
  \opSem{\re}{\parDer{}{}}=\opSem{\re}{\der{}}
 \]
\end{theorem}

\begin{corollary}\label[corollary]{cor:parDer}
 For all $\re\in\reSet$
 \[
  \opSem{\re}{\parDer{}{}}=\sem{\re}
 \]
\end{corollary}
% \begin{theorem}
%  For all regular expressions $\re$
%  \[
%   \opSem{\re}{\derSet{}}=\opSem{\re}{\der{}}
%  \]
% \end{theorem}

\begin{figure}
 $$
  \begin{array}{c}
   \theight{\eps}=\theight{\sym}=0                                                              \\
   \theight{\re_0\catop\re_1}=\theight{\re_0\orop\re_1}=\max(\theight{\re_0},\theight{\re_1})+1 \\
   \theight{\starop{\re_0}}=\theight{\re_0}+1
  \end{array}
 $$
 \caption{Definition of the height of regular expressions}
 \label{fig:height}
\end{figure}

\begin{figure}
 $$
  \begin{array}{l}
   % \gtFun{\re_0}{\re_1}=
   % \begin{cases}
   %   1, & \text{if } \theight{\re_0} > \theight{\re_1} \\
   %   0, & \text{ otherwise}
   % \end{cases} \quad
   \geqFun{\re_0}{\re_1}=
   \begin{cases}
    1, & \text{if } \theight{\re_0} \geq \theight{\re_1} \\
    0, & \text{if } \theight{\re_0} < \theight{\re_1}
   \end{cases} \\[4ex]
   \incFun{\sym}=\incFun{\eps}=0                                      \qquad
   \incFun{\re_0\catop\re_1}=\geqFun{\re_0}{\re_1}\cdot\incFun{\re_0} \qquad
   \incFun{\re_0\orop\re_1}=0                                         \qquad
   \incFun{\starop{\re_0}}=1
  \end{array}
 $$
 \caption{Definition of the increment function $\incFun{\re}$}
 \label{fig:incFun}
\end{figure}

\subsection{Height of partial derivatives}
In this section we investigate the upper bound on the hight of the partial derivatives of a given regular expression.

The standard definition of the height of a regular expression considered as a tree can be found in \cref{fig:height}.

The height of the partial derivative of an expression $\re$  \wrt~ a word $\word$ both depends on the shape of $\re$ and on $\word$.
Here we want to reason on any possible $\re$ and $\word$. Moreover, we want to provide a general proof methodology which can be followed to prove results when considering also shuffle and the size of expressions.

To show a couple of examples let us consider the following reduction steps which compute the partial derivatives of $\starop{\sym}\catop\starop{\asym}$~
\wrt~ $\sym\asym$ and $\asym\asym$:
\begin{flushleft}
 $
  \begin{array}{l}
   \starop{\sym}\catop\starop{\asym}\parDer{\sym}(\eps\catop\starop{\sym})\catop\starop{\asym}\parDer{\asym}\eps\catop\starop{\asym} \\
   \starop{\sym}\catop\starop{\asym}\parDer{\asym}\eps\catop\starop{\asym}\parDer{\asym}\eps\catop\starop{\asym}                     \\
   \theight{\starop{\sym}\catop\starop{\asym}}=2\qquad \theight{(\eps\catop\starop{\sym})\catop\starop{\asym}}=3 \qquad \theight{\eps\catop\starop{\asym}}=2
  \end{array}
 $
\end{flushleft}
In the first example the height increases by one at the first step and decreases by one at the second, while in the second example the height of the expressions is unchanged.

What we are going to prove are the following main properties:
\begin{enumerate}
 \item the height of the partial derivative of a regular expression \wrt~ a symbol can increase \textbf{at most by one};
 \item the height of the partial derivative of a partial derivative \wrt~a non-empty word, \textbf{cannot} increase;
 \item from 1. and 2. and from the definition of partial derivative, one can deduce that if $\re\parDer{\word}\re'$, then $\theight{\re'}\leq\theight{\re}+1$.
\end{enumerate}
Although these properties can be proved in a direct way, they are quite strong, and do not hold when shuffle or the size of expressions is considered. Therefore, as motived above, more general and weaker claims will be proved.

In order to do that, \cref{fig:incFun} defines the function $\incFun{\re}$ which computes the upper bound of $\theight{\re'}-\theight{\re}$, where
$\re\parDer{\sym}\re'$, with $\re'\in\reSet$, $\sym\in\symAlph$.
That is, $M\stackrel{\mathrm{def.}}{=}\max\{\theight{\re'}-\theight{\re} \mid \re\parDer{\sym}\re', \re'\in\reSet, \sym\in\symAlph\}\leq\incFun{\re}$.

Before commenting the definition of $\incFun{\re}$, we introduce the following straightforward lemma.
\begin{lemma}\label[lemma]{lemma:bounds}
 For all $\re\in\reSet$
 \[0\leq\incFun{\re}\leq 1\]
\end{lemma}
\begin{proof}
 Directly by induction on the definition of $\incFun{\re}$.
\end{proof}

The constraint $0\leq\incFun{\re}$ is ensured by the definition of $\incFun{\re}$ because we are interested in investigating the upper bound,\footnote{$M$ as defined above, can be an arbitrary negative integer. For instance, the partial derivatives of $\sym\orop\asym$ \wrt~a symbol all have height 0, hence $M=-1$ in this case.}. The constraint $\incFun{\re}\leq 1$ corresponds to the intuition shown above, that needs to be proved.

The definition of $\incFun{\re}$ is of course driven by the reduction rules in \cref{fig:parDer}. We comment only the non trivial cases: if $\re=\re_0\orop\re_1$, then we expect $\theight{\re'_i}\leq\theight{\re_i}+1$, $i=0,1$,
hence $\theight{\re'_i}\leq\theight{\re}$ and $\incFun{\re}=0$;
if $\re=\starop{\re_0}$, then we expect $\theight{\re'_0}\leq\theight{\re_0}+1$,
hence $\theight{\re'_0\catop\starop{\re_0}}\leq\theight{\re}+1$ and $\incFun{\re}=1$. Finally, the definition for $\re=\re_0\catop\re_1$ requires more care:
if rule $\rn{l-cat}$ is applied, then $\theight{\re_0'\catop\re_1}=\theight{\re}+1$, but only when $\theight{\re_0'}=\theight{\re_0}+1$ and $\theight{\re_0}\geq\theight{\re_1}$, otherwise $\theight{\re_0'\catop\re_1}\leq\theight{\re}$; if rule $\rn{r-cat}$ is applied, then $\theight{\re'_1}\leq\theight{\re}$, as happens for rule $\rn{r-or}$. Therefore $\incFun{\re}=1$ iff $\incFun{\re_0}=1$ and $\theight{\re_0}\geq\theight{\re_1}$, that is, $\geqFun{\re_0}{\re_1}=1$. In all other cases, that is $\incFun{\re_0}=0$ or $\geqFun{\re_0}{\re_1}=0$, $\incFun{\re}=0$.

The following theorem proves the soundness of the definition of $\incFun{}$ \wrt~ its intended meaning. As a consequence, the height of the derivative of an expression $\re$ \wrt~ a symbol is always bounded by $\theight{\re}+1$.
\begin{theorem}\label[theorem]{theo:inc-bound}
 For all $\re,\re'\in\reSet$, $\sym\in\symAlph$, if $\re\parDer{\sym}\re'$, then $\theight{\re'}\leq \theight{\re}+\incFun{\re}$.
\end{theorem}
\begin{proof}
 By induction and case analysis on the rules defining $\re\parDer{\sym}\re'$.
 \begin{description}
  \item[base case:] The only base rule is \rn{sym}.
   Hence, $\re=\sym$, $\re'=\eps$, and
   $\theight{\eps}=0=\theight{\sym}=\theight{\sym}+0=\theight{\sym}+\incFun{\sym}$.
  \item[inductive step:] the proof proceeds by case analysis on the rule applied
   at the root of the derivation tree.
   \begin{itemize}
    \item $\Rule{l-cat}{\re_0\parDer{\sym}\re'_0}{\re_0\catop\re_1\parDer{\sym}\re'_0\catop\re_1}{}$\\[2ex]
          By inductive hypothesis $\theight{\re_0'}\leq \theight{\re_0}+\incFun{\re_0}$.
          We distinguish two cases:
          \begin{itemize}
           \item $\theight{\re_0}\geq\theight{\re_1}$:
                 $\theight{\re_0'\catop\re_1}\stackrel{\mathrm{def}}{=}\max(\theight{\re'_0},\theight{\re_1})+1\leq\max(\theight{\re_0}+\incFun{\re_0},\theight{\re_1})+1\leq\max(\theight{\re_0},\theight{\re_1})+1+\incFun{\re_0}=\theight{\re_0\catop\re_1}+\incFun{\re_0}=\theight{\re_0\catop\re_1}+\incFun{\re_0\catop\re_1}$, by the inductive hypothesis, the definition of $\max$, $\theight{\ }$, $\incFun{}$, and $\geqSym()$, the assumption $\theight{\re_0}\geq\theight{\re_1}$, and \cref{lemma:bounds}.

           \item $\theight{\re_0}<\theight{\re_1}$:
                 $\theight{\re_0'\catop\re_1}\stackrel{\mathrm{def}}{=}\max(\theight{\re'_0},\theight{\re_1})+1\leq\max(\theight{\re_0}+\incFun{\re_0},\theight{\re_1})+1=\theight{\re_0\catop\re_1}=\theight{\re_0\catop\re_1}+\incFun{\re_0\catop\re_1}$, by the inductive hypothesis, the definition of $\max$, $\theight{\ }$, $\incFun{}$, and $\geqSym()$, the assumption $\theight{\re_0}<\theight{\re_1}$, and \cref{lemma:bounds}.
          \end{itemize}

    \item $\Rule{r-cat}{\re_1\parDer{\sym}\re'_1}{\re_0\catop\re_1\parDer{\sym}\re'_1}{\hasEps{\re_0}=\eps}$\\[2ex]
          By inductive hypothesis $\theight{\re_1'}\leq \theight{\re_1}+\incFun{\re_1}$.

          Therefore, $\theight{\re_1'}\leq \theight{\re_1}+\incFun{\re_1}\leq\theight{\re_1}+1\leq\theight{\re_0\catop\re_1}\leq\theight{\re_0\catop\re_1}+\incFun{\re_0\catop\re_1}$, by the inductive hypothesis, the definition of $\max$, $\theight{\ },$ and $\incFun{}$, and \cref{lemma:bounds}.

    \item $\Rule{l-or}{\re_0\parDer{\sym}\re'_0}{\re_0\orop\re_1\parDer{\sym}\re'_0}{}$\\[2ex]
          By inductive hypothesis $\theight{\re_0'}\leq \theight{\re_0}+\incFun{\re_0}$.

          Therefore, $\theight{\re_0'}\leq \theight{\re_0}+\incFun{\re_0}\leq\theight{\re_0}+1\leq\theight{\re_0\orop\re_1}=\theight{\re_0\orop\re_1}+\incFun{\re_0\orop\re_1}$, by the inductive hypothesis, the definition of $\max$, $\theight{\ },$ and $\incFun{}$, and \cref{lemma:bounds}.
    \item rule \rn{r-or} is symmetric to  rule \rn{l-or}.
    \item $\Rule{star}{\re\parDer{\sym}\re'}{\starop{\re}\parDer{\sym}\re'\catop\starop{\re}}{}$\\[2ex]
          By inductive hypothesis $\theight{\re'}\leq \theight{\re}+\incFun{\re}$.

          Therefore, $\theight{\re'\catop\starop{\re}}\stackrel{\mathrm{def}}{=}\max(\theight{\re'},{\theight{\starop{\re}}})+1\leq \max(\theight{\re}+\incFun{\re},{\theight{\re}}+1)+1=\theight{\re}+1+1=\theight{\starop{\re}}+1=\theight{\starop{\re}}+\incFun{\starop{\re}}$, by the inductive hypothesis, the definition of $\max$, $\theight{\ },$ and $\incFun{}$, and \cref{lemma:bounds}.
   \end{itemize}
 \end{description}
\end{proof}

The following corollary can be directly derived from \cref{theo:inc-bound} and \cref{lemma:bounds}.
\begin{corollary}\label[corollary]{cor:bound}
 For all $\re,\re'\in\reSet$, and $\sym\in\symAlph$, if $\re\parDer{\sym}\re'$, then $\theight{\re'}\leq\theight{\re}+1$.
\end{corollary}
~\Cref{theo:inc-bound} shows that the height of the derivative of an expression $\re$ \wrt~ a symbol is bounded by $\theight{\re}+1$.
The following results prove a strong property: the height of the partial derivative of a partial derivative cannot increase; that is, after the first reduction step, the height of the derivative of an expression $\re$ \wrt~ a symbol is bounded by $\theight{\re}$.
Therefore, one can conclude that the height of the derivative of an expression $\re$ \wrt~ an arbitrary  word (not just a symbol) is bounded by $\theight{\re}+1$.
\begin{lemma}\label[lemma]{lemma:zero-inc}
 For all $\re,\re'\in\reSet$, and $\sym\in\symAlph$, if $\re\parDer{\sym}\re'$, then $\incFun{\re'}=0$.
\end{lemma}
\begin{proof}
 By induction and case analysis on the rules defining $\re\parDer{\sym}\re'$.
 \begin{description}
  \item[base case:] The only base rule is \rn{sym}.
   Hence, $\re=\sym$, $\re'=\eps$, and
   $\incFun{\eps}=0$.
  \item[inductive step:] the proof proceeds by case analysis on the rule applied
   at the root of the derivation tree.
   \begin{itemize}
    \item $\Rule{l-cat}{\re_0\parDer{\sym}\re'_0}{\re_0\catop\re_1\parDer{\sym}\re'_0\catop\re_1}{}$\\[2ex]
          If $\geqFun{\re_0}{\re_1}=1$, then by definition of $\incFun{}$ and inductive hypothesis, $\incFun{\re'_0\catop\re_1}=\incFun{\re'_0}=0$.
          If $\geqFun{\re_0}{\re_1}=0$, then by definition of $\incFun{}$ $\incFun{\re'_0\catop\re_1}=0$.

    \item $\Rule{r-cat}{\re_1\parDer{\sym}\re'_1}{\re_0\catop\re_1\parDer{\sym}\re'_1}{\hasEps{\re_0}=\eps}$\\[2ex]
          $\incFun{\re'_1}=0$ directly follows from the inductive hypothesis.

    \item the proof for rules \rn{l-or} and \rn{r-or} is the same as for rule \rn{r-cat}.

    \item the proof for rule \rn{star} is the same as for rule \rn{l-cat}.
   \end{itemize}
 \end{description}
\end{proof}

We can now generalize \Cref{theo:inc-bound} to derivatives \wrt~ any words.
\begin{theorem}\label[theorem]{theo:gen-inc-bound}
 For all $\re,\re'\in\reSet$, and $\word\in\symAlph^*$, if $\re\parDer{\word}\re'$, then $\theight{\re'}\leq\theight{\re}+\incFun{\re}$.
\end{theorem}
\begin{proof}
 By induction on the length of $\word$.
 \begin{description}
  \item[base case:] if $\word=\emptyWord$, then by definition $\re'=\re$, hence
   $\theight{\re'}=\theight{\re}$, and by \cref{lemma:bounds}, $\theight{\re'}=\theight{\re}\leq\theight{\re}+\incFun{\re}$.

  \item[inductive step:]
   if $\word=\sym\strcons\word'$ for some $\sym\in\symAlph$, $\word'\in\symAlph^*$, then by definition there exists $\re''\in\reSet$ s.t. $\re\parDer{\sym}\re''$ and $\re''\parDer{\word'}\re'$. By \cref{theo:inc-bound} $\theight{\re''}\leq\theight{\re}+\incFun{\re}$ and by inductive hypothesis $\theight{\re'}\leq\theight{\re''}+\incFun{\re''}$. Moreover, by \cref{lemma:zero-inc},
   $\incFun{\re''}=0$, therefore $\theight{\re'}\leq\theight{\re''}$ and, by transitivity,
   $\theight{\re'}\leq\theight{\re''}\leq\theight{\re}+\incFun{\re}$.
 \end{description}
\end{proof}

\Cref{lemma:bounds} allows to conclude the final result stating that the height of the derivatives of $\re$ is bounded by $\theight{\re}+1$.
\begin{corollary}\label[corollary]{cor:gen-inc-bound}
 For all $\re,\re'\in\reSet$, and $\word\in\symAlph^*$, if $\re\parDer{\word}\re'$, then $\theight{\re'}\leq\theight{\re}+1$.
\end{corollary}
\section{Regular expressions with shuffle}\label{sec:shuffle}

\subsection{Basic definitions}
In this section we extend the syntax of regular expressions by adding the shuffle operator\footnote{In the examples we assume that the shuffle has lower precedence than the concatenation and union operator.} $\re_0\shuffleop\re_1$ whose semantics is defined as follows by extending \cref{def:langOp} and \cref{def:sem}:
\[
 \begin{array}{l}
  \emptyWord \shuffleop \word = \word \shuffleop \emptyWord = \{\word\}                                                                                                      \\
  (\sym_1\word_1) \shuffleop (\sym_2\word_2) = \{ \sym_1  \} \strcatop (\word_1 \strshuffleop (\sym_2\word_2)) \cup \{\sym_2\}\strcatop((\sym_1\word_1)\strshuffleop\word_2) \\
  \lang_1\strshuffleop\lang_2=\bigcup_{\word_1\in\lang_1,\word_2\in\lang_2}(\word_1\strshuffleop\word_2)                                                                     \\[1ex]
  \sem{\re_0\shuffleop\re_1}=\sem{\re_0}\strshuffleop\sem{\re_1}
 \end{array}
\]
We denote with $\reSetExt$ the set of regular expressions extended with the shuffle operator.

Although it is well-known that the shuffle operator does not increase the abstract expressive power of regular expressions, in practice it is still very useful in RV to write much more compact and clear
specifications when correct system behaviors can be described as independently interleaved event traces. Indeed, it has been proved \cite{BrodaEtAl18,MayerElAl94} that there exists a family of regular expressions $\re$ with shuffle for which the equivalent NFA needs at least $2^{\tsize{\re}}$ states.

Let us consider for instance the distinct events $o_n$, $a_n$, and $c_n$, with the meaning
``file $n$ has been opened'', ``accessed'', and ``closed'', respectively, where $n=1,2$ is the corresponding file descriptor.
If the SUS is allowed to manage the two files independently, then a specification for the correct use of them can be defined quite concisely by the regular expression $o_1\catop\starop{a_1}\catop c_1\shuffleop o_2\catop\starop{a_2}\catop c_2$.

If, for simplicity, we assume that files can be accessed only once, and we generalize over the number of files, then the specification becomes the regular expression $o_1\catop a_1\catop c_1\shuffleop\ldots\shuffleop o_n\catop a_n \catop c_n$ over the alphabet of $3n$ distinct symbols $\{o_1,a_1,c_1,\dots, o_n,a_n,c_n\}$. For such an expression, an equivalent NFA must have at least
$4^n$ states.

\begin{figure}[h]
 $$
  \begin{array}{c}
   \Rule{shf}{\re_0\der{\sym}\re'_0\quad\re_1\der{\sym}\re'_1}{\re_0\shuffleop\re_1\der{\sym}(\re'_0\shuffleop\re_1)\orop(\re_0\shuffleop\re'_1)}{}\qquad \hasEps{\re_0\shuffleop\re_1}=\hasEps{\re_0}\conj\hasEps{\re_1} \\[4ex]
   \Rule{l-shf}{\re_0\parDer{\sym}\re'_0}{\re_0\shuffleop\re_1\parDer{\sym}\re'_0\shuffleop\re_1}{} \qquad
   \Rule{r-shf}{\re_1\parDer{\sym}\re'_1}{\re_0\shuffleop\re_1\parDer{\sym}\re_0\shuffleop\re'_1}{}
  \end{array}
 $$
 \caption{Transition rules defining the derivative and the partial derivatives for the shuffle operator}
 \label{fig:shfParDer}
\end{figure}

The transition rules defining the derivative and the partial derivatives for the shuffle operator
can be found in \cref{fig:shfParDer}. They all correspond to the intuition that events can be interleaved; moreover, the empty trace is contained in $\re_0\shuffleop\re_1$ iff it is contained in $\re_0$ and $\re_1$.

As happens for the other operators, also with the shuffle  the derivative is always defined, while the partial derivative is defined only for non-empty derivatives. For instance, if we assume $\sym_2\neq\sym_0$ and $\sym_2\neq\sym_1$, then
$\sym_0\shuffleop \sym_1\der{\sym_2}(\none\shuffleop\sym_1)\orop(\sym_0\shuffleop\none)$, and $\hasEps{(\none\shuffleop\sym_1)\orop(\sym_0\shuffleop\none)}=\none$, but there exists no $\re$ s.t.
$\sym_0\shuffleop \sym_1\parDer{\sym_2}\re$.

Theorems~\ref{theo:der} and \ref{theo:parDer} still hold along with  \cref{prop:parDer} and \cref{cor:parDer}, when $\reSetExt$ is considered.

\subsection{Height of partial derivatives with the shuffle operator}

We extend the result of \cref{theo:gen-inc-bound} and \cref{cor:gen-inc-bound} to the case of the shuffle operators.

Unfortunately the proofs are more challenging because the claim of \cref{lemma:zero-inc} no longer holds: There exist partial derivatives $\re$ \st~$\incFun{\re}>0$.
As a counter example, let us consider the following reduction steps computing the partial derivatives of $(\eps\shuffleop\starop{\sym})\catop(\asym\shuffleop\starop{\sym})$ \wrt~ $\sym\asym\sym$:
\begin{flushleft}
 $
  \begin{array}{l}
   \re_0\parDer{\sym}\re_1\parDer{\asym}\re_2\parDer{\sym}\re_3 \\
   \re_0=(\eps\shuffleop\starop{\sym})\catop(\asym\shuffleop\starop{\sym}) \qquad
   \re_1=(\eps\shuffleop\eps\catop\starop{\sym})\catop(\asym\shuffleop\starop{\sym})                                     \qquad
   \re_2=\eps\shuffleop\starop{\sym}                                       \qquad \re_3=\eps\shuffleop\eps\catop\starop{\sym}
  \end{array}
 $
\end{flushleft}
We have $\theight{\re_1}=\theight{\re_0}+1$ (first reduction step), but also $\theight{\re_3}=\theight{\re_2}+1$ (third reduction step). Therefore, necessarily $\incFun{\re_2}>0$, by definition of $\incFun{}$.

To prove the extended versions of the results of \cref{sec:parDer} we first extend the definition of $\incFun{}$ given in \cref{fig:incFun} with the case for the shuffle.
\[
 \incFun{\re_0\shuffleop\re_1}=\max(\geqFun{\re_0}{\re_1}\cdot\incFun{\re_0},\geqFun{\re_1}{\re_0}\cdot\incFun{\re_1})
\]
Before explaining the definition of $\incFun{}$ above, we note that the claim of \cref{lemma:bounds} holds also for $\reSetExt$ and can still be proved by induction on the definition of $\incFun{}$: $0\leq\incFun{\re}\leq 1$.
Consequently, if the height of one sub-expression $\re_i$ is strictly greater than the other $\re_{1-i}$, then only the partial derivative of $\re_i$ can contribute to the increment of the height of the partial derivative of $\re_0\shuffleop\re_1$, because increments are always bounded by 1.
Note that $\max(\incFun{\re_i},0)=\incFun{\re_i}$, since $\incFun{\re_i}\geq 0$, therefore $\incFun{\re_0\shuffleop\re_1}=\incFun{\re_i}$, if $\theight{\re_i}>\theight{\re_{1-i}}$ for $i=0,1$.

If $\theight{\re_0}=\theight{\re_1}$, then both the derivatives of $\re_0$ and $\re_1$ can contribute to the increment of the height of the partial derivative of $\re_0\shuffleop\re_1$.
Since \rn{l-shf} or \rn{r-shf} can be non-deterministically applied, the maximum increment
$\max(\incFun{\re_0},\incFun{\re_1})$ needs to be considered.

The correctness of the definition of $\incFun{\re_0\shuffleop\re_1}$ is still ensured by the claim of \cref{theo:inc-bound} extended to $\reSetExt$.

\subsubsection*{Proof of \cref{theo:inc-bound} extended to $\reSetExt$
}
For all $\re,\re'\in\reSetExt$, $\sym\in\symAlph$, if $\re\parDer{\sym}\re'$, then $\theight{\re'}\leq \theight{\re}+\incFun{\re}$.

\begin{proof}
 The structure of the proof is the same, but the two new rules for the shuffle operator need to be considered.
 \begin{itemize}
  \item $\Rule{l-shf}{\re_0\parDer{\sym}\re'_0}{\re_0\shuffleop\re_1\parDer{\sym}\re'_0\shuffleop\re_1}{}$\\[2ex]
        By inductive hypothesis $\theight{\re_0'}\leq \theight{\re_0}+\incFun{\re_0}$.
        We distinguish two cases:
        \begin{itemize}
         \item $\theight{\re_0}\geq\theight{\re_1}$:
               $\theight{\re_0'\shuffleop\re_1}\stackrel{\mathrm{def}}{=}\max(\theight{\re'_0},\theight{\re_1})+1\leq\max(\theight{\re_0}+\incFun{\re_0},\theight{\re_1})+1\leq\max(\theight{\re_0},\theight{\re_1})+1+\incFun{\re_0}\leq\max(\theight{\re_0},\theight{\re_1})+1+\incFun{\re_0\shuffleop\re_1}=\theight{\re_0\shuffleop\re_1}+\incFun{\re_0\shuffleop\re_1}$, where inequalities are derived from the inductive hypothesis, the definition of $\max$, $\theight{\ }$, $\incFun{}$, and $\geqSym()$, the assumption $\theight{\re_0}\geq\theight{\re_1}$, and \cref{lemma:bounds}.

         \item $\theight{\re_0}<\theight{\re_1}$ (that is, $\theight{\re_0}+\incFun{\re_0}\leq\theight{\re_1}$ by \cref{lemma:bounds}):
               $\theight{\re_0'\shuffleop\re_1}\stackrel{\mathrm{def}}{=}\max(\theight{\re'_0},\theight{\re_1})+1\leq\max(\theight{\re_0}+\incFun{\re_0},\theight{\re_1})+1=\theight{\re_0\shuffleop\re_1}\leq\theight{\re_0\shuffleop\re_1}+\incFun{\re_0\shuffleop\re_1}$, where inequalities are derived from the inductive hypothesis, the definition of $\max$, $\theight{\ }$, $\incFun{}$, and $\geqSym()$, the assumption $\theight{\re_0}<\theight{\re_1}$, and \cref{lemma:bounds}.
        \end{itemize}
  \item rule (r-shf) is symmetric to rule (l-shf).
 \end{itemize}
\end{proof}

Since both \cref{lemma:bounds} and \cref{theo:inc-bound} hold for $\reSetExt$, \cref{cor:bound} holds for $\reSetExt$ as well.

To prove the invariant of \cref{cor:inv} for $\reSetExt$, we modularize the proof by introducing two lemmas.

The first lemma is a weaker version of \cref{lemma:zero-inc}.

\begin{lemma}\label[lemma]{lemma:ext-zero-inc}
 For all $\re,\re'\in\reSetExt$, and $\sym\in\symAlph$, if $\re\parDer{\sym}\re'$ and $\theight{\re'}=\theight{\re}+1$, then $\incFun{\re'}=0$.
\end{lemma}
\begin{proof}
 By induction and case analysis on the rules defining $\re\parDer{\sym}\re'$.
 \begin{description}
  \item[base case:] The only base rule is \rn{sym}.
   The case is vacuous because $\re=\sym$, $\re'=\eps$, and $\theight{\sym}=\theight{\eps}=0$.
  \item[inductive step:] \hspace*{\fill}
   \begin{itemize}
    \item $\Rule{l-cat}{\re_0\parDer{\sym}\re'_0}{\re_0\catop\re_1\parDer{\sym}\re'_0\catop\re_1}{}$\\[2ex]
          From the hypothesis and the definition of $\theight{\ }$
          \begin{equation}
           \label{eq:l-cat-one}
           \max(\theight{\re_0'},\theight{\re_1})=\max(\theight{\re_0},\theight{\re_1})+1
          \end{equation}
          Therefore $\theight{\re_1}\leq\max(\theight{\re_0},\theight{\re_1})<\max(\theight{\re_0},\theight{\re_1})+1=\max(\theight{\re_0'},\theight{\re_1})$ by the definition of $\max$ and \cref{eq:l-cat-one}.

          Therefore, by the definition of $\max$
          \begin{equation}
           \label{eq:l-cat-two}
           \max(\theight{\re_0'},\theight{\re_1})=\theight{\re_0'}
          \end{equation}
          Hence $\theight{\re_0}+1\leq\max(\theight{\re_0},\theight{\re_1})+1=\theight{\re'_0}$ by the definition of $\max$, \cref{eq:l-cat-one} and \cref{eq:l-cat-two}.

          Moreover, by \cref{cor:bound} $\theight{\re_0'}\leq\theight{\re_0}+1$, therefore
          $\theight{\re_0'}=\theight{\re_0}+1$, and by inductive hypothesis $\incFun{\re'_0}=0$, which implies $\incFun{\re'_0\catop\re_1}=0$ by definition of $\incFun{}$.

    \item $\Rule{r-cat}{\re_1\parDer{\sym}\re'_1}{\re_0\catop\re_1\parDer{\sym}\re'_1}{\hasEps{\re_0}=\eps}$\\[2ex]
          This case is vacuous because $\theight{\re_1'}\leq\theight{\re_1}+1\leq\max(\theight{\re_0},\theight{\re_1})+1=\theight{\re_0\catop\re_1}$ by \cref{cor:bound} and the definition of $\max$ and $\theight{\ }$.

    \item The case for rules \rn{l-or} and \rn{r-or} is vacuous for the same reason shown for \rn{r-cat}.
    \item $\Rule{l-shf}{\re_0\parDer{\sym}\re'_0}{\re_0\shuffleop\re_1\parDer{\sym}\re'_0\shuffleop\re_1}{}$\\[2ex]
          The same proof for \rn{l-cat} shows that $\incFun{\re_0'}=\incFun{\re_0}+1$ and $\theight{\re_1}<\theight{\re'_0}$, therefore $\incFun{\re'_0}=0$ by inductive hypothesis, and $\incFun{\re'_0\shuffleop\re_1}=\max(1\cdot\incFun{\re_0'},0\cdot\incFun{\re_1})=0$ by $\theight{\re_1}<\theight{\re'_0}$ and the definition of $\max$, $\incFun{}$ and $\geqSym()$.
    \item rule \rn{r-shf} is symmetric to rule \rn{l-shf}.
    \item $\Rule{star}{\re\parDer{\sym}\re'}{\starop{\re}\parDer{\sym}\re'\catop\starop{\re}}{}$\\[2ex]
          By definition of $\theight{\ }$ and by the hypothesis $\theight{\re'\catop\starop{\re}}=\theight{\starop{\re}}+1$ we have
          $\max(\theight{\re'},\theight{\starop{\re}})+1=\theight{\starop{\re}}+1$, therefore
          $\max(\theight{\re'},\theight{\starop{\re}})=\theight{\starop{\re}}$.

          If $\theight{\re'}<\theight{\starop{\re}}$ then by the definition of $\incFun{}$ and $\geqSym()$ we have $\incFun{\re'\catop\starop{\re}}=\geqFun{\re'}{\starop{\re}}\cdot\incFun{\re'}=0\cdot\incFun{\re'}=0$.

          If $\theight{\re'}\geq\theight{\starop{\re}}$ then $\theight{\re'}=\theight{\starop{\re}}$, since $\max(\theight{\re'},\theight{\starop{\re}})=\theight{\starop{\re}}$.
          By the definition of $\theight{\ }$ we have $\theight{\re'}=\theight{\re}+1$, therefore
          by inductive hypothesis $\incFun{\re'}=0$ and by definition of $\incFun{}$ and $\geqSym()$ we have $\incFun{\re'\catop\starop{\re}}=\geqFun{\re'}{\starop{\re}}\cdot\incFun{\re'}=1\cdot\incFun{\re'}=0$.
   \end{itemize}
 \end{description}
\end{proof}

The second lemma establishes an invariant on $\incFun{}$ when the height of the derivatives does not change.

\begin{lemma}\label[lemma]{lemma:ext-leq-inc}
 For all $\re,\re'\in\reSetExt$, and $\sym\in\symAlph$, if $\re\parDer{\sym}\re'$ and $\theight{\re'}=\theight{\re}$, then $\incFun{\re'}\leq\incFun{\re}$.
\end{lemma}
\begin{proof}
 By induction and case analysis on the rules defining $\re\parDer{\sym}\re'$.
 \begin{description}
  \item[base case:] The only base rule is \rn{sym}.
   In this case $\re=\sym$, $\re'=\eps$, therefore $\incFun{\eps}=0=\incFun{\sym}$.
  \item[inductive step:] \hspace*{\fill}
   \begin{itemize}
    \item $\Rule{l-cat}{\re_0\parDer{\sym}\re'_0}{\re_0\catop\re_1\parDer{\sym}\re'_0\catop\re_1}{}$\\[2ex]
          From the hypothesis and the definition of $\theight{\ }$
          \begin{equation}
           \label{eq:l-cat-three}
           \max(\theight{\re_0'},\theight{\re_1})=\max(\theight{\re_0},\theight{\re_1})
          \end{equation}
          Two different cases may occur:
          \begin{itemize}
           \item $\theight{\re_0}\geq\theight{\re_1}$\\
                 From \cref{eq:l-cat-three} and the definition of $\max$, $\max(\theight{\re_0'},\theight{\re_1})=\theight{\re_0}$, hence $\theight{\re_0'}\leq\theight{\re_0}$ by the definition of $\max$.

                 If $\theight{\re_0'}<\theight{\re_0}$, then $\theight{\re_0'}<\theight{\re_1}$ by \cref{eq:l-cat-three} and the definition of $\max$. Therefore $\incFun{\re_0'\catop\re_1}=0\leq\incFun{\re_0\catop\re_1}$ by definition of $\incFun{}$ and \cref{lemma:bounds}.

                 Therefore $\theight{\re_1}\leq\max(\theight{\re_0},\theight{\re_1})<\max(\theight{\re_0},\theight{\re_1})+1=\max(\theight{\re_0'},\theight{\re_1})$ by the definition of $\max$ and \cref{eq:l-cat-one}.

                 If $\theight{\re_0'}=\theight{\re_0}$, then $\incFun{\re_0'}\leq\incFun{\re_0}$ by inductive hypothesis. Since $\theight{\re_0'}=\theight{\re_0}\geq\theight{\re_1}$, we have $\incFun{\re_0'\catop\re_1}=\incFun{\re_0'}\leq\incFun{\re_0}=\incFun{\re_0\catop\re_1}$ by the definition of $\incFun{}$ and $\geqSym()$.
           \item $\theight{\re_0}<\theight{\re_1}$\\
                 In this case $\incFun{\re_0\catop\re_1}=0$ by the definition of $\incFun{}$ and $\geqSym()$. Moreover, from \cref{eq:l-cat-three} $\max(\theight{\re_0'},\theight{\re_1})=\theight{\re_1}$, hence $\theight{\re_0'}\leq\theight{\re_1}$ by the definition of $\max$.

                 If $\theight{\re_0'}<\theight{\re_1}$ then $\incFun{\re_0'\catop\re_1}=0=\incFun{\re_0\catop\re_1}$ by the definition of $\incFun{}$ and $\geqSym()$. If $\theight{\re_0'}=\theight{\re_1}$ then $\theight{\re_0}<\theight{\re_1}=\theight{\re_0'}$, therefore $\theight{\re_0}+1\leq\theight{\re_0'}$. Moreover, by \cref{cor:bound} $\theight{\re'_0}\leq\theight{\re_0}+1$, hence $\theight{\re_0'}=\theight{\re_0}+1$, and, by \cref{lemma:ext-zero-inc}, $\incFun{\re_0'}=0$. Finally,
                 $\incFun{\re_0'\catop\re_1}=\incFun{\re_0'}=0=\incFun{\re_0\catop\re_1}$ by the definition of $\incFun{}$ and $\geqSym()$.
          \end{itemize}

    \item $\Rule{r-cat}{\re_1\parDer{\sym}\re'_1}{\re_0\catop\re_1\parDer{\sym}\re'_1}{\hasEps{\re_0}=\eps}$\\[2ex]
          From the hypothesis and the definition of $\theight{\ }$
          \begin{equation}
           \label{eq:l-cat}
           \theight{\re_1'}=\max(\theight{\re_0},\theight{\re_1})+1
          \end{equation}
          Two different cases may occur:
          \begin{itemize}
           \item $\theight{\re_0}\geq\theight{\re_1}$\\
                 From \cref{eq:l-cat} and the definition of $\max$, $\theight{\re_1'}=\max(\theight{\re_0},\theight{\re_1})+1\geq\theight{\re_1}+1$. Moreover, by \cref{lemma:bounds},
                 $\theight{\re_1'}\leq\theight{\re}+1$, therefore $\theight{\re_1'}=\theight{\re}+1$, hence by \cref{lemma:ext-zero-inc} $\incFun{\re_1'}=0$.
                 Finally, $\incFun{\re_1'}=0\leq\incFun{\re_0\catop\re_1}$ by \cref{lemma:bounds}.

           \item $\theight{\re_0}<\theight{\re_1}$\\
                 In this case $\incFun{\re_0\catop\re_1}=0$ by the definition of $\incFun{}$ and $\geqSym()$. Moreover, from \cref{eq:l-cat} and the definition of $\max$, $\theight{\re_1'}=\theight{\re_1}+1$, hence  by \cref{lemma:ext-zero-inc}, $\incFun{\re_1'}=0$. Finally,
                 $\incFun{\re_1'}=0\leq\incFun{\re_0\catop\re_1}$ by \cref{lemma:bounds}.
          \end{itemize}

    \item The proofs for the rules \rn{l-or} and \rn{r-or} are analogous to the proof for the rule \rn{r-cat}.
    \item $\Rule{l-shf}{\re_0\parDer{\sym}\re'_0}{\re_0\shuffleop\re_1\parDer{\sym}\re'_0\shuffleop\re_1}{}$\\[2ex]
          The proof is similar to that for \rn{l-cat}, except from some details on the definition of $\incFun{\re_0\shuffleop\re_1}$, which differs from that of $\incFun{\re_0\catop\re_1}$. We report it for the sake of completeness.

          From the hypothesis and the definition of $\theight{\ }$
          \begin{equation}
           \label{eq:l-shuffle}
           \max(\theight{\re_0'},\theight{\re_1})=\max(\theight{\re_0},\theight{\re_1})
          \end{equation}
          Two different cases may occur:
          \begin{itemize}
           \item $\theight{\re_0}\geq\theight{\re_1}$\\
                 From \cref{eq:l-shuffle} and the definition of $\max$, $\max(\theight{\re_0'},\theight{\re_1})=\theight{\re_0}$, hence $\theight{\re_0'}\leq\theight{\re_0}$ by the definition of $\max$.

                 If $\theight{\re_0'}<\theight{\re_0}$, then $\theight{\re_0'}<\theight{\re_1}=\theight{\re_0}$ by \cref{eq:l-shuffle} and the definition of $\max$. Therefore $\geqFun{\re_0'}{\re_1}= 0$ and $\geqFun{\re_1}{\re_0'}=\geqFun{\re_0}{\re_1}=\geqFun{\re_1}{\re_0}=1$ by the definition of $\geqSym()$, and
                 $\incFun{\re_0'\shuffleop\re_1}=\incFun{\re_1}\leq\max(\incFun{\re_0},\incFun{\re_1})=\incFun{\re_0\shuffleop\re_1}$ by the definition of $\incFun{}$ and $\max$.


                 If $\theight{\re_0'}=\theight{\re_0}$, then $\incFun{\re_0'}\leq\incFun{\re_0}$ by inductive hypothesis. Moreover,
                 $\geqFun{\re_0}{\re_1}=\geqFun{\re'_0}{\re_1}=1$ and $\geqFun{\re_1}{\re_0}=\geqFun{\re_1}{\re'_0}$ by the definition of $\geqSym()$.
                 Therefore, $\incFun{\re_0'\shuffleop\re_1}=\max(\incFun{\re'_0},\geqFun{\re_1}{\re_0'}\cdot\incFun{\re_1})\leq\max(\incFun{\re_0},\geqFun{\re_1}{\re_0}\cdot\incFun{\re_1})=\incFun{\re_0\shuffleop\re_1}$ by the definition of $\max$.
           \item $\theight{\re_0}<\theight{\re_1}$\\
                 In this case $\incFun{\re_0\shuffleop\re_1}=\incFun{\re_1}$ by the definition of $\incFun{}$ and $\geqSym()$. Moreover, from \cref{eq:l-shuffle} $\max(\theight{\re_0'},\theight{\re_1})=\theight{\re_1}$, hence $\theight{\re_0'}\leq\theight{\re_1}$ by the definition of $\max$.

                 If $\theight{\re_0'}<\theight{\re_1}$ then $\incFun{\re_0'\shuffleop\re_1}=\incFun{\re_1}=\incFun{\re_0\shuffleop\re_1}$ by the definition of $\incFun{}$ and $\geqSym()$. If $\theight{\re_0'}=\theight{\re_1}$ then $\theight{\re_0}<\theight{\re_1}=\theight{\re_0'}$, therefore $\theight{\re_0}+1\leq\theight{\re_0'}$. Moreover, by \cref{cor:bound} $\theight{\re'_0}\leq\theight{\re_0}+1$, hence $\theight{\re_0'}=\theight{\re_0}+1$, and, by \cref{lemma:ext-zero-inc}, $\incFun{\re_0'}=0$. Finally,
                 $\incFun{\re_0'\shuffleop\re_1}=\max(\incFun{\re_0'},\incFun{\re_1})=\max(0,\incFun{\re_1})=\incFun{\re_0\shuffleop\re_1}$ by the definition of $\incFun{}$ and $\geqSym()$.
          \end{itemize}


    \item The rule \rn{r-shf} is symmetric to \rn{l-shf}.
    \item $\Rule{star}{\re\parDer{\sym}\re'}{\starop{\re}\parDer{\sym}\re'\catop\starop{\re}}{}$\\[2ex]
          Directly by the definition of $\incFun{}$ and \cref{lemma:bounds}
          $\incFun{\re'\catop\starop{\re}}\leq 1=\incFun{\starop{\re}}$.
   \end{itemize}
 \end{description}
\end{proof}

\begin{theorem}\label[theorem]{theo:inv}
 For all $\re,\re'\in\reSetExt$, $\sym\in\symAlph$, if $\re\parDer{\sym}\re'$, then $\theight{\re'}\leq\theight{\re}+\incFun{\re}-\incFun{\re'}$.
\end{theorem}

\begin{proof}
 By \cref{cor:bound} (extended to $\reSetExt$) $\theight{\re'}\leq\theight{\re}+ 1$.
 Three cases are distinguished.
 \begin{itemize}
  \item $\theight{\re'}=\theight{\re}+1$: by \cref{lemma:ext-zero-inc}
        $\incFun{\re'}=0$, therefore by \cref{theo:inc-bound} (extended to $\reSetExt$) $\theight{\re'}\leq\theight{\re}+\incFun{\re}=\theight{\re}+\incFun{\re}-\incFun{\re'}$.
  \item $\theight{\re'}=\theight{\re}$: by \cref{lemma:ext-leq-inc} $\incFun{\re}-\incFun{\re'}\geq 0$, hence
        $\theight{\re'}=\theight{\re}\leq\theight{\re}+\incFun{\re}-\incFun{\re'}$.
  \item $\theight{\re'}<\theight{\re}$: by \cref{lemma:bounds} (extended to $\reSetExt$) $-1\leq\incFun{\re}-\incFun{\re'}$, hence
        $\theight{\re'}\leq\theight{\re}-1\leq\theight{\re}+\incFun{\re}-\incFun{\re'}$.
 \end{itemize}
\end{proof}

From \cref{theo:inv} and \cref{lemma:bounds} (extended to $\reSetExt$) we can
extend for free \cref{theo:gen-inc-bound} and \cref{cor:gen-inc-bound}  to $\reSetExt$.

\begin{theorem}
 For all $\re,\re'\in\reSetExt$, and $\word\in\symAlph^*$, if $\re\parDer{\word}\re'$, then $\theight{\re'}\leq\theight{\re}+\incFun{\re}-\incFun{\re'}$.
\end{theorem}

\begin{corollary}
 For all $\re,\re'\in\reSetExt$, and $\word\in\symAlph^*$, if $\re\parDer{\word}\re'$, then $\theight{\re'}\leq\theight{\re}+1$.
\end{corollary}

% It is now possible to prove \cref{theo:gen-inc-bound} extended to $\reSetExt$.

% \subsubsection*{Proof of \cref{theo:gen-inc-bound} extended to $\reSetExt$
% }
% For all $\re,\re'\in\reSetExt$, $\word\in\symAlph^*$, if $\re\parDer{\word}\re'$, then $\theight{\re'}\leq \theight{\re}+\incFun{\re}$.

% \begin{proof}
%  By induction on the length of $\word$.
%  \begin{description}
%   \item[base case:] if $\word=\emptyWord$, then by definition $\re'=\re$, hence
%    $\theight{\re'}=\theight{\re}$, and by \cref{lemma:bounds}, $\theight{\re'}=\theight{\re}\leq\theight{\re}+\incFun{\re}$.

%   \item[inductive step:]
%    if $\word=\sym\strcons\word'$ for some $\sym\in\symAlph$, $\word'\in\symAlph^*$, then by definition there exists $\re''\in\reSet$ s.t. $\re\parDer{\sym}\re''$ and $\re''\parDer{\word'}\re'$. By inductive hypothesis and by \cref{lemma:triang}, $\theight{\re'}\leq\theight{\re''}+\incFun{\re''}=\theight{\re''}-\theight{\re}+\incFun{\re''}+\theight{\re}\leq\theight{\re}+\incFun{\re}$.

%    %    Since by \cref{cor:bound} $\theight{\re''}\leq\theight{\re}+1$, we can distinguish three cases:
%    %    \begin{itemize}
%    %     \item $\theight{\re''}=\theight{\re}+1$: by \cref{lemma:ext-zero-inc}, $\incFun{\re''}=0$,
%    %           therefore $\theight{\re'}\leq\theight{\re}+\incFun{\re}+\incFun{\re''}=\theight{\re}+\incFun{\re}$.
%    %     \item $\theight{\re''}=\theight{\re}$: by inductive hypothesis and \cref{lemma:ext-leq-inc},
%    %           $\theight{\re'}\leq\theight{\re''}+\incFun{\re''}=\theight{\re}+\incFun{\re''}\leq\theight{\re}+\incFun{\re}$.
%    %     \item $\theight{\re''}<\theight{\re}$ (that is, $\theight{\re''}\leq\theight{\re}-1$): by inductive hypothesis and \cref{lemma:bounds},
%    %           $\theight{\re'}\leq\theight{\re''}+\incFun{\re''}\leq\theight{\re}-1+\incFun{\re''}\leq\theight{\re}\leq\theight{\re}+\incFun{\re}$.
%    %    \end{itemize}
%  \end{description}
% \end{proof}

% Finally, from the versions of \cref{theo:gen-inc-bound} and \cref{lemma:bounds}, both extended to $\reSetExt$, \cref{cor:gen-inc-bound} can be directly extended as well: For all $\re,\re'\in\reSetExt$, and $\word\in\symAlph^*$, if $\re\parDer{\word}\re'$, then $\theight{\re'}\leq\theight{\re}+1$.

\subsection{Size of partial derivatives with the shuffle operator}

Similar as done in the previous section, we extend the result of \cref{theo:gen-inc-bound-size} and \cref{cor:gen-inc-bound-size} to the case of the shuffle operators in the case of the size $\tsize{\re}$ of a regular expression $\re$.

Also for the size, the introduction of shuffle does not affect the space complexity of partial derivatives. However, in this case the proofs can be extended in an easier way, although the definition of $\incFunSize{\re_0\shuffleop\re_1}$ requires some care.

Indeed, one would be tempted to consider the following (incorrect) definition:
\[
 \incFunSize{\re_0\shuffleop\re_1}=\max(\incFunSize{\re_0},\incFunSize{\re_1})
 \qquad\qquad\qquad\qquad\mbox{(incorrect definition)}
\]
This is in line with $\incFunSize{\re_0\catop\re_1}$,
but, unfortunately, verifies only the weaker inequality
$\tsize{\re}\leq\tsize{\re_0\shuffleop\re_1}+\incFunSize{\re_0\shuffleop\re_1}$, if $\re_0\shuffleop\re_1\parDer{\sym}\re$, but not $\tsize{\re}\leq\tsize{\re_0\shuffleop\re_1}+\incFunSize{\re_0\shuffleop\re_1}-\incFunSize{\re}$.

To show this, let us consider the following reduction step computing the partial derivative of $\starop{\sym}\shuffleop\starop{\asym}$ \wrt~ $\sym$:
\begin{flushleft}
 $
  \begin{array}{l}
   \starop{\sym}\shuffleop\starop{\asym}\parDer{\sym}\eps\catop\starop{\sym}\shuffleop\starop{\asym} \\
   \tsize{\starop{\sym}\shuffleop\starop{\asym}}=5 \qquad
   \incFunSize{\starop{\sym}\shuffleop\starop{\asym}}=2 \qquad
   \tsize{\eps\catop\starop{\sym}\shuffleop\starop{\asym}}=7 \qquad
   \incFunSize{\eps\catop\starop{\sym}\shuffleop\starop{\asym}}=2
  \end{array}
 $
\end{flushleft}
We have $\tsize{\eps\catop\starop{\sym}\shuffleop\starop{\asym}}=7\leq 5+2=\tsize{\starop{\sym}\shuffleop\starop{\asym}}+\incFunSize{\starop{\sym}\shuffleop\starop{\asym}}$, but
$\tsize{\eps\catop\starop{\sym}\shuffleop\starop{\asym}}=7\not\leq 5+2-2=\tsize{\starop{\sym}\shuffleop\starop{\asym}}+\incFunSize{\starop{\sym}\shuffleop\starop{\asym}}-\incFunSize{\eps\catop\starop{\sym}\shuffleop\starop{\asym}}$.

To guarantee the invariant $\tsize{\re'}\leq\tsize{\re}+\incFunSize{\re}-\incFunSize{\re'}$, the following definition has to be considered:
\[
 \incFunSize{\re_0\shuffleop\re_1}=\incFunSize{\re_0}+\incFunSize{\re_1} \qquad\qquad\qquad\qquad\mbox{(correct definition)}
\]
This new definition takes into account increments due to multiple reduction steps. In this way, we obtain $\tsize{\eps\catop\starop{\sym}\shuffleop\starop{\asym}}=7\leq 5+4-2=\tsize{\starop{\sym}\shuffleop\starop{\asym}}+\incFunSize{\starop{\sym}\shuffleop\starop{\asym}}-\incFunSize{\eps\catop\starop{\sym}\shuffleop\starop{\asym}}$.

The correctness of the definition of $\incFunSize{\re_0\shuffleop\re_1}$ is ensured by the claim of \cref{theo:inc-bound-size} extended to $\reSetExt$.

\subsubsection*{Proof of \cref{theo:inc-bound-size} extended to $\reSetExt$
}
For all $\re,\re'\in\reSetExt$, $\sym\in\symAlph$, if $\re\parDer{\sym}\re'$, then $\tsize{\re'}\leq \theight{\re}+\incFunSize{\re}-\incFunSize{\re'}$.

\begin{proof}
 The structure of the proof is the same, but the two new rules for the shuffle operator need to be considered.
 \begin{itemize}
  \item $\Rule{l-shf}{\re_0\parDer{\sym}\re'_0}{\re_0\shuffleop\re_1\parDer{\sym}\re'_0\shuffleop\re_1}{}$\\[2ex]
        By inductive hypothesis $\tsize{\re_0'}\leq \tsize{\re_0}+\incFunSize{\re_0}-\incFunSize{\re_0'}$.

        We have $\tsize{\re'_0\shuffleop\re_1}=\tsize{\re_0'}+\tsize{\re_1}+1\leq\tsize{\re_0}+\incFunSize{\re_0}-\incFunSize{\re_0'}+\tsize{\re_1}+1=\tsize{\re_0\shuffleop\re_1}+\incFunSize{\re_0\shuffleop\re_1}-\incFunSize{\re_1}-\incFunSize{\re_0'\shuffleop\re_1}+\incFunSize{\re_1}=\tsize{\re_0\shuffleop\re_1}+\incFunSize{\re_0\shuffleop\re_1}-\incFunSize{\re_0'\shuffleop\re_1}$, by the inductive hypothesis, the definition of $\tsize{\ }$, and $\incFunSize{}$.
  \item rule (r-shf) is symmetric to rule (l-shf).
 \end{itemize}
\end{proof}

From \cref{theo:inc-bound-size} and \cref{lemma:bounds-size} (extended to $\reSetExt$) we can
extend for free \cref{theo:gen-inc-bound-size} and \cref{cor:gen-inc-bound-size}  to $\reSetExt$.

\begin{theorem}
 For all $\re,\re'\in\reSetExt$, and $\word\in\symAlph^*$, if $\re\parDer{\word}\re'$, then $\tsize{\re'}\leq\tsize{\re}+\incFunSize{\re}-\incFunSize{\re'}$.
\end{theorem}

\begin{corollary}
 For all $\re,\re'\in\reSetExt$, and $\word\in\symAlph^*$, if $\re\parDer{\word}\re'$, then $\tsize{\re'}\leq\tsize{\re}+\tsize{\re}^2$. Hence $\tsize{\re'}$ is $O(\tsize{\re}^2)$.
\end{corollary}
\bibliography{main}
\end{document}
