%% The first command in your LaTeX source must be the \documentclass command.
%%
%% Options:
%% twocolumn : Two column layout.
%% hf: enable header and footer.
\documentclass{ceurart}

%%
%% One can fix some overfulls
\sloppy

%%
%% Minted listings support 
%% Need pygment <http://pygments.org/> <http://pypi.python.org/pypi/Pygments>
\usepackage{listings}
%% auto break lines
\lstset{breaklines=true,mathescape=true,basicstyle={\ttfamily\small},keywordstyle={\ttfamily\bfseries}}
\usepackage{centernot}
\usepackage{graphicx}
\usepackage{proof} % compact proof trees for examples
\usepackage{hyperref} % links
\usepackage{verbatim} % multiline comments
\usepackage[dvipsnames]{xcolor}
\usepackage{mathtools}
\usepackage{amssymb}
\usepackage{stmaryrd}
%\usepackage{xspace}
\usepackage{color}
\usepackage{amssymb}
\usepackage{pifont}
%\usepackage{amsmath}
%\usepackage{amsthm}
\usepackage{pifont}
\usepackage{stmaryrd}
%\usepackage{xspace}
\usepackage{listings}
\usepackage{color}
\usepackage{pgfplots}
\usepackage{subcaption}
\usepackage{cleveref}
%
%%
%% end of the preamble, start of the body of the document source.

% editing macros
\newif\ifsubmit
\submitfalse
%\submittrue

\ifsubmit
 \newcommand{\DA}[1]{{#1}}
 \newcommand{\DAComm}[1]{}
 \newcommand{\AF}[1]{{#1}}
 \newcommand{\AFComm}[1]{}
\else
 \newcommand{\DA}[1]{\textcolor{blue}{#1}}
 \newcommand{\DAComm}[1]{{\scriptsize\textcolor{blue}{[\bf{Angelo: }#1}]}}
 \newcommand{\AF}[1]{\textcolor{red}{#1}}
 \newcommand{\AFComm}[1]{{\scriptsize\textcolor{red}{[\bf{Angelo: }#1}]}}
\fi

%abbreviations
\newcommand{\wrt}{w.~r.~t.}
\newcommand{\st}{s.~t.}

%environments

\newtheorem{definition}{Definition}[section]
\newtheorem{theorem}[definition]{Theorem}
\newtheorem{lemma}[definition]{Lemma}
\newtheorem{corollary}[definition]{Corollary}
\newtheorem{proposition}[definition]{Proposition}

\newenvironment{proof}{\textbf{Proof:}\\}{\hspace*{\fill}$\Box$}

%matematiche
\newcommand{\dom}[1]{\mathit{dom}(#1)}
\newcommand{\Tuple}[1]    {({#1})}
\newcommand{\Pair}[2]     {\Tuple{{#1},{#2}}}
%\newcommand{\Triple}[3]     {\Tuple{{#1},{#2},{#3}}}
%\newcommand{\fv}[1]{\textit{fv}(#1)}
%\newcommand{\vars}[1]{\textit{vars}(#1)}
\newcommand{\simpleRule}[3]{\displaystyle\frac{#1}{#2}\ \begin{array}{l} #3 \end{array}}
\newcommand{\rn}[1]{\textsc{({#1})}} %% rule name
\newcommand{\Rule}[4]{{\tiny{\rn{#1}}}\displaystyle\frac{#2}{#3}\ \begin{array}{l} #4 \end{array}}
\newcommand{\infeRule}[3]{\infer[\tiny\textsc{({#1})}]{#3}{#2}}

%syntax
\newenvironment{grammar}{$\begin{array}[t]{llll}}{\end{array}$}
\newcommand{\production}[3]{#1&{:}{:}=&#2 & \mbox{{\small{#3}}}}
\newcommand{\nlproduction}[1]{&&#1}
\newcommand{\term}[1]{\ensuremath{\mathtt {#1}}}
\newcommand{\nonterm}[1]{\ensuremath{\mathit {#1}}}
\newcommand{\re}{\nonterm{e}}
\newcommand{\reSet}{\nonterm{RE}}
\newcommand{\reSetExt}{\nonterm{RE+}}
\newcommand{\eps}{\term{\epsilon}}
\newcommand{\none}{\term{0}}
\newcommand{\sym}{\term{a}}
\newcommand{\asym}{\term{b}}
\newcommand{\yasym}{\term{c}}
\newcommand{\all}{\term{1}}
\newcommand{\orop}{\mathbin{+}}
\newcommand{\shuffleop}{\mathbin{||}}
\newcommand{\catop}{}
\newcommand{\strcons}{\mathbin{:}}
\newcommand{\strcatop}{\mathbin{\cdot}}
\newcommand{\strshuffleop}{\shuffleop}
\newcommand{\andop}{\mathbin{\&}}
\newcommand{\starop}[1]{{#1}*}
\newcommand{\symAlph}{\Sigma}

%derivative
\newcommand{\der}[1]{\xrightarrow{#1}}
\newcommand{\hasEps}[1]{\nu({#1})}
\newcommand{\conj}{\mathbin{\mathsf{and}}}
\newcommand{\disj}{\mathbin{\mathsf{or}}}
\newcommand{\noDetDer}[2]{\xhookrightarrow{#1}_{#2}}
\newcommand{\parDer}[1]{\xrightharpoonup{#1}}
\newcommand{\noParDer}[1]{\centernot{\parDer{#1}}}
\newcommand{\derSet}[1]{\xRightarrow{#1}}
\newcommand{\emptyPath}{\lambda}
\newcommand{\pathCons}{\mathbin{:}}
\newcommand{\tpath}{p} %% generic tree path
\newcommand{\pathEl}{b} %% generic path element
\newcommand{\incSym}{\Delta_{\max}}
\newcommand{\incSymSize}{\eta_{\max}}
\newcommand{\incFun}[1]{{\incSym}({#1})}
\newcommand{\incFunSize}[1]{{\incSymSize}({#1})}
\newcommand{\incConst}{\mathit{k}}
\newcommand{\theight}[1]{{\mid}{#1}{\mid}} %% tree height
\newcommand{\tsizeSym}{{\mid\!\mid}}
\newcommand{\tsize}[1]{{\tsizeSym}{#1}{\tsizeSym}} %% tree height
\newcommand{\gtSym}{\mathit{gt}}
\newcommand{\gtFun}[2]{\gtSym({#1},{#2})}
\newcommand{\geqSym}{\mathit{geq}}
\newcommand{\geqFun}[2]{\geqSym({#1},{#2})}

%semantics
\newcommand{\word}{\nonterm{w}}
\newcommand{\emptyWord}{\lambda}
\newcommand{\sem}[1]{\left\llbracket{#1}\right\rrbracket}
\newcommand{\opSem}[2]{\left\llbracket{#1}\right\rrbracket_{#2}}
\newcommand{\wordUniv}{\starop{\symAlph}}
\newcommand{\first}[1]{\nonterm{first}(#1)}
\newcommand{\lang}{\nonterm{L}}


\begin{document}

\conference{ICTCS 2025}

%
\title{On The Space Complexity of Partial Derivatives of Regular Expressions with Shuffle}

\author[1]{Davide Ancona}[orcid=0000-0002-6297-2011,email=davide.ancona@unige.it]

\cormark[1]

\address[1]{University of Genova, Italy}

\author[2]{Angelo Ferrando}[orcid=0000-0002-8711-4670,
 email=angelo.ferrando@unimore.it
]

\address[2]{University of Modena and Reggio Emilia, Italy}


%% Footnotes
\cortext[1]{Corresponding author.}

\begin{keywords}
 partial derivatives \sep
 regular expressions with shuffle \sep
 rewriting-based runtime verification \sep
 space complexity
\end{keywords}

\maketitle
%
\begin{abstract}
 Partial derivatives of regular expressions are an elegant formalism introduced by Antimirov to define an algorithm that generates, from regular expressions, equivalent non-deterministic finite automata (NFA) with a limited number of states.

 While Antimirov's work addresses the classical problem of word recognition in regular languages, here we are interested in runtime verification (RV) of simple properties expressible with regular expressions. In this context, the words to be recognized are finite traces of monitorable events, which define the alphabet of the language, and the corresponding NFA generated from the regular expression may have an intractable number of states.

 This often occurs because the properties to be verified consist of sub-traces of mutually independent events, which are allowed to interleave during the execution of the system under scrutiny (SUS). For this reason, regular expressions used for RV are extended with the shuffle operator. Although regular languages are closed under shuffle, the operator allows for more readable and compact specifications.

 By exploiting partial derivatives, it is possible to follow a rewriting-based approach to RV, where only a single partial derivative needs to be stored at each reduction step, instead of generating a finite automaton with an intractable number of states.

 This raises the question of the space complexity of the largest partial derivative that can be generated. While the upper bound on the total number of generated partial derivatives has been shown to be linear in the size of the initial regular expression, no results can be found in the literature regarding the size of the largest derivative.

 In this paper, we investigate this problem with respect to two different metrics of regular expressions: their height and their total number of nodes when considered as trees. In particular, we show that the height of the largest partial derivative can increase by at most one, while the number of nodes of the largest partial derivative is bounded by $n^2$, where $n$ is the number of nodes of the initial regular expression.

 Surprisingly, these results still hold when regular expressions are extended with shuffle.
\end{abstract}
%
%
\section{Derivatives for regular expressions: a standard deterministic definition}\label{regExp}

\begin{figure}[h]
 \begin{small}
  \begin{center}
   \begin{grammar}
    \production{\re}{\none \mid \eps \mid \sym \mid \re_0\catop\re_1 \mid \re_0\orop\re_1 \mid \starop{\re}}{with $\sym\in\symAlph$} \\
   \end{grammar}
  \end{center}
 \end{small}
 \caption{Syntax of regular expressions}
 \label{regExpSyn}
\end{figure}

\begin{definition}
 \[
  \word^{-1}(\lang) = \{ \word' \mid \word\strcatop\word'\in\lang \}
 \]
\end{definition}

\begin{figure}[h]
 \begin{center}
  $$
   \begin{array}{c}
    \Rule{empty}{}{\none\der{\sym}\none}\quad
    \Rule{eps}{}{\eps\der{\sym}\none}\quad
    \Rule{sym-eq}{}{\sym\der{\sym}\eps}\quad
    \Rule{sym-neq}{}{\sym\der{\asym}\none}{}                                 \\[4ex]
    \Rule{or}{\re_0\der{\sym}\re'_0\quad\re_1\der{\sym}\re'_1}{\re_0\catop\re_1\der{\sym}\re'_0\catop\re_1\orop\hasEps{\re_0}\catop\re'_1}{} \quad
    \Rule{cat}{\re_0\der{\sym}\re'_0\quad\re_1\der{\sym}\re'_1}{\re_0\orop\re_1\der{\sym}\re'_0\orop\re'_1}{}\quad
    \Rule{star}{\re\der{\sym}\re'}{\starop{\re}\der{\sym}\re'\starop{\re}}{} \\[8ex]
    \hasEps{\eps}=\hasEps{\starop{\re_0}}=\eps\qquad\hasEps{\none}=\hasEps{\sym}=\none \quad
    \hasEps{\re_0\catop\re_1}=\hasEps{\re_0}\conj\hasEps{\re_1}\quad
    \hasEps{\re_0\orop\re_1}=\hasEps{\re_0}\disj\hasEps{\re_1}               \\[2ex]
    %% \eps\conj\eps=\eps\quad\none\conj\re=\re\conj\none=\none\qquad\none\disj\none=\none\quad\eps\disj\re=\re\disj\eps=\eps
   \end{array}
  $$
  \begin{tabular}{ccc}
   \begin{tabular}{|c|c|c|}
    \hline
    $\conj$ & $\quad\none\quad$ & $\quad\eps\quad$ \\
    \hline
    $\none$ & $\none$           & $\none$          \\
    \hline
    $\eps$  & $\none$           & $\eps$           \\
    \hline
   \end{tabular}
    & \qquad &
   \begin{tabular}{|c|c|c|}
    \hline
    $\disj$ & $\quad\none\quad$ & $\quad\eps\quad$ \\
    \hline
    $\none$ & $\none$           & $\eps$           \\
    \hline
    $\eps$  & $\eps$            & $\eps$           \\
    \hline
   \end{tabular}
  \end{tabular}
 \end{center}

 \caption{Transition system defining the derivatives of regular expressions}
 \label{deriv}
\end{figure}

\begin{definition}
 \[
  \begin{array}{l}
   \re\der{\emptyWord}\re \\
   \re_0\der{\sym\strcatop\word}\re_1 \mbox{ iff } \re_0\der{\sym}\re \mbox{ and } \re\der{\word}\re_1
   \\
   \opSem{\re}{\der{}}=\{\word\in\wordUniv \mid \mbox{there exists } \re' \mbox{ s.t. } \re\der{\word}\re' \mbox{ and } \hasEps{\re'}=\eps\}
  \end{array}
 \]
\end{definition}

~\Cref{deriv} contains the transition rules defining the derivatives of regular expressions.
Except for the presentation style, the definition coincides with that introduced by Brzozowski~\cite{Brzozowski64}. By using the notation of Antimirov, we have
that $\sym^{-1}(\re_0)=\re_1$ iff $\re_0\der{\sym}\re_1$.


\begin{definition}
 \[
  \begin{array}{l}
   \lang_1\strcatop\lang_2=\{\word_1\strcatop\word_2\mid \word_1\in\lang_1,\word_2\in\lang_2\} \\
   \lang^0=\{\emptyWord\}, \lang^{n+1}=\lang\strcatop\lang^n \mbox{for all $n\geq 0$}          \\
   \starop{\lang} = \bigcup_{n\geq 0}\lang^n
  \end{array}
 \]
\end{definition}
\begin{definition}
 \[
  \begin{array}{l}
   \sem{\none}=\emptyset\quad \sem{\eps}=\{\emptyWord\} \quad \sem{\sym}=\{\sym\} \quad
   \sem{\re_0\catop\re_1}=\sem{\re_0}\strcatop\sem{\re_1} \quad \sem{\re_0\orop\re_1}=\sem{\re_0}\cup\sem{\re_1} \\ \sem{\starop{\re_0}}=\starop{\sem{\re_0}}
  \end{array}
 \]
\end{definition}

\begin{theorem}
 For all regular expressions $\re$
 \[
  \sem{\re}=\opSem{\re}{\der{}}
 \]
\end{theorem}


\section{Partial derivatives of regular expressions}\label{sec:parDer}

\begin{figure}
 $$
  \begin{array}{c}
   \Rule{sym}{}{\sym\parDer{\sym}\eps}\quad
   \Rule{l-cat}{\re_0\parDer{\sym}\re'_0}{\re_0\catop\re_1\parDer{\sym}\re'_0\catop\re_1}{}\quad
   \Rule{r-cat}{\re_1\parDer{\sym}\re'_1}{\re_0\catop\re_1\parDer{\sym}\re'_1}{\hasEps{\re_0}=\eps} \\[4ex]
   \Rule{l-or}{\re_0\parDer{\sym}\re'_0}{\re_0\orop\re_1\parDer{\sym}\re'_0}{} \quad
   \Rule{r-or}{\re_1\parDer{\sym}\re'_1}{\re_0\orop\re_1\parDer{\sym}\re'_1}{} \quad
   \Rule{star}{\re\parDer{\sym}\re'}{\starop{\re}\parDer{\sym}\re'\starop{\re}}{}
  \end{array}
 $$
 \caption{Transition system defining the partial derivatives of regular expressions}
 \label{parDer}
\end{figure}

~\Cref{parDer} contains the transition rules defining the partial derivatives of regular expressions.
As for the definition of derivatives, the presentation deviates from that of Antimirov \cite{Antimirov96}, and keeps following the transition system style. Besides the different notation, more importantly, here we deliberately define a non-deterministic system, where a single transition step yields a single (non-zero) partial derivative among all possible ones, while in Antimirov's definition the set of all partial derivatives is returned.

For instance, according to Antimirov's definition\footnote{See \cite{Antimirov96}, definition 2.8.},
the partial derivative of $ab \orop ac$ w.r.t. $a$ returns the set of regular expressions $\{b,c\}$ (that is, $\delta_a(ab \orop ac)=\{b,c\}$ with Antimirov's notation). This corresponds to the fact that there exist two possible transition steps from $ab \orop ac$ labeled with $a$, namely,
$ab \orop ac\parDer{a}b$ and $ab \orop ac\parDer{a}c$.

This difference is directly related to the opposite aim of our work: while Antimirov's approach uses partial derivatives for generating an NFA, here
we use them to define a non-deterministic transition system which already represents the corresponding NFA.

\begin{definition}
 For all $\re,\re_0,\re_1\in\reSet$, $\sym\in\symAlph$, $\word\in\symAlph^*$
 \[
  \begin{array}{l}
   \\
   \re\parDer{\emptyWord}\re                                                                                       \\
   \re_0\parDer{\sym\strcons\word}\re_1 \mbox{ iff } \re_0\parDer{\sym}{}\re \mbox{ and } \re\parDer{\word}{}\re_1 \\
   \opSem{\re}{\parDer{}}=\{\word\in\wordUniv \mid \mbox{there exists } \re'\in\reSet \mbox{ s.t. } \re\parDer{\word}\re' \mbox{ and } \hasEps{\re'}=\eps\}
  \end{array}
 \]
\end{definition}


\begin{proposition}\label[proposition]{prop:parDer}
 Let $\delta_\word(\re)$ denotes the set of all partial derivatives of $\re\in\reSet$ w.r.t. $\word\in\symAlph^*$.

 Then $\delta_\word(\re)=\{\re'\in\reSet \mid \re\parDer{\word}\re'\}$.
\end{proposition}

% \begin{definition}
%  Let $\reSet_1$ and $\reSet_2$ be sets of regular expressions.
%  \[
%   \begin{array}{l}
%    \reSet_1\derSet{\sym}\reSet_2 \mbox{ iff } \reSet_2=\{\re_1 \mid \mbox{ there exists } \re_0\in\reSet_1 \mbox{ s.t. } \re_0\parDer{\sym}{}\re_1 \} \\
%    \hasEps{\reSet}=\eps \mbox{ iff there exists } \re\in\reSet \mbox{ s.t. } \hasEps{\re}                                                             \\
%    \re_0\derSet{\sym\word}\re_1 \mbox{ iff } \re_0\derSet{\sym}\re \mbox{ and } \re\derSet{\word}\re_1                                                \\
%    \re\derSet{\emptyWord}{}\re                                                                                                                        \\
%    \opSem{\re}{\derSet{}}=\{\word\in\wordUniv \mid \mbox{there exists } \reSet \mbox{ s.t. } \{\re\}\derSet{\word}\reSet \mbox{ and } \hasEps{\reSet}=\eps\}
%   \end{array}
%  \]
% \end{definition}

\begin{theorem}\label[theorem]{theo:parDer}
 For all $\re\in\reSet$
 \[
  \opSem{\re}{\parDer{}{}}=\opSem{\re}{\der{}}
 \]
\end{theorem}

\begin{corollary}\label[corollary]{cor:parDer}
 For all $\re\in\reSet$
 \[
  \opSem{\re}{\parDer{}{}}=\sem{\re}
 \]
\end{corollary}
% \begin{theorem}
%  For all regular expressions $\re$
%  \[
%   \opSem{\re}{\derSet{}}=\opSem{\re}{\der{}}
%  \]
% \end{theorem}

\begin{figure}
 $$
  \begin{array}{c}
   \theight{\eps}=\theight{\sym}=0                                                              \\
   \theight{\re_0\catop\re_1}=\theight{\re_0\orop\re_1}=\max(\theight{\re_0},\theight{\re_1})+1 \\
   \theight{\starop{\re_0}}=\theight{\re_0}+1
  \end{array}
 $$
 \caption{Definition of the height of regular expressions}
 \label{fig:height}
\end{figure}

\begin{figure}
 $$
  \begin{array}{l}
   % \gtFun{\re_0}{\re_1}=
   % \begin{cases}
   %   1, & \text{if } \theight{\re_0} > \theight{\re_1} \\
   %   0, & \text{ otherwise}
   % \end{cases} \quad
   \geqFun{\re_0}{\re_1}=
   \begin{cases}
    1, & \text{if } \theight{\re_0} \geq \theight{\re_1} \\
    0, & \text{if } \theight{\re_0} < \theight{\re_1}
   \end{cases} \\[4ex]
   \incFun{\sym}=\incFun{\eps}=0                                      \qquad
   \incFun{\re_0\catop\re_1}=\geqFun{\re_0}{\re_1}\cdot\incFun{\re_0} \qquad
   \incFun{\re_0\orop\re_1}=0                                         \qquad
   \incFun{\starop{\re_0}}=1
  \end{array}
 $$
 \caption{Definition of the increment function $\incFun{\re}$}
 \label{fig:incFun}
\end{figure}

\begin{lemma}\label[lemma]{lemma:bounds}
 For all $\re\in\reSet$
 \[0\leq\incFun{\re}\leq 1\]
\end{lemma}
\begin{proof}
 Directly by induction on the definition of $\incFun{\re}$.
\end{proof}

The following theorem proves the soundness of the definition of $\incFun{}$~w.r.t. its intended meaning. As a consequence, the height of the derivative of an expression $\re$ w.r.t. a symbol is always bounded by $\theight{\re}+1$.
\begin{theorem}\label[theorem]{theo:inc-bound}
 For all $\re,\re'\in\reSet$, $\sym\in\symAlph$, if $\re\parDer{\sym}\re'$, then $\theight{\re'}\leq \theight{\re}+\incFun{\re}$.
\end{theorem}
\begin{proof}
 By induction and case analysis on the rules defining $\re\parDer{\sym}\re'$.
 \begin{description}
  \item[base case:] The only base rule is \rn{sym}.
   Hence, $\re=\sym$, $\re'=\eps$, and
   $\theight{\eps}=0=\theight{\sym}=\theight{\sym}+0=\theight{\sym}+\incFun{\sym}$.
  \item[inductive step:] the proof proceeds by case analysis on the rule applied
   at the root of the derivation tree.
   \begin{itemize}
    \item $\Rule{l-cat}{\re_0\parDer{\sym}\re'_0}{\re_0\catop\re_1\parDer{\sym}\re'_0\catop\re_1}{}$\\[2ex]
          By inductive hypothesis $\theight{\re_0'}\leq \theight{\re_0}+\incFun{\re_0}$.
          We distinguish two cases:
          \begin{itemize}
           \item $\theight{\re_0}\geq\theight{\re_1}$:
                 $\theight{\re_0'\catop\re_1}\stackrel{\mathrm{def}}{=}\max(\theight{\re'_0},\theight{\re_1})+1\leq\max(\theight{\re_0}+\incFun{\re_0},\theight{\re_1})+1\leq\max(\theight{\re_0},\theight{\re_1})+1+\incFun{\re_0}=\theight{\re_0\catop\re_1}+\incFun{\re_0}=\theight{\re_0\catop\re_1}+\incFun{\re_0\catop\re_1}$, by the inductive hypothesis, the definition of $\max$, $\theight{\ }$, $\incFun{}$, and $\geqSym()$, the assumption $\theight{\re_0}\geq\theight{\re_1}$, and \cref{lemma:bounds}.

           \item $\theight{\re_0}<\theight{\re_1}$:
                 $\theight{\re_0'\catop\re_1}\stackrel{\mathrm{def}}{=}\max(\theight{\re'_0},\theight{\re_1})+1\leq\max(\theight{\re_0}+\incFun{\re_0},\theight{\re_1})+1=\theight{\re_0\catop\re_1}=\theight{\re_0\catop\re_1}+\incFun{\re_0\catop\re_1}$, by the inductive hypothesis, the definition of $\max$, $\theight{\ }$, $\incFun{}$, and $\geqSym()$, the assumption $\theight{\re_0}<\theight{\re_1}$, and \cref{lemma:bounds}.
          \end{itemize}

    \item $\Rule{r-cat}{\re_1\parDer{\sym}\re'_1}{\re_0\catop\re_1\parDer{\sym}\re'_1}{\hasEps{\re_0}=\eps}$\\[2ex]
          By inductive hypothesis $\theight{\re_1'}\leq \theight{\re_1}+\incFun{\re_1}$.

          Therefore, $\theight{\re_1'}\leq \theight{\re_1}+\incFun{\re_1}\leq\theight{\re_1}+1\leq\theight{\re_0\catop\re_1}\leq\theight{\re_0\catop\re_1}+\incFun{\re_0\catop\re_1}$, by the inductive hypothesis, the definition of $\max$, $\theight{\ },$ and $\incFun{}$, and \cref{lemma:bounds}.

    \item $\Rule{l-or}{\re_0\parDer{\sym}\re'_0}{\re_0\orop\re_1\parDer{\sym}\re'_0}{}$\\[2ex]
          By inductive hypothesis $\theight{\re_0'}\leq \theight{\re_0}+\incFun{\re_0}$.

          Therefore, $\theight{\re_0'}\leq \theight{\re_0}+\incFun{\re_0}\leq\theight{\re_0}+1\leq\theight{\re_0\orop\re_1}=\theight{\re_0\orop\re_1}+\incFun{\re_0\orop\re_1}$, by the inductive hypothesis, the definition of $\max$, $\theight{\ },$ and $\incFun{}$, and \cref{lemma:bounds}.
    \item rule \rn{r-or} is symmetric to  rule \rn{l-or}.
    \item $\Rule{star}{\re\parDer{\sym}\re'}{\starop{\re}\parDer{\sym}\re'\starop{\re}}{}$\\[2ex]
          By inductive hypothesis $\theight{\re'}\leq \theight{\re}+\incFun{\re}$.

          Therefore, $\theight{\re'\catop\starop{\re}}\stackrel{\mathrm{def}}{=}\max(\theight{\re'},{\theight{\starop{\re}}})+1\leq \max(\theight{\re}+\incFun{\re},{\theight{\re}}+1)+1=\theight{\re}+1+1=\theight{\starop{\re}}+1=\theight{\starop{\re}}+\incFun{\starop{\re}}$, by the inductive hypothesis, the definition of $\max$, $\theight{\ },$ and $\incFun{}$, and \cref{lemma:bounds}.
   \end{itemize}
 \end{description}
\end{proof}

The following corollary can be directly derived from \cref{theo:inc-bound} and \cref{lemma:bounds}.
\begin{corollary}\label[corollary]{cor:bound}
 For all $\re,\re'\in\reSet$, and $\sym\in\symAlph$, if $\re\parDer{\sym}\re'$, then $\theight{\re'}\leq\theight{\re}+1$.
\end{corollary}
~\Cref{theo:inc-bound} shows that the height of the derivative of an expression $\re$ w.r.t. a symbol is bounded by $\theight{\re}+1$.
The following theorems prove a stronger property: after the first reduction step, the height of the derivative of an expression $\re$ w.r.t. a symbol is bounded by $\theight{\re}$.
Therefore, one can conclude that the height of the derivative of an expression $\re$~w.r.t. an arbitrary  word (not just a symbol) is bounded by $\theight{\re}+1$.
\begin{lemma}\label[lemma]{lemma:zero-inc}
 For all $\re,\re'\in\reSet$, and $\sym\in\symAlph$, if $\re\parDer{\sym}\re'$, then $\incFun{\re'}=0$.
\end{lemma}
\begin{proof}
 By induction and case analysis on the rules defining $\re\parDer{\sym}\re'$.
 \begin{description}
  \item[base case:] The only base rule is \rn{sym}.
   Hence, $\re=\sym$, $\re'=\eps$, and
   $\incFun{\eps}=0$.
  \item[inductive step:] the proof proceeds by case analysis on the rule applied
   at the root of the derivation tree.
   \begin{itemize}
    \item $\Rule{l-cat}{\re_0\parDer{\sym}\re'_0}{\re_0\catop\re_1\parDer{\sym}\re'_0\catop\re_1}{}$\\[2ex]
          If $\geqFun{\re_0}{\re_1}=1$, then by definition of $\incFun{}$ and inductive hypothesis, $\incFun{\re'_0\catop\re_1}=\incFun{\re'_0}=0$.
          If $\geqFun{\re_0}{\re_1}=0$, then by definition of $\incFun{}$ $\incFun{\re'_0\catop\re_1}=0$.

    \item $\Rule{r-cat}{\re_1\parDer{\sym}\re'_1}{\re_0\catop\re_1\parDer{\sym}\re'_1}{\hasEps{\re_0}=\eps}$\\[2ex]
          $\incFun{\re'_1}=0$ directly follows from the inductive hypothesis.

    \item the proof for rules \rn{l-or} and \rn{r-or} is the same as for rule \rn{r-cat}.

    \item the proof for rule \rn{star} is the same as for rule \rn{l-cat}.
   \end{itemize}
 \end{description}
\end{proof}

We can now generalize \Cref{theo:inc-bound} to derivatives w.r.t. any words.
\begin{theorem}\label[theorem]{theo:gen-inc-bound}
 For all $\re,\re'\in\reSet$, and $\word\in\symAlph^*$, if $\re\parDer{\word}\re'$, then $\theight{\re'}\leq\theight{\re}+\incFun{\re}$.
\end{theorem}
\begin{proof}
 By induction on the length of $\word$.
 \begin{description}
  \item[base case:] if $\word=\emptyWord$, then by definition $\re'=\re$, hence
   $\theight{\re'}=\theight{\re}$, and by \cref{lemma:bounds}, $\theight{\re'}=\theight{\re}\leq\theight{\re}+\incFun{\re}$.

  \item[inductive step:]
   if $\word=\sym\strcons\word'$ for some $\sym\in\symAlph$, $\word'\in\symAlph^*$, then by definition there exists $\re''\in\reSet$ s.t. $\re\parDer{\sym}\re''$ and $\re''\parDer{\word'}\re'$. By \cref{theo:inc-bound} $\theight{\re''}\leq\theight{\re}+\incFun{\re}$ and by inductive hypothesis $\theight{\re'}\leq\theight{\re''}+\incFun{\re''}$. Moreover, by \cref{lemma:zero-inc},
   $\incFun{\re''}=0$, therefore $\theight{\re'}\leq\theight{\re''}$ and, by transitivity,
   $\theight{\re'}\leq\theight{\re''}\leq\theight{\re}+\incFun{\re}$.
 \end{description}
\end{proof}

\Cref{lemma:bounds} allows to conclude the final result stating that the height of the derivatives of $\re$ is bounded by $\theight{\re}+1$.
\begin{corollary}\label[corollary]{cor:gen-inc-bound}
 For all $\re,\re'\in\reSet$, and $\word\in\symAlph^*$, if $\re\parDer{\word}\re'$, then $\theight{\re'}\leq\theight{\re}+1$.
\end{corollary}
\section{Regular expressions with shuffle}\label{sec:shuffle}

\subsection{Basic definitions}
In this section we extend the syntax of regular expressions by adding the shuffle operator\footnote{In the examples we assume that the shuffle has lower precedence than the concatenation and union operator.} $\re_0\shuffleop\re_1$ whose semantics is defined as follows by extending \cref{def:langOp} and \cref{def:sem}:
\[
 \begin{array}{l}
  \emptyWord \shuffleop \word = \word \shuffleop \emptyWord = \{\word\}                                                                                                      \\
  (\sym_1\word_1) \shuffleop (\sym_2\word_2) = \{ \sym_1  \} \strcatop (\word_1 \strshuffleop (\sym_2\word_2)) \cup \{\sym_2\}\strcatop((\sym_1\word_1)\strshuffleop\word_2) \\
  \lang_1\strshuffleop\lang_2=\bigcup_{\word_1\in\lang_1,\word_2\in\lang_2}(\word_1\strshuffleop\word_2)                                                                     \\[1ex]
  \sem{\re_0\shuffleop\re_1}=\sem{\re_0}\strshuffleop\sem{\re_1}
 \end{array}
\]
Although it is well-known that the shuffle operator does not increase the abstract expressive power of regular expressions, in practice it is still very useful in RV to write more compact and clear
specifications when correct system behaviors can be described as independently interleaved event traces.

Let us consider for instance the events $o_n$, $a_n$, and $c_n$, with the meaning
``file $n$ has been opened'', ``accessed'', and ``closed'', respectively, where $n=0,1$ is the corresponding file descriptor.
If the SUS is allowed to manage the two files independently, then a specification for the correct use of them can be defined quite concisely by the regular expression $o_0\catop\starop{a_0}\catop c_0\shuffleop o_1\catop\starop{a_1}\catop c_1$.

\begin{figure}[h]
 $$
  \begin{array}{c}
   \Rule{shf}{\re_0\der{\sym}\re'_0\quad\re_1\der{\sym}\re'_1}{\re_0\shuffleop\re_1\der{\sym}(\re'_0\shuffleop\re_1)\orop(\re_0\shuffleop\re'_1)}{}\qquad \hasEps{\re_0\shuffleop\re_1}=\hasEps{\re_0}\conj\hasEps{\re_1} \\[4ex]
   \Rule{l-shf}{\re_0\parDer{\sym}\re'_0}{\re_0\shuffleop\re_1\parDer{\sym}\re'_0\shuffleop\re_1}{} \qquad
   \Rule{r-shf}{\re_1\parDer{\sym}\re'_1}{\re_0\shuffleop\re_1\parDer{\sym}\re_0\shuffleop\re'_1}{}
  \end{array}
 $$
 \caption{Transition rules defining the derivative and the partial derivatives for the shuffle operator}
 \label{fig:shfParDer}
\end{figure}

The transition rules defining the derivative and the partial derivatives for the shuffle operator
can be found in \cref{fig:shfParDer}. They all correspond to the intuition that events can be interleaved; moreover, the empty trace is contained in $\re_0\shuffleop\re_1$ iff it is contained in $\re_0$ and $\re_1$.

As happens for the other operators, also with the shuffle  the derivative is always defined, while the partial derivative is defined only for non-zero derivatives. For instance, if we assume $\sym_2\neq\sym_0$ and $\sym_2\neq\sym_1$, then
$\sym_0\shuffleop \sym_1\der{\sym_2}(\none\shuffleop\sym_1)\orop(\sym_0\shuffleop\none)$, and $\hasEps{(\none\shuffleop\sym_1)\orop(\sym_0\shuffleop\none)}=\none$, but there exists no $\re$ s.t.
$\sym_0\shuffleop \sym_1\parDer{\sym_2}\re$.

Theorems~\ref{theo:der} and \ref{theo:parDer} still hold along with  \cref{prop:parDer} and \cref{cor:parDer}, when extended with the shuffle operator.

\subsection{Height of partial derivatives with the shuffle operator}

Theorems~\ref{theo:inc-bound} and \ref{theo:zero-inc} show  that the height of a derivative can grow by one only in the first transition
step. . This fact allows us to derive that
\bibliography{main}
\end{document}
