\section{Derivatives of regular expressions}\label{sec:der}

This section provides the basic notions on derivatives, by assuming familiarity with regular expressions.

Let $\symAlph$ denote a given non-empty finite set of symbols $\symAlph$, called \emph{alphabet}. We call \emph{word} an element of $\symAlph^*$, and denote with $\emptyWord$ the empty word. Furthermore, $\sym\strcons\word$ denotes the word where $\sym$ is the first symbol, and $\word$ the rest of the word, while $\word\strcatop\word'$ denotes word concatenation.
The set $\reSet$ of regular expressions over $\symAlph$ is defined by the grammar in \cref{fig:syn}. We use the standard precedence rules: the Kleene star operator $\starop{\re}$  has higher precedence than concatenation $\re_0\catop\re_1$, which in turn has higher precedence than union $\re_0\orop\re_1$.
\begin{figure}[h]
 \begin{small}
  \begin{center}
   \begin{grammar}
    \production{\re}{\none \mid \eps \mid \sym \mid \re_0\catop\re_1 \mid \re_0\orop\re_1 \mid \starop{\re}}{with $\sym\in\symAlph$} \\
   \end{grammar}
  \end{center}
 \end{small}
 \caption{Syntax of regular expressions}
 \label{fig:syn}
\end{figure}

\Cref{def:langOp} introduces the standard operators on regular languages, and \cref{def:sem} the standard semantics of regular expressions.

\begin{definition}\label[definition]{def:langOp}
 For all $\lang,\lang_0,\lang_1\subseteq\symAlph^*$
 \[
  \begin{array}{l}
   \lang_1\strcatop\lang_2=\{\word_1\strcatop\word_2\mid \word_1\in\lang_1,\word_2\in\lang_2\} \\
   \lang^0=\{\emptyWord\}, \lang^{n+1}=\lang\strcons\lang^n \mbox{for all $n\geq 0$}           \\
   \starop{\lang} = \bigcup_{n\geq 0}\lang^n
  \end{array}
 \]
\end{definition}
\begin{definition}\label[definition]{def:sem}
 \[
  \begin{array}{l}
   \sem{\none}=\emptyset\quad \sem{\eps}=\{\emptyWord\} \quad \sem{\sym}=\{\sym\} \quad
   \sem{\re_0\catop\re_1}=\sem{\re_0}\strcatop\sem{\re_1} \quad \sem{\re_0\orop\re_1}=\sem{\re_0}\cup\sem{\re_1} \\ \sem{\starop{\re_0}}=\starop{\sem{\re_0}}
  \end{array}
 \]
\end{definition}

Note the difference between $\eps$ (a constant in the syntax of regular expressions) and $\emptyWord$ (the empty word).

The derivative of a language $L$ \wrt~a word $\word\in\symAlph^*$, denoted by $\word^{-1}(\lang)$, is defined as follows:
\begin{definition}\label[definition]{def:der}
 For all $L\subseteq\symAlph^*$, $\word^{-1}(\lang)$
 \[
  \word^{-1}(\lang) = \{ \word' \mid \word\strcatop\word'\in\lang \}
 \]
\end{definition}
Directly from \cref{def:der} one can derive
\begin{equation}
 \label{eq:der}
 \word\in\lang \mbox{ iff }
 \emptyWord\in\word^{-1}(\lang)
\end{equation}

While the notion of derivative applies to any formal language, regular languages enjoy the following closure property: if $\lang$ is regular, then
$\word^{-1}(\lang)$ is regular as well, for any word $\word$. In other words, the derivative of a regular expression can be defined by another regular expression.

This property the definition of the derivatives of regular expressions in an elegant way.
First, the derivative \wrt~a single symbol is defined (see~\cref{fig:der}), then
the more general notion of derivative \wrt~a word is derived by reflexive  transitive closure (see~\cref{def:der}).

Here we deliberately follow the style of a labelled transition system, defined by means of inference rules, to highlight two interesting related aspects:
\begin{itemize}
 \item derivatives provide an intuitive way to define a small-step semantics for regular expressions;
 \item a reduction step $\re\der{\sym}\re'$ corresponds to a transition step from state $\re$ to $\re'$ with symbol $\sym$, for an automaton which recognizes the language defined by $\re$.
\end{itemize}
By using the notation of Antimirov~\cite{Antimirov96}, which is an adaptation of that introduced by Brzozowski~\cite{Brzozowski64}, we have
that $\sym^{-1}(\re_0)=\re_1$ iff $\re_0\der{\sym}\re_1$, and $\word^{-1}(\re_0)=\re_1$ iff $\re_0\der{\word}\re_1$.

The definition of $\hasEps{\re}$ in~\cref{fig:der} deserves some comments:
the intended meaning is that $\hasEps{\re}=\eps$ iff
$\emptyWord\in\sem{\re}$, and, dually, $\hasEps{\re}=\none$ iff
$\emptyWord\not\in\sem{\re}$. The base cases of the definition are straightforward; thanks to the auxiliary functions $\conj$, $\disj$ defined on $\eps$ and $\none$ in the corresponding tables, we can deduce that, as expected, $\re_0\catop\re_1$ recognizes $\emptyWord$ iff both $\re_0$ and $\re_1$ recognize $\emptyWord$, while  $\re_0\orop\re_1$ recognizes $\emptyWord$ iff $\re_0$ or $\re_1$ recognizes $\emptyWord$.

The fact that $\hasEps{\re}$ does not return a Boolean value was a deliberate choice of Brzozowski to define rule \rn{cat} in a more compact way:
if $\hasEps{\re_0}=\eps$, then $\re_0\catop\re_1\der{\sym}\re'_0\orop\eps\catop\re_1'$, since in this case $\re_1$ contributes to the
derivative. Otherwise, $\re_0\catop\re_1\der{\sym}\re'_0\orop\none\catop\re_1'$, since $\re_1$ does not contribute to the
derivative.

Note that by definition of the rules, for all $\re$ and $\sym$, there always exists a unique derivative of $\re$ \wrt~$\sym$.
For instance, if $\sym\neq\asym$, $\sym\neq\yasym$, then
$\sym \asym \orop \sym\yasym\der{\sym}(\eps\catop\asym\orop\none\catop\none)\orop(\eps\catop\yasym\orop\none\catop\none)$ and $\sym \asym \orop \sym \yasym\der{\asym}(\none\catop\asym\orop\none\catop\none)\orop(\none\catop\yasym\orop\none\catop\eps)$.
\begin{figure}[t]
 \begin{center}
  $$
   \begin{array}{c}
    \Rule{empty}{}{\none\der{\sym}\none}\quad
    \Rule{eps}{}{\eps\der{\sym}\none}\quad
    \Rule{sym-eq}{}{\sym\der{\sym}\eps}\quad
    \Rule{sym-neq}{}{\sym\der{\asym}\none}{\sym\neq\asym}                          \\[4ex]
    \Rule{cat}{\re_0\der{\sym}\re'_0\quad\re_1\der{\sym}\re'_1}{\re_0\catop\re_1\der{\sym}\re'_0\catop\re_1\orop\hasEps{\re_0}\catop\re'_1}{} \quad
    \Rule{or}{\re_0\der{\sym}\re'_0\quad\re_1\der{\sym}\re'_1}{\re_0\orop\re_1\der{\sym}\re'_0\orop\re'_1}{}\quad
    \Rule{star}{\re\der{\sym}\re'}{\starop{\re}\der{\sym}\re'\catop\starop{\re}}{} \\[8ex]
    \hasEps{\eps}=\hasEps{\starop{\re_0}}=\eps\qquad\hasEps{\none}=\hasEps{\sym}=\none \quad
    \hasEps{\re_0\catop\re_1}=\hasEps{\re_0}\conj\hasEps{\re_1}\quad
    \hasEps{\re_0\orop\re_1}=\hasEps{\re_0}\disj\hasEps{\re_1}                     \\[2ex]
    %% \eps\conj\eps=\eps\quad\none\conj\re=\re\conj\none=\none\qquad\none\disj\none=\none\quad\eps\disj\re=\re\disj\eps=\eps
   \end{array}
  $$
  \begin{tabular}{ccc}
   \begin{tabular}{|c|c|c|}
    \hline
    $\conj$ & $\quad\none\quad$ & $\quad\eps\quad$ \\
    \hline
    $\none$ & $\none$           & $\none$          \\
    \hline
    $\eps$  & $\none$           & $\eps$           \\
    \hline
   \end{tabular}
    & \qquad &
   \begin{tabular}{|c|c|c|}
    \hline
    $\disj$ & $\quad\none\quad$ & $\quad\eps\quad$ \\
    \hline
    $\none$ & $\none$           & $\eps$           \\
    \hline
    $\eps$  & $\eps$            & $\eps$           \\
    \hline
   \end{tabular}
  \end{tabular}
 \end{center}

 \caption{Transition system defining the derivatives of regular expressions}
 \label{fig:der}
\end{figure}

\Cref{def:der} introduces the general notion of derivative \wrt~a word $\word$, by computing the reflexive transitive closure of the one-step relation defined in~\cref{fig:der}. As expected, the definition is by induction on the length of $\word$. The definition introduces also the operational semantics of regular expressions \wrt~ derivatives, driven by~\cref{eq:der}. In the view above, which considers the labelled transition system as a deterministic automaton,
the definition corresponds to that of accepted language, where the set of final states is the set of all derivatives $\re'$ \st~$\hasEps{\re'}=\eps$.
It is straightforward to prove that the set of all syntactically distinct derivatives of a regular expression $\re$ is unbounded. Hence, the underlying automaton is infinite, unless some finite quotient algebra of states modulo a suitable congruence is considered.
\begin{definition}\label[definition]{def:der}
 For all $\re,\re_0,\re_1\in\reSet$, $\sym\in\symAlph$, $\word\in\symAlph^*$
 \[
  \begin{array}{l}
   \re\der{\emptyWord}\re \qquad
   \re_0\der{\sym\strcons\word}\re_1 \mbox{ iff } \re_0\der{\sym}\re \mbox{ and } \re\der{\word}\re_1
   \\
   \opSem{\re}{\der{}}=\{\word\in\symAlph^* \mid \mbox{there exists } \re'\in\reSet \mbox{ s.t. } \re\der{\word}\re' \mbox{ and } \hasEps{\re'}=\eps\}
  \end{array}
 \]
\end{definition}


\Cref{theo:der} establishes the results proved by Brzozowski:
the definitions of $\hasEps{\re}$ and $\re\der{\word}\re'$ are sound
\wrt\ their intended semantics, and the operational semantics of regular expressions is equivalent to the standard one.
\begin{theorem}\label[theorem]{theo:der}
 For all $\re,\re'\in\reSet$, $\word\in\symAlph^*$
 \[
  \begin{array}{l}
   \hasEps{\re} = \eps \mbox{ iff } \emptyWord\in\sem{\re}              \\
   \re\der{\word}\re' \mbox{ implies } \word^{-1}(\sem{\re})=\sem{\re'} \\
   \opSem{\re}{\der{}}=\sem{\re}
  \end{array}
 \]
\end{theorem}

